\chapter{Eine Ungleichung für die Kondition von Vandermonde Matrizen}
In diesem Abschnitt sei ein $n$-elementiger Vektor $z = (z_0, \dots, z_{n-1}) \in \Cn$ gegeben.
Die zugehörige Vandermonde-Matrix sei mit $V \defeq \Vand{z}$ bezeichnet.
Ziel ist es, eine Abschätzung für die Norm der Inversen $\norm{V^{-1}}$
und damit für die Kondition $\cond{V}$
zu finden.

\begin{mydef}[Lagrange-Polynome]
    Für $ z = (z_0, \dots, z_{n-1}) \in \Cn $ definiere die
    \emph{Lagrange-Polynome} als
    \[
        l_j(z) = \prod_{\substack{r=0\\ r \neq j}}^{n-1} \frac{z - z_r}{z_j - z_r}, \; j = 0, \dots, n-1.
    \]
\end{mydef}

\begin{remark}
    Es gilt $l_j \in \Pi_{n-1}$, wobei $\Pi_{n-1}$ den Polynom-Raum vom Grad $n-1$ bezeichne.
    Einfaches Nachrechnen liefert $l_j(z_r) = \delta_{jr}$,
    wobei $\delta_{jr}$ das Kronecker-Delta bezeichne.
\end{remark}

Wegen $l_j \in \Pi_{n-1}$, können die Lagrange-Polynome ausmultipliziert als
\[
    l_j(z) = \sum_{r = 0}^{n-1} u_{jr} z^{r}
\]
mit Koeffizienten $u_{jr} \in \C$ geschrieben werden.
Es zeigt sich nun
\begin{lemma}
    Sei $z = (z_0, \dots, z_{n-1}) \in \Cn$ und seien
    $l_j$, $j = 0, \dots, n-1$ die zugehörigen Lagrange-Polynome
    mit Koeffizienten $u_{jr}$ wie oben.
    Dann ist die Koeffizienten-Matrix $U = (u_{jr})_{j,r = 0}^{n-1}$ die
    Inverse der Vandermonde-Matrix $V = \Vand{z} = (z_{j}^{k})_{k,j = 0}^{n-1}$.
\end{lemma}

\begin{proof}
    Betrachte das Gleichungssystem $V \alpha = y$ mit $\alpha = (\alpha_0,
    \dots, \alpha_{n-1}) \in \Cn$ und $y = (y_0, \dots, y_{n-1}) \in \Cn$.
    Zum Beweis muss die Gleichung $U y = \alpha$ gezeigt werden.
    Es gilt für $j = 0, \dots, n-1$
    \[
        \begin{split}
            \sum_{k=0}^{n-1} u_{jk} y_k &= \sum_{k=0}^{n-1} u_{jk} \sum_{r=0}^{n-1} z_r^k \alpha_r = \sum_{r=0}^{n-1} \alpha_r \sum_{k=0}^{n-1} u_{jk} z_r^k\\
                                        &= \sum_{r=0}^{n-1} \alpha_r l_j(z_r) = \sum_{r=0}^{n-1} \alpha_r \delta_{jr} = \alpha_j,
        \end{split}
    \]
    wie behauptet.
\end{proof}

\section{Vandermonde Matrizen mit reellen Stützstellen}
