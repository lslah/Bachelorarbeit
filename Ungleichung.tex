\chapter{Eine Ungleichung für die Kondition von Vandermonde-Matrizen}
In diesem Abschnitt sei ein $n$-elementiger Vektor $z = (z_0, \dots, z_{n-1})
\in \Cn$ gegeben.  Die zugehörige Vandermonde-Matrix sei mit $V \defeq
\Vand{z}$ bezeichnet.  Ziel ist es, eine Abschätzung der Zeilensummennorm
inverser Vandermonde-Matrizen und damit eine Abschätzung der
Kondition $\cond[\infty]{V}$ zu finden.
% TODO: better reference
Wir orientieren uns dabei an der Herleitung aus \cite{gautschi1}.

\section{Inversion der Vandermonde-Matrix}
Es werden zunächst die \emph{Lagrange-Polynome} zu den Knoten
$z_0, \dots, z_{n-1}$ eingeführt, deren Koeffizienten sich als Einträge der
inversen Vandermonde-Matrix herausstellen werden.

\begin{mydef}[Lagrange-Polynome]
    Für $ z = (z_0, \dots, z_{n-1}) \in \Cn $ definiere die
    \emph{Lagrange-Polynome} durch
    \[
        l_j(z)
        \defeq \prod_{\substack{r=0\\ r \neq j}}^{n-1} \frac{z - z_r}{z_j - z_r} \text{ für } j = 0, \dots, n-1.
    \]
\end{mydef}

\begin{remark}
    $ \; $
    \begin{enumerate}
        \item Es gilt $l_j \in \Pi_{n-1}$, wobei $\Pi_{n-1}$ den Raum der Polynome bis
        Grad $n-1$ bezeichne.
        \item  Einfaches Nachrechnen liefert
        $l_j(z_r) = \delta_{jr}$, wobei $\delta_{jr}$ das Kronecker-Delta
        sei.
    \end{enumerate}
\end{remark}

\noindent Wegen $l_j \in \Pi_{n-1}$ können die Lagrange-Polynome ausmultipliziert als
\begin{equation}
    \label{eq:lagrange}
    l_j(z) = \sum_{r = 0}^{n-1} u_{jr} z^{r}
\end{equation}
mit den Koeffizienten $u_{jr} \in \C$ geschrieben werden.
Im folgenden Lemma zeigt sich, dass diese Koeffizienten genau den Einträgen der
inversen Vandermonde-Matrix entsprechen.
\begin{lemma}[\cite{gautschi3}]
    \label{lemma:vandermonde-inversion}
    Sei $z = (z_0, \dots, z_{n-1}) \in \Cn$ und seien
    $l_j$ die zugehörigen Lagrange-Polynome für $j = 0, \dots, n-1$
    mit den Koeffizienten $u_{jr}$ wie in (\ref{eq:lagrange}).
    Dann ist die Koeffizienten-Matrix $U = (u_{jr})_{j,r = 0}^{n-1}$ die
    Inverse der Vandermonde-Matrix $V \defeq (z_{j}^{k})_{k,j = 0}^{n-1}$.
\end{lemma}
\begin{proof}
    Wir betrachten das Gleichungssystem $V \alpha = y$ mit
    $\alpha = (\alpha_0, \dots, \alpha_{n-1}) \in \Cn$
    und $y = (y_0, \dots, y_{n-1}) \in \Cn$.
    Zum Beweis muss die Gleichung $U y = \alpha$ für alle $\alpha,  y \in \Cn$
    gezeigt werden.
    Es gilt für $j = 0, \dots, n-1$
    \[
        \begin{split}
            \sum_{k=0}^{n-1} u_{jk} y_k &= \sum_{k=0}^{n-1} u_{jk} \sum_{r=0}^{n-1} z_r^k \alpha_r = \sum_{r=0}^{n-1} \alpha_r \sum_{k=0}^{n-1} u_{jk} z_r^k\\
                                        &= \sum_{r=0}^{n-1} \alpha_r l_j(z_r) = \sum_{r=0}^{n-1} \alpha_r \delta_{jr} = \alpha_j,
        \end{split}
    \]
    wie behauptet.
\end{proof}

\noindent Unter Verwendung von Gleichung (\ref{eq:lagrange}) und den
elementarsymmetrischen Polynomen kann eine explizite Darstellung der
inversen Vandermonde-Matrix angegeben werden.
Dazu schreiben wir für $j = 0, \dots, n-1$
\begin{equation*}
    \sum_{r = 0}^{n-1} u_{jr} z^{r}
    = l_j(z)
    = \prod_{\substack{r=0\\ r \neq j}}^{n-1} \frac{z - z_r}{z_j - z_r}
    = \Pi_j \cdot \prod_{\substack{r=0\\ r \neq j}}^{n-1} \left( z - z_r \right)
\end{equation*}
mit
\begin{equation}
    \label{def:pi_j}
    \Pi_j \defeq \prod_{\substack{r=0\\ r \neq j}}^{n-1} \left( z_j - z_r \right)^{-1}.
\end{equation}

\noindent Nutzen wir Lemma
\ref{lemma:elementary_symmetric_polynomials_const_multiplication}, so erhalten
wir
\begin{lemma}
    Für $j, r = 0, \dots, n-1$ gilt
    \begin{equation}
        \label{eq:explicit_inverse_vandermonde}
        \begin{split}
            u_{jr}
            &= \Pi_j \cdot \sigma_{n-1-r}^{j}(-z_0, \dots, -z_{n-1})\\
            &\overset{(\ref{eq:elementary_symmetric_polynomials_const_multiplication})}{=}
                (-1)^{n-1-r} \cdot \Pi_j \cdot \sigma_{n-1-r}(z_0, \dots, z_{j-1}, z_{j+1}, \dots, z_{n-1}),
        \end{split}
    \end{equation}
    mit $\Pi_j \defeq \prod_{\substack{r=0\\ r \neq j}}^{n-1} \left( z_j - z_r \right)^{-1}$.
\end{lemma}

\section{Eine Abschätzung der Zeilensummennorm inverser Vandermonde-Matrizen}

Die Grundlage für eine obere Schranke von $\norm{V^{-1}}_\infty$
haben wir bereits im Abschnitt über die elementarsymmetrischen Polynome geschaffen.
Mit Hilfe der dort erbrachten Abschätzung der Betragssumme
elementarsymmetrischer Polynome können wir den folgenden Satz beweisen.
Die Aussage und die Idee des Beweises entstammen dabei aus
\cite[S. 196-197]{gautschi1}.
\begin{theorem}
  \label{thm:inverse_vandermonde_upper_bound}
  Seien $z_0, \dots, z_{n-1} \in \C$ paarweise verschieden.
  Mit $V \defeq \Vand{z}$ gilt
  \begin{equation}
    \label{eq:inverse_vandermonde_upper_bound}
    \norm{V^{-1}}_{\infty}
    \leq \max_{j = 0, \dots, n-1} \left( \prod_{\substack{k = 0\\ k \neq j}}^{n-1} \frac{1 + \abs{z_k}}{\abs{z_j - z_k}} \right).
  \end{equation}
  Gleichheit gilt, wenn $z_k = r_k \cdot e^{i \varphi}$
  für ein festes $\varphi \in \R$ und $r_k \in \R_{+}$ mit $k = 0, \dots, n-1$ gilt.
\end{theorem}
\begin{proof}
    Die explizite Darstellung von $V^{-1}$ in Gleichung
    (\ref{eq:explicit_inverse_vandermonde}) und die Ungleichung
    über elementarsymmetrische Polynome
    (\ref{eq:elementary_symmetric_polynomials_inequality})
    aus Lemma \ref{lemma:elementary_symmetric_polynomials_inequality}
    liefern die Behauptung:
    \begin{equation*}
        \begin{split}
            \norm{V^{-1}}_{\infty}
            &= \max_{j=0, \dots, n-1} \sum_{r=0}^{n-1} \abs{u_{jr}}\\
            &\overset{(\ref{eq:explicit_inverse_vandermonde})}{=}
              \max_{j=0, \dots, n-1} \sum_{r=0}^{n-1} \abs{(-1)^{n-1-r} \Pi_j \sigma_{n-1-r}^{j}}
            = \max_{j=0, \dots, n-1} \abs{\Pi_j} \sum_{r=0}^{n-1} \abs{\sigma_{n-1-r}^{j}}\\
            &\overset{(\ref{eq:elementary_symmetric_polynomials_inequality})}{\leq}
              \max_{j=0, \dots, n-1} \abs{\Pi_j} \prod_{\substack{k=0\\ k \neq j}}^{n-1} \left( 1 + \abs{z_k} \right)
            \overset{(\ref{def:pi_j})}{=}
              \max_{j=0, \dots, n-1} \prod_{\substack{k=0\\ k \neq j}}^{n-1} \frac{1 + \abs{z_k}}{\abs{z_j - z_k}}.
        \end{split}
    \end{equation*}
\end{proof}

\noindent Für die Herleitung einer unteren Schranke stützen wir uns auf \cite{gautschi2}.
Anders als dort formuliert, kann die Aussage jedoch nur für
Vandermonde-Matrizen mit Stützstellen $z_k \neq 0$ für alle $k = 0, \dots, n-1$
bewiesen werden.
Diese Einschränkung folgt aus Jensens Formel, die in \cite{gautschi2} zum
Beweis verwendet wird.
Jensens Formel liefert eine Aussage über analytische Funktionen $f$, welche unter
anderem $f(0) \neq 0$ erfüllen müssen.
Diese Voraussetzung ist verletzt, wenn $0$ als Stützstelle der
Vandermonde-Matrix gewählt wird.

Zunächst sei ohne Beweis an Jensens Formel in Integralform erinnert.
\begin{theorem}[\cite{lang}]
    \label{thm:jensens_formula}
    Sei $\rho \in \R_+$ und $F: \C \mapsto \C$ eine auf $\abs{x} \leq \rho$
    analytische Funktion mit $F(0) \neq 0$.
    Bezeichne mit $\zeta_1, \dots, \zeta_n \in \C$ die Nullstellen von $F$, die
    $\abs{\zeta_j} \leq \rho$ erfüllen.
    Dabei kommen die Nullstellen entsprechend ihrer Vielfachheit eventuell mehrfach
    vor.
    Jensens Formel in Integralform lautet dann
    \begin{equation}
        \frac{1}{2\pi} \int_{0}^{2 \pi} \log \abs{F\left( \rho e^{i\phi} \right)} d\phi
        = \log \abs{F(0)} + \sum_{j=1}^{n} \log \frac{\rho}{\abs{\zeta_j}}.
    \end{equation}
\end{theorem}

% TODO: a_0 /= 0 aus Jensen wird nicht beachtet.
\noindent Unter Verwendung von Satz \ref{thm:jensens_formula} beweisen
wir einen Zusammenhang zwischen der Koeffizientensumme eines Polynoms vom Grad
$n$ und dessen Nullstellen.
Dieser Zusammenhang liefert uns im nächsten Satz schließlich die Abschätzung
der Zeilensummennorm inverser Vandermonde-Matrizen.
\begin{lemma}[\cite{gautschi2}]
    \label{lemma:polynom_coefficient_sum_inequality}
    Sei $p(z) = \sum_{j = 0}^{n} a_j z^j$ ein Polynom $n$-ten Grades mit
    $a_j \in \C$ für $j = 0, \dots, n$, $a_n \neq 0$ und Nullstellen
    $\zeta_j \in \C\setminus\{0\}$ mit $j = 1, \dots, n$.
    Dann gilt
    \begin{equation}
        \label{eq:polynom_coefficient_sum_inequality}
        \sum_{j=0}^{n} \abs{a_j} \geq \abs{a_n} \prod_{j=1}^{n} \max \left(1, \abs{\zeta_j} \right).
    \end{equation}
\end{lemma}

% TODO: Beautify this proof.
\begin{proof}
    Ohne Einschränkung können wir annehmen, dass die Nullstellen sortiert
    vorliegen, so dass
    \[
        \abs{\zeta_1} \leq \dots \leq \abs{\zeta_r} \leq 1 < \abs{\zeta_{r+1}} \leq \dots \leq \abs{\zeta_n}
    \]
    gilt.
    Wir identifizieren $p$ mit $F$ und wählen $\rho = 1$ in Satz \ref{thm:jensens_formula}.
    Mit $\zeta_j \neq 0$ für $j = 1, \dots, n$ aus der Voraussetzung folgt sofort $p(0) \neq 0$.
    Weiterhin ist $p$ als Polynom analytisch auf ganz $\C$ und somit insbesondere auch für $\abs{z} \leq 1$.
    Tatsächlich kann also Satz \ref{thm:jensens_formula} angewendet werden.
    Zusammen mit $\log \frac{1}{z} = -\log z$ liefert dieser
    \[
        \log \abs{a_0}
        = \log \abs{p(0)}
        = \sum_{k=1}^{r} \log \abs{\zeta_k} + \frac{1}{2\pi} \int_{0}^{2\pi} \log \abs{p\left( e^{i\varphi} \right)} d\varphi
    \]
    oder äquivalent dazu
    \begin{equation}
        \label{eq:jensen_polynom}
        \abs{a_0} \prod_{k=1}^{r} \abs{\zeta_k}^{-1}
        = \exp\left( \frac{1}{2\pi} \int_{0}^{2\pi} \log \abs{p\left( e^{i\varphi} \right)} d\varphi \right).
    \end{equation}

    \noindent Mit Hilfe der Nullstellen $\zeta_j$ können wir $p$ als Produkt
    seiner Linearfaktoren darstellen:
    \[
        p(z) = \sum_{j = 0}^{n} a_j z^j = a_n \prod_{k=1}^n (z-\zeta_k).
    \]

    \noindent Die Auswertung an $z=0$ liefert dann
    \[
        a_0 = p(0) = a_n (-1)^n \prod_{k=1}^n \zeta_k,
    \]
    so dass wir (\ref{eq:jensen_polynom}) vereinfacht darstellen können:
    \[
        \abs{a_n} \prod_{k=r+1}^{n} \abs{\zeta_k}
        = \exp\left( \frac{1}{2\pi} \int_{0}^{2\pi} \log \abs{p\left( e^{i\varphi} \right)} d\varphi \right).
    \]

    \noindent Mit
    \begin{equation}
        \label{eq:jensen_integral_estimation}
        \exp\left( \frac{1}{2\pi} \int_{0}^{2\pi} \log \abs{p\left( e^{i\varphi} \right)} d\varphi \right)
        \leq \max_{0 \leq \varphi \leq 2\pi} \abs{p\left( e^{i\varphi} \right) }
        = \max_{0 \leq \varphi \leq 2\pi} \abs{\sum_{j=0}^{n} a_j e^{i\varphi}}
        \leq \sum_{j=0}^{n} \abs{a_j}
    \end{equation}
    folgt nun wie behauptet
    \[
        \sum_{j=0}^n \abs{a_j} \geq \abs{a_n} \prod_{k=1}^n \max(1, \zeta_k).
    \]
\end{proof}

\noindent Nun können wir eine Abschätzung der Zeilensummennorm der inversen
Vandermonde-Matrix angeben und beweisen:
\begin{theorem}
  \label{thm:inverse_vandermonde_lower_bound}
  Seien $z_0, \dots, z_{n-1} \in \C\setminus\{0\}$ paarweise verschieden.
  Mit $V \defeq \Vand{z}$ gilt
  \begin{equation}
    \label{eq:inverse_vandermonde_lower_bound}
    \norm{V^{-1}}_{\infty}
    \geq \max_{j = 0, \dots, n-1} \left( \prod_{\substack{k = 0\\ k \neq j}}^{n-1} \frac{\max(1, \abs{z_k})}{\abs{z_j - z_k}} \right).
  \end{equation}
\end{theorem}
\begin{proof}
    \noindent Wir wählen ein festes
    $j \in \{0, \dots, n-1\}$ und betrachten das Polynom
    \[
        p(z) = \sum_{k=0}^{n-1} a_k z^k
        \defeq \Pi_j \cdot \prod_{\substack{k=0\\ k \neq j}}^{n-1} \left( z - z_k \right)
        \overset{(\ref{eq:explicit_inverse_vandermonde})}{=} \sum_{k=0}^{n-1} (-1)^{n-1-k} \cdot \Pi_j \cdot \sigma_{n-1-k}^j(z_0, \dots, z_{n-1}) \cdot z^k,
    \]
    d.h. die Koeffizienten von $p$ ergeben sich für $k = 0, \dots, n-1$ durch
    \[
        a_k = (-1)^{n-1-k} \cdot \Pi_j \cdot \sigma_{n-1-k}^j(z_0, \dots, z_{n-1}).
    \]
    Offensichtlich hat $p$ die $n-1$ Nullstellen
    $\zeta_r \defeq z_r$ für $r \in \{0,\dots,n-1\} \setminus \{j\}$.
    Mit Ungleichung (\ref{eq:polynom_coefficient_sum_inequality}) aus Lemma
    \ref{lemma:polynom_coefficient_sum_inequality} folgt
    \[
        \begin{split}
            \sum_{r=0}^{n-1} \abs{u_{jr}}
            &\overset{(\ref{eq:explicit_inverse_vandermonde})}{=}
            \sum_{r=0}^{n-1} \abs{(-1)^{n-1-r} \Pi_j \sigma_{n-1-r}^j}\\
            &= \sum_{k=0}^{n-1} \abs{a_k}
            \overset{(\ref{eq:polynom_coefficient_sum_inequality})}{\geq}
                \abs{a_{n}} \prod_{\substack{r=0\\ r \neq j}}^{n-1} \max \left(1, \abs{\zeta_r} \right)\\
            &= \abs{\Pi_j} \underbrace{\abs{\sigma_0^j}}_{=1} \prod_{\substack{r=0\\ r \neq j}}^{n-1} \max \left(1, \abs{z_r} \right)
            = \prod_{\substack{r=0\\ r \neq j}}^{n-1} \frac{\max \left(1, \abs{z_r} \right)}{\abs{z_j - z_r}}.
        \end{split}
    \]

    \noindent Da diese Ungleichung für alle $j \in \{0, \dots, n-1\}$ erfüllt
    ist, folgt die Behauptung:
    \[
        \max_{j = 0, \dots, n-1} \sum_{r=0}^{n-1} \abs{u_{jr}}
        \geq \max_{j = 0, \dots, n-1} \prod_{\substack{r=0\\ r \neq j}}^{n-1} \frac{\max \left(1, \abs{z_r} \right)}{\abs{z_j - z_r}}.
    \]
\end{proof}
