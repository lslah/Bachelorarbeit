\chapter{Stützstellen auf dem Einheitskreis}
% FIXME: Bad section title.
\section{Numerische Vorbetrachungen}

Im Folgenden zeigen wir, dass die Kondition einer Vandermonde-Matrix bezüglich
gewöhnlicher Normen\footnote{}%TODO
invariant unter Rotation und Spiegelung der Knoten ist.

%TODO: Was bringt das?

\begin{lemma}
    Die Kondition der Vandermonde-Matrix bezüglich der Zeilensummennorm ist
    invariant unter Multiplikation der Knoten mit einer komplexen Zahl
    $\alpha = e^{i\varphi} \in \C$, $\varphi \in \R$.
\end{lemma}

\begin{proof}
    Sei $z = (z_0, \dots, z_{n-1}) \in \Cn$.
    Wegen $\abs{e^{i\varphi}} = 1$ für alle $\varphi \in \R$ gilt
    \[
        \begin{split}
            \norm{\Vand{\alpha z}}_\infty
            &= \max_{k=0, \dots, n-1} \sum_{j=0}^{n-1} \abs{\left(e^{i\varphi} z_j\right)^k}
            = \max_{k=0, \dots, n-1} \sum_{j=0}^{n-1} \abs{e^{ik\varphi}} \abs{z_j^k}\\
            &= \max_{k=0, \dots, n-1} \sum_{j=0}^{n-1} \abs{z_j^k}
            = \norm{\Vand{z}}_\infty.
        \end{split}
    \]

    \noindent Für die Zeilensummennorm der inversen Vandermonde-Matrix erinnern wir uns an Gleichung
    (\ref{eq:inverse_vandermonde_const_multiplication})
    aus Lemma \ref{lemma:inverse_vandermonde_const_multiplication}:
    $\tilde{u}_{jr} = u_{jr} \alpha^{-n+r+1}$.
    Damit ist sofort ersichtlich
    \[
        \begin{split}
            \norm{\Vand{\alpha z}^{-1}}_\infty
            &= \max_{j=0, \dots, n-1} \sum_{r=0}^{n-1} \abs{\tilde{u}_{jr}}
            = \max_{j=0, \dots, n-1} \sum_{r=0}^{n-1} \abs{u_{jr}} \abs{\alpha^{-n+r+1}}\\
            &= \max_{j=0, \dots, n-1} \sum_{r=0}^{n-1} \abs{u_{jr}}
            = \norm{\Vand{\alpha}^{-1}}_\infty.
        \end{split}
    \]

    \noindent Wie behauptet, folgt insgesamt
    \[
        \begin{split}
            \cond{\Vand{\alpha z}}_\infty
            &= \norm{\Vand{\alpha z}}_\infty \norm{\Vand{\alpha z}^{-1}}_\infty\\
            &= \norm{\Vand{z}}_\infty \norm{\Vand{z}^{-1}}_\infty
            = \cond{\Vand{z}}_\infty.
        \end{split}
    \]

\end{proof}

\begin{lemma}
    Sei $\emptynorm$ eine Matrix-Norm die invariant unter Multiplikation mit unitären
    Matrizen ist, d.h. für jede Matrix $A \in \C^{n\times n}$ und jede
    unitäre Matrix $U \in \C^{n\times n}$, gilt $\norm{UA} = \norm{AU} = \norm{A}$.
    Weiter seien ein Vektor $z = (z_0, \dots, z_{n-1}) \in \Cn$ und eine
    komplexe Zahl $\alpha \in \C$ mit Betrag $\abs{\alpha} = 1$ gegeben.
    Dann gilt für die Kondition bezüglich der Norm $\emptynorm$:
    \[
        \cond{\Vand{\alpha z}} = \cond{\Vand{z}},
    \]
    d.h. die Kondition der Vandermonde-Matrix ist in diesem Fall invariant
    unter Multiplikation der Knoten mit einer komplexen Zahl $\alpha$ mit
    Betrag $\abs{\alpha} = 1$.
\end{lemma}

\begin{proof}
    Wir setzen
    \[
        V = (v_{kj})_{k,j=0}^{n-1} \defeq \Vand{z}, \;
        \tilde{V} = (\tilde{v}_{kj})_{k,j=0}^{n-1} \defeq \Vand{\alpha z}
    \]
    und zeigen $\norm{\tilde{V}} = \norm{V}$ und
    $\norm{\tilde{V}^{-1}} = \norm{V^{-1}}$.
    Es gilt für $j,k = 0,\dots,n-1$:
    \[
        \tilde{v}_{kj} = (\alpha z_j)^k = \alpha^k z_j^k = \alpha^k v_{kj},
    \]
    d.h. wir können
    \[
        \tilde{V} = \diag{\alpha^0, \dots, \alpha^{n-1}} \cdot V
    \]
    schreiben.
    % TODO: End this proof.

    \noindent Für die inverse Vandermonde-Matrix hatten wir bereits in
    Lemma (\ref{lemma:inverse_vandermonde_const_multiplication})
    gesehen, dass $ \tilde{V}^{-1} = V^{-1} \cdot \diag{\alpha^{0}, \dots, \alpha^{n-1}} $ gilt.

    \noindent Weiter können wir uns leicht davon überzeugen, dass
    $\diag{\alpha^0, \dots, \alpha^{n-1}}$
    für $\alpha = e^{i\varphi}$, $\varphi \in \R$, d.h. für $\abs{\alpha} = 1$
    eine unitäre Matrix ist, denn es gilt:
    \[
        \begin{split}
               \diag{\alpha^0, \dots, \alpha^{n-1}} \cdot \diag{\alpha^0, \dots, \alpha^{n-1}}^{H}
            &= \diag{\alpha^0, \dots, \alpha^{n-1}} \cdot \overline{\diag{\alpha^0, \dots, \alpha^{n-1}}}\\
            &= \diag{1, e^{i\varphi} \cdot e^{-i\varphi}, \dots, e^{i\varphi (n-1)} \cdot e^{-i \varphi (-n+1)}}\\
            &= \diag{1, \dots, 1}.
        \end{split}
    \]
    Damit erhalten wir
    \[
        \begin{split}
            \cond{\Vand{\alpha z}}
            = \cond{\tilde{V}}
            &= \norm{\tilde{V}} \norm{\tilde{V}^{-1}}\\
            &= \norm{\diag{\alpha^0, \dots, \alpha^{n-1}} \cdot V} \norm{V^{-1} \cdot \diag{\alpha^0, \dots, \alpha^{n-1}}}\\
            &= \norm{V} \norm{V^{-1}}
            = \cond{V} = \cond{\Vand{z}}.
        \end{split}
    \]
\end{proof}

\section{Freie Wahl der Stützstellen}
\section{Stützstellen auf dem Gitter der $N$-ten Einheitswurzeln}
Seien $M, N \in \N$ mit $M < N$.
In diesem Abschnitt untersuchen wir Anordnungen von $M$ Knoten auf dem Gitter
der $N$-ten Einheitswurzeln und die Kondition der daraus resultierenden
Vandermonde-Matrizen.
Da in unserem Fall
\footnote{Wir betrachten nur die Zeilensummennorm und unitär invariante Normen bzw. die Kondition bzgl. dieser Normen.}
die Kondition einer Vandermonde-Matrix invariant unter
Permutation der Spalten ist, kommt es nicht auf die Reihenfolge der Knoten an.
Wir betrachten also genau die $M$-elementigen Teilmengen von
$\{ \exp( \frac{2 \pi i k}{N} ) \; | \; k \leq N \}$.
Diese können wir bekanntlich auf eindeutige Weise mit $M$-elementigen
Teilmengen von $ \Z / N\Z$ identifizieren, indem wir die Abbildung
\[
    \left\{ \exp( \frac{2 \pi i k_j}{N} ) \; | \; j = 1,\dots, M \right\} \mapsto \left\{ k_j \in \Z/N\Z \; | \; j = 1, \dots, M \right\}
\]
verwenden.
Diese motiviert die folgende Definition:

\begin{mydef}[Knotenset]
    Wir bezeichnen eine $M$-elementige Teilmenge von $\Z/N\Z$
    als \emph{Knotenset der Ordnung $(M,N)$}.\\
    Wir identifizieren ein Knotenset
    $S = \{k_1, \dots, k_m\} \subset \Z/N\Z$
    mit einer Menge von $M$ verschiedenen $N$-ten Einheitswurzeln durch
    \[
        S \mapsto e(S, N) \defeq \left\{ e^{\frac{2 \pi i k_1}{N}}, \dots, e^{\frac{2 \pi i k_n}{N}} \right\}.
    \]
    Da die Kondition einer Vandermonde-Matrix invariant unter Permutationen der
    Knoten ist, definieren wir weiter die Kondition eines Knotensets $S$ durch
    \[
        \cond{S} \defeq \cond{\Vand{ e^{\frac{2 \pi i k_1}{N}}, \dots, e^{\frac{2 \pi i k_n}{N}} }}.
    \]
\end{mydef}
