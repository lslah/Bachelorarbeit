\chapter{Vandermonde-Matrizen mit Stützstellen auf dem Einheitskreis}

Wie das vorherige Kapitel zeigt, sind Vandermonde-Matrizen mit reellen
Stützstellen schlecht konditioniert und werden daher hauptsächlich für
theoretische Betrachtungen verwendet.

Anders verhält es sich, wenn man Knoten in der komplexen Ebene zulässt.
Wählt man als Stützstellen die $n$-ten Einheitswurzeln, so erreicht die
Vandermonde-Matrix sogar die perfekte Kondition $1$ bezüglich der Spektralnorm.
Auch bezüglich der Frobenius- und der Zeilensummennorm wächst die Kondition
dieser Vandermonde-Matrizen nur linear in $n$.

In diesem Abschnitt untersuchen wir Vandermonde-Matrizen mit Knoten auf dem
komplexen Einheitskreis, die jedoch von dem perfekten Fall der $n$-ten
Einheitswurzeln abweichen.
Für diese speziellen Konfigurationen leiten wir explizite Formeln zur
Berechnung der Kondition bezüglich der Zeilensummennorm und der Frobeniusnorm
her.

\section{Vandermonde-Matrizen zu den \boldmath $n$-ten Einheitswurzeln}

% TODO: Es ist \norm{I}_F = \sqrt{n}, aber \cond[F]{I} = n. Kann man zeigen, dass \cond[F]{A} \geq n für alle A \in \C^{n \times n} gilt?

Wir beweisen zunächst die Aussagen über die Frobenius- und die Zeilensummennorm
von Vandermonde-Matrizen zu den Knoten der $n$-ten Einheitswurzeln.
\begin{lemma}
    \label{lemma:frobenius_norm_vandermonde_unit_circle}
    Für Knoten auf dem Einheitskreis
    $z = \left(e^{i \varphi_0}, \dots, e^{i \varphi_{n-1}} \right) \in \C$
    mit $\varphi_j \in \R$ für $j = 0, \dots, n-1$ gilt
    \begin{equation}
        \norm{\Vand{z}}_F = n.
    \end{equation}
\end{lemma}
\begin{proof}
    Es gilt
    \[
        \norm{\Vand{z}}_F^2
        = \sum_{k=0}^{n-1} \sum_{j=0}^{n-1} \abs{ \left(e^{i \varphi_j}\right)^k }^2
        = \sum_{k=0}^{n-1} \sum_{j=0}^{n-1} 1
        = n^2.
    \]
\end{proof}

\begin{lemma}
    \label{lemma:frobenius_condition_unit_roots}
    Sei $n \in \N$ und $z = (\wn^0, \wn^1, \dots, \wn^{n-1})  \in \Cn$ der Vektor der
    $n$-ten Einheitswurzeln, d.h. $\wn \defeq e^{\frac{2 \pi i}{n}}$.
    Dann gilt für die Kondition bzgl. der Frobeniusnorm
    \[
        \cond[F]{\Vand{z}} = n.
    \]
\end{lemma}
\begin{proof}
    Wir setzen $W = (w_{jr})_{j,r=0}^{n-1} \in \C^{n\times n}$ mit
    $w_{jr} \defeq \frac{1}{n} \wn^{-jr}$
    und zeigen $V^{-1} = W$.
    Seien dazu $a_{kr} \in \C$ die Elemente des Matrixproduktes
    $A = V W \in \C^{n\times n}$.
    Dann gilt für $k, r = 0, \dots, n-1$
    \[
        \begin{split}
            a_{kr}
            &= \sum_{j=0}^{n-1} v_{kj} w_{jr}
            = \sum_{j=0}^{n-1} \frac{1}{n} \wn^{kj} \wn^{-jr} \\
            &= \frac{1}{n} \sum_{j=0}^{n-1} \wn^{j (k-r)}
            = \frac{1}{n} \sum_{j=0}^{n-1} \left( \wn^{k-r} \right)^j\\
            &= \frac{1}{n} \cdot \left\{
                \begin{array}{ll}
                    n, \text{ falls } k-r \equiv 0 \mod n\\
                    0, \text{ falls } k-r \not\equiv 0 \mod n
                \end{array}
              \right.\\
            &= \left\{
                \begin{array}{ll}
                    1, \text{ falls } k = r \\
                    0, \text{ falls } k \neq r .
                \end{array}
              \right.
        \end{split}
    \]
    Damit ist $V^{-1} = W$ gezeigt.
    Nach Lemma \ref{lemma:frobenius_norm_vandermonde_unit_circle} gilt $\norm{V}_F = n$.
    Komplexe Konjugation ändert nichts an der Frobeniusnorm einer Matrix, da
    diese nur von den Beträgen der Matrixelemente abhängt.
    Wir erhalten damit $\norm{V^{-1}}_F = \norm{W}_F = \frac{1}{n}
    \norm{\overline V}_F = 1$, was die Behauptung zeigt.
\end{proof}

\begin{lemma}
    Sei erneut $z = (\wn^0, \wn^1, \dots, \wn^{n-1}) \in \Cn$ mit
    $\wn = e^{\frac{2 \pi i}{n}}$.
    Dann gilt für die Kondition bzgl. der Zeilensummennorm
    \[
        \cond[\infty]{\Vand{z}} = n.
    \]
\end{lemma}
\begin{proof}
    Wie im Beweis von Lemma \ref{lemma:frobenius_condition_unit_roots} gezeigt, gilt
    $ \Vand{z}^{-1} = \frac{1}{n} \overline{\Vand{z}} $.
    Genau wie bei der Frobeniusnorm, ändert komplexe Konjugation
    nichts an der Zeilensummennorm einer Matrix, da diese nur von den Beträgen
    der Matrixelemente abhängt. Insbesondere ergibt
    sich $\norm{\overline{\Vand{z}}}_\infty = \norm{\Vand{z}}_\infty$.
    Mit Lemma \ref{lemma:infty_norm_vandermonde_unit_roots} folgt dann
    \[
        \cond[\infty]{\Vand{z}}
        = \norm{\Vand{z}}_\infty \cdot \frac{1}{n} \cdot \norm{\overline{\Vand{z}}}_\infty
        \overset{(\ref{eq:infty_norm_vandermonde_unit_roots})}= n.
    \]
\end{proof}


\section{Invarianz der Kondition unter Rotation der Knoten}

Im Folgenden zeigen wir, dass sich die Kondition einer Vandermonde-Matrix
bezüglich der Zeilensummennorm und aller unitär invarianten Normen weder unter
Rotation noch unter Spiegelung der Knoten verändert.

\begin{lemma}
    \label{lemma:vandermonde_rotation_invariance}
    Die Kondition der Vandermonde-Matrix bezüglich der Zeilensummennorm ist
    invariant unter Multiplikation der Knoten mit einer komplexen Zahl
    $\alpha = e^{i\varphi} \in \C$ mit $\varphi \in \R$.
\end{lemma}
\begin{proof}
    Sei $z = (z_0, \dots, z_{n-1}) \in \Cn$.
    Wegen $\abs{e^{i\varphi}} = 1$ für alle $\varphi \in \R$ gilt
    \[
        \begin{split}
            \norm{\Vand{\alpha z}}_\infty
            &= \max_{k=0, \dots, n-1} \sum_{j=0}^{n-1} \abs{\left(e^{i\varphi} z_j\right)^k}
            = \max_{k=0, \dots, n-1} \sum_{j=0}^{n-1} \abs{e^{ik\varphi}} \abs{z_j^k}\\
            &= \max_{k=0, \dots, n-1} \sum_{j=0}^{n-1} \abs{z_j^k}
            = \norm{\Vand{z}}_\infty.
        \end{split}
    \]

    \noindent Für die Zeilensummennorm der inversen Vandermonde-Matrix erinnern
    wir uns an Gleichung
    (\ref{eq:inverse_vandermonde_const_multiplication})
    aus Lemma \ref{lemma:vandermonde_const_multiplication}:
        $\Vand{\alpha z}^{-1}
        = \Vand{z}^{-1} \cdot \diag{\alpha^0, \alpha^{-1}, \dots, \alpha^{-n+1}}.$
    Damit ist sofort ersichtlich, dass
    \[
        \begin{split}
            \norm{\Vand{\alpha z}^{-1}}_\infty
            &= \max_{j=0, \dots, n-1} \sum_{r=0}^{n-1} \abs{u_{jr}} \abs{\alpha^{-r}}\\
            &= \max_{j=0, \dots, n-1} \sum_{r=0}^{n-1} \abs{u_{jr}}
            = \norm{\Vand{\alpha}^{-1}}_\infty.
        \end{split}
    \]

    \noindent Wie behauptet folgt insgesamt
    \[
        \begin{split}
            \cond[\infty]{\Vand{\alpha z}}
            &= \norm{\Vand{\alpha z}}_\infty \norm{\Vand{\alpha z}^{-1}}_\infty\\
            &= \norm{\Vand{z}}_\infty \norm{\Vand{z}^{-1}}_\infty
            = \cond[\infty]{\Vand{z}}.
        \end{split}
    \]
\end{proof}

\begin{lemma}
    Sei $\emptynorm$ eine Matrix-Norm, die invariant unter Multiplikation mit unitären
    Matrizen ist, d.h. für jede Matrix $A \in \C^{n\times n}$ und jede
    unitäre Matrix $U \in \C^{n\times n}$ gilt $\norm{UA} = \norm{AU} = \norm{A}$.
    Weiter seien ein Vektor $z = (z_0, \dots, z_{n-1}) \in \Cn$ und eine
    komplexe Zahl $\alpha \in \C$ mit Betrag $\abs{\alpha} = 1$ gegeben.
    Dann gilt für die Kondition bezüglich der Norm $\emptynorm$
    \[
        \cond{\Vand{\alpha z}} = \cond{\Vand{z}},
    \]
    d.h. die Kondition der Vandermonde-Matrix ist in diesem Fall invariant
    unter Multiplikation der Knoten mit einer komplexen Zahl $\alpha$ vom
    Betrag $\abs{\alpha} = 1$.
\end{lemma}
\begin{proof}
    Nach Lemma \ref{lemma:vandermonde_const_multiplication} gelten
    \[
        \norm{ \Vand{\alpha z} }
        \overset{(\ref{eq:vandermonde_const_multiplication})}{=}
            \norm{\diag{\alpha^0, \dots, \alpha^{n-1}} \cdot \Vand{z}}
    \]
    und
    \[
        \norm{ \Vand{\alpha z}^{-1} }
        \overset{(\ref{eq:inverse_vandermonde_const_multiplication})}{=}
            \norm{\Vand{z}^{-1} \cdot \diag{\alpha^0, \alpha^{-1}, \dots, \alpha^{-n+1}}}.
    \]
    Mit $e^{i \varphi} \defeq \alpha$ für ein $\varphi \in \R$ stellen wir
    leicht fest, dass
    $\diag{\alpha^0, \alpha^1, \dots, \alpha^{n-1}}$ eine unitäre Matrix ist:
    \[
        \begin{split}
            \diag{\alpha^0, \alpha^1, \dots, \alpha^{n-1}}^H
            &= \diag{1, e^{i \varphi}, \dots, e^{(n-1) i \varphi}}^H\\
            &= \diag{1, e^{-i \varphi}, \dots, e^{-(n-1) i \varphi}}\\
            &= \diag{1, e^{i \varphi}, \dots, e^{(n-1) i \varphi}}^{-1}\\
            &= \diag{\alpha^0, \alpha^1, \dots, \alpha^{n-1}}^{-1},
        \end{split}
    \]
    wobei $A^H$ die adjungierte Matrix von $A$ bezeichne.
    \noindent Insgesamt folgt
    \[
        \begin{split}
            \cond{\Vand{\alpha z}}
            &= \norm{\Vand{\alpha z}} \norm{\Vand{\alpha z}^{-1}}\\
            &= \norm{\diag{\alpha^0, \dots, \alpha^{n-1}} \cdot \Vand{z}} \norm{\Vand{z}^{-1} \cdot \diag{\alpha^0, \alpha^{-1}, \dots, \alpha^{-n+1}}}\\
            &= \norm{\Vand{z}} \norm{\Vand{z}^{-1}}
            = \cond{\Vand{z}}.
        \end{split}
    \]
\end{proof}

% SECTION:OUTLIER
\section{Vandermonde-Matrizen aus \boldmath{$(n\!-\!1)$} Einheitswurzeln und einem Ausreißer}

Im Folgenden untersuchen wir die Kondition einer Vandermonde-Matrix zu den
Knoten der $n$-ten Einheitswurzeln, wobei einer der Knoten von seiner
ursprünglichen Position um einen Winkel
$\varphi \in \left(-\frac{2\pi}{n},\frac{2 \pi}{n}\right)$
auf dem Einheitskreis ausgelenkt wird.
Wie bereits in Lemma \ref{lemma:vandermonde_rotation_invariance} gezeigt,
ändern weder Drehungen noch Spiegelungen der Knoten die Kondition der
Vandermonde-Matrix.
Daher können wir, ohne das Problem zu beschränken, stets den Knoten auslenken,
welcher der $n$-ten Einheitswurzel $1$ zugeordnet ist.

Wir betrachten also für $\delta\!\in\!(-1, 1)$ den Knotenvektor
$z(\delta) = (z_0(\delta), z_1, \dots, z_{n-1}) \in \C^n$
mit
$z_0(\delta) = e^{2 \pi i \delta / n}$
und
$z_j = e^{2 \pi i j / n}$ für $j = 1, \dots, n-1$.
Die dazu gehörende Vandermonde-Matrix bezeichnen wir in diesem Abschnitt mit
$\Vand{\delta} \defeq \Vand{z(\delta)}$.
Häufig werden wir auf die explizite Angabe der Abhängigkeit von $\delta$
verzichten, und beispielsweise nur kurz $z_0$ anstelle von $z_0(\delta)$
schreiben.

\subsection{Kondition bezüglich der Zeilensummennorm}

Vandermonde-Matrizen mit Stützstellen innerhalb des komplexen Einheitskreises
haben die Zeilensummennorm $n$, wie bereits in
\lemmaref{infty_norm_vandermonde_unit_roots} gezeigt wurde.
Dies trifft offensichtlich auch in unserem Fall zu, so dass wir für die
Berechnung der Kondition nur noch die Zeilensummennorm der inversen
Vandermonde-Matrix $\Vand{\delta}^{-1}$ ermitteln müssen.

\begin{lemma}
    \label{lemma:inverse_outlier_vandermonde_first_row_abs_sum}
    Sei $z(\delta) = (z_0(\delta), \dots, z_{n-1}) \in \C^n$ mit
    $z_0(\delta) = e^{2 \pi i \delta / n}$
    und
    $z_j = e^{2 \pi i j / n}$ für $j = 1, \dots, n-1$.
    Weiter seien $u_{jr} \in \C$ für $j,r = 0,\dots,n-1$ die Elemente der
    inversen Vandermonde-Matrix $\Vand{z(\delta)}^{-1}$.
    Dann gilt
    \[
        \sum_{r=0}^{n-1} \abs{ u_{0r} }
        = n \cdot \prod_{k=1}^{n-1} \abs{z_0 - z_k}^{-1}
        = n \cdot \prod_{j=1}^{n-1} \frac{1}{2 \cdot \sin \left(\frac{\pi (j - \delta)}{n} \right)}.
    \]
\end{lemma}

\begin{proof}
    Nach Gleichung (\ref{eq:explicit_inverse_vandermonde}) gilt
    \[
        u_{0r} = (-1)^{n-1-r} \cdot \Pi_0 \cdot \sigma_{n-1-r}^{0}(z_1, \dots, z_{n-1})
    \]
    mit
    \[
        \Pi_0 = \prod_{k=1}^{n-1} \abs{ z_0 - z_k }^{-1}.
    \]

    \noindent Sei $k \in \{1, \dots, n-1\}$ und $\varphi \defeq \frac{\pi (k - \delta)}{n}$.
    Dann gilt
    \[
        \begin{split}
            \abs{z_0 - z_k}^2
            &= (z_0 - z_k) (\conj{z}_0 - \conj{z}_k)
            = \abs{z_0}^2 + \abs{z_k}^2 - z_0\conj{z}_k - z_k\conj{z}_0\\
            &= 2 - (e^{2 \varphi i} + e^{-2 \varphi i})
            = 2 - ((e^{\varphi i})^2 + (e^{-\varphi i})^2)\\
            &= 2 - ( (e^{\varphi i} - e^{-\varphi i})^2 + 2 e^{(\varphi - \varphi) i})
            = (e^{\varphi i} - e^{-\varphi i})^2\\
            &= 4 \cdot \sin^2 (\varphi)
            = 4 \cdot \sin^2 \left( \frac{\pi (k-\delta)}{n} \right),
        \end{split}
    \]
    also
    \begin{equation}
        \label{eq:pi_0}
        \Pi_0
        = \prod_{\substack{k = 1}}^{n-1} \abs{ z_0 - z_k }^{-1}
        = \prod_{\substack{k = 1}}^{n-1} \frac{1}{2 \cdot \sin \left( \frac{\pi (k-\delta)}{n} \right)}.
    \end{equation}

    \noindent Zusammen mit der Aussage des Korollars
    \ref{corollary:outlier_sigma_zero_row_sum} folgt nun die Behauptung:
    \[
        \begin{split}
            \sum_{r=0}^{n-1} \abs{ u_{0r} }
            &= \sum_{r=0}^{n-1} \abs{ (-1)^{n-1-r} \cdot \Pi_0 \cdot \sigma_{n-1-r}(z_1, \dots, z_{n-1}) }\\
            &= \left( \sum_{r=0}^{n-1} \abs{ \sigma_{n-1-r}(z_1, \dots, z_{n-1}) } \right) \cdot \abs{\Pi_0}\\
            &\overset{(\ref{eq:outlier_sigma_zero_row_sum})}{=}
                n \cdot \prod_{j=1}^{n-1} \frac{1}{2 \cdot \sin \left(\frac{\pi (j - \delta)}{n} \right)}.
        \end{split}
    \]
\end{proof}

Der Beweis der folgenden Vermutung kann im Rahmen dieser Arbeit nicht
erbracht werden. Numerische Untersuchungen konnten die Aussage jedoch nicht
widerlegen.

\begin{assumption}
    \label{assumption:infty_outlier_condition}
    Es gilt
    \begin{equation}
        \label{eq:infty_condition_assumption}
        \cond[\infty]{ \Vand{\delta} }
        = n^2 \prod_{j=1}^{n-1} \abs{ z_0(\delta) - z_j }^{-1}
        = n^2 \prod_{j=1}^{n-1} \frac{1}{2 \cdot \sin \left(\frac{\pi (j - \delta)}{n} \right)}.
    \end{equation}
\end{assumption}

\begin{remark}
    Zum Beweis der Vermutung muss gezeigt werden, dass
    \[
        \norm{\Vand{\delta}^{-1}}_\infty
        = n \cdot \prod_{j=1}^{n-1} \frac{1}{2 \cdot \sin \left(\frac{\pi (j - \delta)}{n} \right)}.
    \]
    gilt.
    Bezeichnen wir erneut mit $u_{jr}$ für $j,r = 0, \dots, n-1$ die Elemente von $\Vand{\delta}^{-1}$,
    so bleibt zu zeigen, dass für $j = 1, \dots, n-1$
    \[
        \sum_{r=0}^{n-1} \abs{u_{jr}} \leq \sum_{r=0}^{n-1} \abs{u_{0r}}
    \]
    gilt.
    Lemma \ref{lemma:inverse_outlier_vandermonde_first_row_abs_sum} liefert dann die Behauptung.

    \noindent Der Ansatz, die beiden Faktoren $\Pi_j$ und
    $\sum_{r=0}^{n-1} \abs{\sigma_r^j}$ einzeln abzuschätzen,
    d.h. die Ungleichungen
    \[
        \Pi_j \leq \Pi_0
    \]
    und
    \[
        \sum_{r=0}^{n-1} \abs{\sigma_r^j} \leq \sum_{r=0}^{n-1} \abs{\sigma_r^0}
    \]
    für alle $ j=1, \dots, n-1$ zu beweisen, muss im Allgemeinen scheitern, wie
    die Figuren \ref{fig:pi_j} und \ref{fig:sigma_row_sum} für den Fall $n=5$ zeigen.
    Die schwarzen Graphen, die den Fall $j=0$ darstellen, nehmen dort niemals
    den Maximalwert unter allen Graphen von $j=0, \dots, n-1$ an.
    Die Figur \ref{fig:row_j} legt jedoch nahe, dass die Behauptung des
    Satzes tatsächlich wahr ist, denn der Graph des Produktes ist stets für
    $j=0$ maximal.

    \begin{figure}[H]
        \centering
        %% Creator: Matplotlib, PGF backend
%%
%% To include the figure in your LaTeX document, write
%%   \input{<filename>.pgf}
%%
%% Make sure the required packages are loaded in your preamble
%%   \usepackage{pgf}
%%
%% Figures using additional raster images can only be included by \input if
%% they are in the same directory as the main LaTeX file. For loading figures
%% from other directories you can use the `import` package
%%   \usepackage{import}
%% and then include the figures with
%%   \import{<path to file>}{<filename>.pgf}
%%
%% Matplotlib used the following preamble
%%   \usepackage[utf8x]{inputenc}
%%   \usepackage[T1]{fontenc}
%%
\begingroup%
\makeatletter%
\begin{pgfpicture}%
\pgfpathrectangle{\pgfpointorigin}{\pgfqpoint{5.208661in}{3.219130in}}%
\pgfusepath{use as bounding box}%
\begin{pgfscope}%
\pgfsetbuttcap%
\pgfsetroundjoin%
\definecolor{currentfill}{rgb}{1.000000,1.000000,1.000000}%
\pgfsetfillcolor{currentfill}%
\pgfsetlinewidth{0.000000pt}%
\definecolor{currentstroke}{rgb}{1.000000,1.000000,1.000000}%
\pgfsetstrokecolor{currentstroke}%
\pgfsetdash{}{0pt}%
\pgfpathmoveto{\pgfqpoint{0.000000in}{0.000000in}}%
\pgfpathlineto{\pgfqpoint{5.208661in}{0.000000in}}%
\pgfpathlineto{\pgfqpoint{5.208661in}{3.219130in}}%
\pgfpathlineto{\pgfqpoint{0.000000in}{3.219130in}}%
\pgfpathclose%
\pgfusepath{fill}%
\end{pgfscope}%
\begin{pgfscope}%
\pgfsetbuttcap%
\pgfsetroundjoin%
\definecolor{currentfill}{rgb}{1.000000,1.000000,1.000000}%
\pgfsetfillcolor{currentfill}%
\pgfsetlinewidth{0.000000pt}%
\definecolor{currentstroke}{rgb}{0.000000,0.000000,0.000000}%
\pgfsetstrokecolor{currentstroke}%
\pgfsetstrokeopacity{0.000000}%
\pgfsetdash{}{0pt}%
\pgfpathmoveto{\pgfqpoint{0.651083in}{0.321913in}}%
\pgfpathlineto{\pgfqpoint{4.687795in}{0.321913in}}%
\pgfpathlineto{\pgfqpoint{4.687795in}{2.897217in}}%
\pgfpathlineto{\pgfqpoint{0.651083in}{2.897217in}}%
\pgfpathclose%
\pgfusepath{fill}%
\end{pgfscope}%
\begin{pgfscope}%
\pgfpathrectangle{\pgfqpoint{0.651083in}{0.321913in}}{\pgfqpoint{4.036712in}{2.575304in}} %
\pgfusepath{clip}%
\pgfsetrectcap%
\pgfsetroundjoin%
\pgfsetlinewidth{1.003750pt}%
\definecolor{currentstroke}{rgb}{0.000000,0.000000,0.000000}%
\pgfsetstrokecolor{currentstroke}%
\pgfsetdash{}{0pt}%
\pgfpathmoveto{\pgfqpoint{0.852918in}{2.554665in}}%
\pgfpathlineto{\pgfqpoint{0.873102in}{2.338557in}}%
\pgfpathlineto{\pgfqpoint{0.893285in}{2.158906in}}%
\pgfpathlineto{\pgfqpoint{0.913469in}{2.007290in}}%
\pgfpathlineto{\pgfqpoint{0.933653in}{1.877693in}}%
\pgfpathlineto{\pgfqpoint{0.953836in}{1.765706in}}%
\pgfpathlineto{\pgfqpoint{0.974020in}{1.668019in}}%
\pgfpathlineto{\pgfqpoint{0.994203in}{1.582103in}}%
\pgfpathlineto{\pgfqpoint{1.014387in}{1.505992in}}%
\pgfpathlineto{\pgfqpoint{1.034570in}{1.438133in}}%
\pgfpathlineto{\pgfqpoint{1.054754in}{1.377283in}}%
\pgfpathlineto{\pgfqpoint{1.074937in}{1.322438in}}%
\pgfpathlineto{\pgfqpoint{1.095121in}{1.272776in}}%
\pgfpathlineto{\pgfqpoint{1.115305in}{1.227617in}}%
\pgfpathlineto{\pgfqpoint{1.135488in}{1.186395in}}%
\pgfpathlineto{\pgfqpoint{1.155672in}{1.148636in}}%
\pgfpathlineto{\pgfqpoint{1.175855in}{1.113938in}}%
\pgfpathlineto{\pgfqpoint{1.196039in}{1.081958in}}%
\pgfpathlineto{\pgfqpoint{1.216222in}{1.052404in}}%
\pgfpathlineto{\pgfqpoint{1.236406in}{1.025022in}}%
\pgfpathlineto{\pgfqpoint{1.276773in}{0.975929in}}%
\pgfpathlineto{\pgfqpoint{1.317140in}{0.933243in}}%
\pgfpathlineto{\pgfqpoint{1.357507in}{0.895857in}}%
\pgfpathlineto{\pgfqpoint{1.397874in}{0.862905in}}%
\pgfpathlineto{\pgfqpoint{1.438242in}{0.833699in}}%
\pgfpathlineto{\pgfqpoint{1.478609in}{0.807686in}}%
\pgfpathlineto{\pgfqpoint{1.518976in}{0.784418in}}%
\pgfpathlineto{\pgfqpoint{1.559343in}{0.763526in}}%
\pgfpathlineto{\pgfqpoint{1.599710in}{0.744707in}}%
\pgfpathlineto{\pgfqpoint{1.640077in}{0.727706in}}%
\pgfpathlineto{\pgfqpoint{1.700628in}{0.705155in}}%
\pgfpathlineto{\pgfqpoint{1.761179in}{0.685634in}}%
\pgfpathlineto{\pgfqpoint{1.821729in}{0.668680in}}%
\pgfpathlineto{\pgfqpoint{1.882280in}{0.653925in}}%
\pgfpathlineto{\pgfqpoint{1.942831in}{0.641071in}}%
\pgfpathlineto{\pgfqpoint{2.023565in}{0.626483in}}%
\pgfpathlineto{\pgfqpoint{2.104299in}{0.614405in}}%
\pgfpathlineto{\pgfqpoint{2.185033in}{0.604491in}}%
\pgfpathlineto{\pgfqpoint{2.265768in}{0.596479in}}%
\pgfpathlineto{\pgfqpoint{2.366685in}{0.588832in}}%
\pgfpathlineto{\pgfqpoint{2.467603in}{0.583557in}}%
\pgfpathlineto{\pgfqpoint{2.568521in}{0.580463in}}%
\pgfpathlineto{\pgfqpoint{2.669439in}{0.579443in}}%
\pgfpathlineto{\pgfqpoint{2.770357in}{0.580463in}}%
\pgfpathlineto{\pgfqpoint{2.871275in}{0.583557in}}%
\pgfpathlineto{\pgfqpoint{2.972192in}{0.588832in}}%
\pgfpathlineto{\pgfqpoint{3.073110in}{0.596479in}}%
\pgfpathlineto{\pgfqpoint{3.153844in}{0.604491in}}%
\pgfpathlineto{\pgfqpoint{3.234579in}{0.614405in}}%
\pgfpathlineto{\pgfqpoint{3.315313in}{0.626483in}}%
\pgfpathlineto{\pgfqpoint{3.396047in}{0.641071in}}%
\pgfpathlineto{\pgfqpoint{3.456598in}{0.653925in}}%
\pgfpathlineto{\pgfqpoint{3.517149in}{0.668680in}}%
\pgfpathlineto{\pgfqpoint{3.577699in}{0.685634in}}%
\pgfpathlineto{\pgfqpoint{3.638250in}{0.705155in}}%
\pgfpathlineto{\pgfqpoint{3.698801in}{0.727706in}}%
\pgfpathlineto{\pgfqpoint{3.739168in}{0.744707in}}%
\pgfpathlineto{\pgfqpoint{3.779535in}{0.763526in}}%
\pgfpathlineto{\pgfqpoint{3.819902in}{0.784418in}}%
\pgfpathlineto{\pgfqpoint{3.860269in}{0.807686in}}%
\pgfpathlineto{\pgfqpoint{3.900636in}{0.833699in}}%
\pgfpathlineto{\pgfqpoint{3.941003in}{0.862905in}}%
\pgfpathlineto{\pgfqpoint{3.981370in}{0.895857in}}%
\pgfpathlineto{\pgfqpoint{4.021738in}{0.933243in}}%
\pgfpathlineto{\pgfqpoint{4.062105in}{0.975929in}}%
\pgfpathlineto{\pgfqpoint{4.082288in}{0.999594in}}%
\pgfpathlineto{\pgfqpoint{4.102472in}{1.025022in}}%
\pgfpathlineto{\pgfqpoint{4.122655in}{1.052404in}}%
\pgfpathlineto{\pgfqpoint{4.142839in}{1.081958in}}%
\pgfpathlineto{\pgfqpoint{4.163023in}{1.113938in}}%
\pgfpathlineto{\pgfqpoint{4.183206in}{1.148636in}}%
\pgfpathlineto{\pgfqpoint{4.203390in}{1.186395in}}%
\pgfpathlineto{\pgfqpoint{4.223573in}{1.227617in}}%
\pgfpathlineto{\pgfqpoint{4.243757in}{1.272776in}}%
\pgfpathlineto{\pgfqpoint{4.263940in}{1.322438in}}%
\pgfpathlineto{\pgfqpoint{4.284124in}{1.377283in}}%
\pgfpathlineto{\pgfqpoint{4.304307in}{1.438133in}}%
\pgfpathlineto{\pgfqpoint{4.324491in}{1.505992in}}%
\pgfpathlineto{\pgfqpoint{4.344675in}{1.582103in}}%
\pgfpathlineto{\pgfqpoint{4.364858in}{1.668019in}}%
\pgfpathlineto{\pgfqpoint{4.385042in}{1.765706in}}%
\pgfpathlineto{\pgfqpoint{4.405225in}{1.877693in}}%
\pgfpathlineto{\pgfqpoint{4.425409in}{2.007290in}}%
\pgfpathlineto{\pgfqpoint{4.445592in}{2.158906in}}%
\pgfpathlineto{\pgfqpoint{4.465776in}{2.338557in}}%
\pgfpathlineto{\pgfqpoint{4.485960in}{2.554665in}}%
\pgfpathlineto{\pgfqpoint{4.485960in}{2.554665in}}%
\pgfusepath{stroke}%
\end{pgfscope}%
\begin{pgfscope}%
\pgfpathrectangle{\pgfqpoint{0.651083in}{0.321913in}}{\pgfqpoint{4.036712in}{2.575304in}} %
\pgfusepath{clip}%
\pgfsetrectcap%
\pgfsetroundjoin%
\pgfsetlinewidth{1.003750pt}%
\definecolor{currentstroke}{rgb}{0.000000,0.000000,1.000000}%
\pgfsetstrokecolor{currentstroke}%
\pgfsetdash{}{0pt}%
\pgfpathmoveto{\pgfqpoint{0.852918in}{0.484718in}}%
\pgfpathlineto{\pgfqpoint{1.175855in}{0.492351in}}%
\pgfpathlineto{\pgfqpoint{1.458425in}{0.501195in}}%
\pgfpathlineto{\pgfqpoint{1.720811in}{0.511652in}}%
\pgfpathlineto{\pgfqpoint{1.942831in}{0.522605in}}%
\pgfpathlineto{\pgfqpoint{2.144666in}{0.534654in}}%
\pgfpathlineto{\pgfqpoint{2.326318in}{0.547611in}}%
\pgfpathlineto{\pgfqpoint{2.487787in}{0.561211in}}%
\pgfpathlineto{\pgfqpoint{2.629072in}{0.575085in}}%
\pgfpathlineto{\pgfqpoint{2.770357in}{0.591219in}}%
\pgfpathlineto{\pgfqpoint{2.891458in}{0.607246in}}%
\pgfpathlineto{\pgfqpoint{3.012559in}{0.625761in}}%
\pgfpathlineto{\pgfqpoint{3.113477in}{0.643501in}}%
\pgfpathlineto{\pgfqpoint{3.214395in}{0.663798in}}%
\pgfpathlineto{\pgfqpoint{3.295129in}{0.682250in}}%
\pgfpathlineto{\pgfqpoint{3.375864in}{0.703062in}}%
\pgfpathlineto{\pgfqpoint{3.456598in}{0.726697in}}%
\pgfpathlineto{\pgfqpoint{3.517149in}{0.746625in}}%
\pgfpathlineto{\pgfqpoint{3.577699in}{0.768785in}}%
\pgfpathlineto{\pgfqpoint{3.638250in}{0.793561in}}%
\pgfpathlineto{\pgfqpoint{3.698801in}{0.821434in}}%
\pgfpathlineto{\pgfqpoint{3.739168in}{0.842029in}}%
\pgfpathlineto{\pgfqpoint{3.779535in}{0.864484in}}%
\pgfpathlineto{\pgfqpoint{3.819902in}{0.889060in}}%
\pgfpathlineto{\pgfqpoint{3.860269in}{0.916065in}}%
\pgfpathlineto{\pgfqpoint{3.900636in}{0.945874in}}%
\pgfpathlineto{\pgfqpoint{3.941003in}{0.978942in}}%
\pgfpathlineto{\pgfqpoint{3.981370in}{1.015827in}}%
\pgfpathlineto{\pgfqpoint{4.021738in}{1.057222in}}%
\pgfpathlineto{\pgfqpoint{4.062105in}{1.104000in}}%
\pgfpathlineto{\pgfqpoint{4.102472in}{1.157276in}}%
\pgfpathlineto{\pgfqpoint{4.122655in}{1.186785in}}%
\pgfpathlineto{\pgfqpoint{4.142839in}{1.218492in}}%
\pgfpathlineto{\pgfqpoint{4.163023in}{1.252651in}}%
\pgfpathlineto{\pgfqpoint{4.183206in}{1.289555in}}%
\pgfpathlineto{\pgfqpoint{4.203390in}{1.329548in}}%
\pgfpathlineto{\pgfqpoint{4.223573in}{1.373033in}}%
\pgfpathlineto{\pgfqpoint{4.243757in}{1.420485in}}%
\pgfpathlineto{\pgfqpoint{4.263940in}{1.472472in}}%
\pgfpathlineto{\pgfqpoint{4.284124in}{1.529674in}}%
\pgfpathlineto{\pgfqpoint{4.304307in}{1.592914in}}%
\pgfpathlineto{\pgfqpoint{4.324491in}{1.663198in}}%
\pgfpathlineto{\pgfqpoint{4.344675in}{1.741770in}}%
\pgfpathlineto{\pgfqpoint{4.364858in}{1.830183in}}%
\pgfpathlineto{\pgfqpoint{4.385042in}{1.930406in}}%
\pgfpathlineto{\pgfqpoint{4.405225in}{2.044969in}}%
\pgfpathlineto{\pgfqpoint{4.425409in}{2.177182in}}%
\pgfpathlineto{\pgfqpoint{4.445592in}{2.331457in}}%
\pgfpathlineto{\pgfqpoint{4.465776in}{2.513811in}}%
\pgfpathlineto{\pgfqpoint{4.485960in}{2.732668in}}%
\pgfpathlineto{\pgfqpoint{4.485960in}{2.732668in}}%
\pgfusepath{stroke}%
\end{pgfscope}%
\begin{pgfscope}%
\pgfpathrectangle{\pgfqpoint{0.651083in}{0.321913in}}{\pgfqpoint{4.036712in}{2.575304in}} %
\pgfusepath{clip}%
\pgfsetrectcap%
\pgfsetroundjoin%
\pgfsetlinewidth{1.003750pt}%
\definecolor{currentstroke}{rgb}{0.000000,0.500000,0.000000}%
\pgfsetstrokecolor{currentstroke}%
\pgfsetdash{}{0pt}%
\pgfpathmoveto{\pgfqpoint{0.852918in}{0.574783in}}%
\pgfpathlineto{\pgfqpoint{1.135488in}{0.570144in}}%
\pgfpathlineto{\pgfqpoint{1.438242in}{0.567425in}}%
\pgfpathlineto{\pgfqpoint{1.740995in}{0.566916in}}%
\pgfpathlineto{\pgfqpoint{2.043748in}{0.568595in}}%
\pgfpathlineto{\pgfqpoint{2.346502in}{0.572536in}}%
\pgfpathlineto{\pgfqpoint{2.629072in}{0.578416in}}%
\pgfpathlineto{\pgfqpoint{2.891458in}{0.586000in}}%
\pgfpathlineto{\pgfqpoint{3.133661in}{0.595072in}}%
\pgfpathlineto{\pgfqpoint{3.355680in}{0.605417in}}%
\pgfpathlineto{\pgfqpoint{3.577699in}{0.618046in}}%
\pgfpathlineto{\pgfqpoint{3.779535in}{0.631885in}}%
\pgfpathlineto{\pgfqpoint{3.961187in}{0.646640in}}%
\pgfpathlineto{\pgfqpoint{4.122655in}{0.661942in}}%
\pgfpathlineto{\pgfqpoint{4.263940in}{0.677335in}}%
\pgfpathlineto{\pgfqpoint{4.405225in}{0.694943in}}%
\pgfpathlineto{\pgfqpoint{4.485960in}{0.706156in}}%
\pgfpathlineto{\pgfqpoint{4.485960in}{0.706156in}}%
\pgfusepath{stroke}%
\end{pgfscope}%
\begin{pgfscope}%
\pgfpathrectangle{\pgfqpoint{0.651083in}{0.321913in}}{\pgfqpoint{4.036712in}{2.575304in}} %
\pgfusepath{clip}%
\pgfsetrectcap%
\pgfsetroundjoin%
\pgfsetlinewidth{1.003750pt}%
\definecolor{currentstroke}{rgb}{1.000000,0.000000,0.000000}%
\pgfsetstrokecolor{currentstroke}%
\pgfsetdash{}{0pt}%
\pgfpathmoveto{\pgfqpoint{0.852918in}{0.706156in}}%
\pgfpathlineto{\pgfqpoint{0.974020in}{0.689663in}}%
\pgfpathlineto{\pgfqpoint{1.115305in}{0.672727in}}%
\pgfpathlineto{\pgfqpoint{1.256590in}{0.657903in}}%
\pgfpathlineto{\pgfqpoint{1.418058in}{0.643152in}}%
\pgfpathlineto{\pgfqpoint{1.599710in}{0.628917in}}%
\pgfpathlineto{\pgfqpoint{1.801546in}{0.615562in}}%
\pgfpathlineto{\pgfqpoint{2.003381in}{0.604388in}}%
\pgfpathlineto{\pgfqpoint{2.225401in}{0.594233in}}%
\pgfpathlineto{\pgfqpoint{2.467603in}{0.585338in}}%
\pgfpathlineto{\pgfqpoint{2.729990in}{0.577921in}}%
\pgfpathlineto{\pgfqpoint{3.012559in}{0.572200in}}%
\pgfpathlineto{\pgfqpoint{3.315313in}{0.568414in}}%
\pgfpathlineto{\pgfqpoint{3.618066in}{0.566882in}}%
\pgfpathlineto{\pgfqpoint{3.920820in}{0.567537in}}%
\pgfpathlineto{\pgfqpoint{4.223573in}{0.570406in}}%
\pgfpathlineto{\pgfqpoint{4.485960in}{0.574783in}}%
\pgfpathlineto{\pgfqpoint{4.485960in}{0.574783in}}%
\pgfusepath{stroke}%
\end{pgfscope}%
\begin{pgfscope}%
\pgfpathrectangle{\pgfqpoint{0.651083in}{0.321913in}}{\pgfqpoint{4.036712in}{2.575304in}} %
\pgfusepath{clip}%
\pgfsetrectcap%
\pgfsetroundjoin%
\pgfsetlinewidth{1.003750pt}%
\definecolor{currentstroke}{rgb}{0.000000,0.750000,0.750000}%
\pgfsetstrokecolor{currentstroke}%
\pgfsetdash{}{0pt}%
\pgfpathmoveto{\pgfqpoint{0.852918in}{2.732668in}}%
\pgfpathlineto{\pgfqpoint{0.873102in}{2.513811in}}%
\pgfpathlineto{\pgfqpoint{0.893285in}{2.331457in}}%
\pgfpathlineto{\pgfqpoint{0.913469in}{2.177182in}}%
\pgfpathlineto{\pgfqpoint{0.933653in}{2.044969in}}%
\pgfpathlineto{\pgfqpoint{0.953836in}{1.930406in}}%
\pgfpathlineto{\pgfqpoint{0.974020in}{1.830183in}}%
\pgfpathlineto{\pgfqpoint{0.994203in}{1.741770in}}%
\pgfpathlineto{\pgfqpoint{1.014387in}{1.663198in}}%
\pgfpathlineto{\pgfqpoint{1.034570in}{1.592914in}}%
\pgfpathlineto{\pgfqpoint{1.054754in}{1.529674in}}%
\pgfpathlineto{\pgfqpoint{1.074937in}{1.472472in}}%
\pgfpathlineto{\pgfqpoint{1.095121in}{1.420485in}}%
\pgfpathlineto{\pgfqpoint{1.115305in}{1.373033in}}%
\pgfpathlineto{\pgfqpoint{1.135488in}{1.329548in}}%
\pgfpathlineto{\pgfqpoint{1.155672in}{1.289555in}}%
\pgfpathlineto{\pgfqpoint{1.175855in}{1.252651in}}%
\pgfpathlineto{\pgfqpoint{1.196039in}{1.218492in}}%
\pgfpathlineto{\pgfqpoint{1.216222in}{1.186785in}}%
\pgfpathlineto{\pgfqpoint{1.236406in}{1.157276in}}%
\pgfpathlineto{\pgfqpoint{1.276773in}{1.104000in}}%
\pgfpathlineto{\pgfqpoint{1.317140in}{1.057222in}}%
\pgfpathlineto{\pgfqpoint{1.357507in}{1.015827in}}%
\pgfpathlineto{\pgfqpoint{1.397874in}{0.978942in}}%
\pgfpathlineto{\pgfqpoint{1.438242in}{0.945874in}}%
\pgfpathlineto{\pgfqpoint{1.478609in}{0.916065in}}%
\pgfpathlineto{\pgfqpoint{1.518976in}{0.889060in}}%
\pgfpathlineto{\pgfqpoint{1.559343in}{0.864484in}}%
\pgfpathlineto{\pgfqpoint{1.599710in}{0.842029in}}%
\pgfpathlineto{\pgfqpoint{1.640077in}{0.821434in}}%
\pgfpathlineto{\pgfqpoint{1.700628in}{0.793561in}}%
\pgfpathlineto{\pgfqpoint{1.761179in}{0.768785in}}%
\pgfpathlineto{\pgfqpoint{1.821729in}{0.746625in}}%
\pgfpathlineto{\pgfqpoint{1.882280in}{0.726697in}}%
\pgfpathlineto{\pgfqpoint{1.942831in}{0.708685in}}%
\pgfpathlineto{\pgfqpoint{2.023565in}{0.687216in}}%
\pgfpathlineto{\pgfqpoint{2.104299in}{0.668212in}}%
\pgfpathlineto{\pgfqpoint{2.185033in}{0.651285in}}%
\pgfpathlineto{\pgfqpoint{2.285951in}{0.632580in}}%
\pgfpathlineto{\pgfqpoint{2.386869in}{0.616160in}}%
\pgfpathlineto{\pgfqpoint{2.507970in}{0.598952in}}%
\pgfpathlineto{\pgfqpoint{2.629072in}{0.583997in}}%
\pgfpathlineto{\pgfqpoint{2.770357in}{0.568888in}}%
\pgfpathlineto{\pgfqpoint{2.931825in}{0.554139in}}%
\pgfpathlineto{\pgfqpoint{3.093294in}{0.541576in}}%
\pgfpathlineto{\pgfqpoint{3.274946in}{0.529566in}}%
\pgfpathlineto{\pgfqpoint{3.476781in}{0.518370in}}%
\pgfpathlineto{\pgfqpoint{3.698801in}{0.508173in}}%
\pgfpathlineto{\pgfqpoint{3.961187in}{0.498435in}}%
\pgfpathlineto{\pgfqpoint{4.243757in}{0.490216in}}%
\pgfpathlineto{\pgfqpoint{4.485960in}{0.484718in}}%
\pgfpathlineto{\pgfqpoint{4.485960in}{0.484718in}}%
\pgfusepath{stroke}%
\end{pgfscope}%
\begin{pgfscope}%
\pgfsetbuttcap%
\pgfsetroundjoin%
\definecolor{currentfill}{rgb}{0.000000,0.000000,0.000000}%
\pgfsetfillcolor{currentfill}%
\pgfsetlinewidth{0.501875pt}%
\definecolor{currentstroke}{rgb}{0.000000,0.000000,0.000000}%
\pgfsetstrokecolor{currentstroke}%
\pgfsetdash{}{0pt}%
\pgfsys@defobject{currentmarker}{\pgfqpoint{0.000000in}{0.000000in}}{\pgfqpoint{0.000000in}{0.055556in}}{%
\pgfpathmoveto{\pgfqpoint{0.000000in}{0.000000in}}%
\pgfpathlineto{\pgfqpoint{0.000000in}{0.055556in}}%
\pgfusepath{stroke,fill}%
}%
\begin{pgfscope}%
\pgfsys@transformshift{0.651083in}{0.321913in}%
\pgfsys@useobject{currentmarker}{}%
\end{pgfscope}%
\end{pgfscope}%
\begin{pgfscope}%
\pgfsetbuttcap%
\pgfsetroundjoin%
\definecolor{currentfill}{rgb}{0.000000,0.000000,0.000000}%
\pgfsetfillcolor{currentfill}%
\pgfsetlinewidth{0.501875pt}%
\definecolor{currentstroke}{rgb}{0.000000,0.000000,0.000000}%
\pgfsetstrokecolor{currentstroke}%
\pgfsetdash{}{0pt}%
\pgfsys@defobject{currentmarker}{\pgfqpoint{0.000000in}{-0.055556in}}{\pgfqpoint{0.000000in}{0.000000in}}{%
\pgfpathmoveto{\pgfqpoint{0.000000in}{0.000000in}}%
\pgfpathlineto{\pgfqpoint{0.000000in}{-0.055556in}}%
\pgfusepath{stroke,fill}%
}%
\begin{pgfscope}%
\pgfsys@transformshift{0.651083in}{2.897217in}%
\pgfsys@useobject{currentmarker}{}%
\end{pgfscope}%
\end{pgfscope}%
\begin{pgfscope}%
\pgftext[x=0.651083in,y=0.266357in,,top]{{\rmfamily\fontsize{8.000000}{9.600000}\selectfont \(\displaystyle -1.0\)}}%
\end{pgfscope}%
\begin{pgfscope}%
\pgfsetbuttcap%
\pgfsetroundjoin%
\definecolor{currentfill}{rgb}{0.000000,0.000000,0.000000}%
\pgfsetfillcolor{currentfill}%
\pgfsetlinewidth{0.501875pt}%
\definecolor{currentstroke}{rgb}{0.000000,0.000000,0.000000}%
\pgfsetstrokecolor{currentstroke}%
\pgfsetdash{}{0pt}%
\pgfsys@defobject{currentmarker}{\pgfqpoint{0.000000in}{0.000000in}}{\pgfqpoint{0.000000in}{0.055556in}}{%
\pgfpathmoveto{\pgfqpoint{0.000000in}{0.000000in}}%
\pgfpathlineto{\pgfqpoint{0.000000in}{0.055556in}}%
\pgfusepath{stroke,fill}%
}%
\begin{pgfscope}%
\pgfsys@transformshift{1.660261in}{0.321913in}%
\pgfsys@useobject{currentmarker}{}%
\end{pgfscope}%
\end{pgfscope}%
\begin{pgfscope}%
\pgfsetbuttcap%
\pgfsetroundjoin%
\definecolor{currentfill}{rgb}{0.000000,0.000000,0.000000}%
\pgfsetfillcolor{currentfill}%
\pgfsetlinewidth{0.501875pt}%
\definecolor{currentstroke}{rgb}{0.000000,0.000000,0.000000}%
\pgfsetstrokecolor{currentstroke}%
\pgfsetdash{}{0pt}%
\pgfsys@defobject{currentmarker}{\pgfqpoint{0.000000in}{-0.055556in}}{\pgfqpoint{0.000000in}{0.000000in}}{%
\pgfpathmoveto{\pgfqpoint{0.000000in}{0.000000in}}%
\pgfpathlineto{\pgfqpoint{0.000000in}{-0.055556in}}%
\pgfusepath{stroke,fill}%
}%
\begin{pgfscope}%
\pgfsys@transformshift{1.660261in}{2.897217in}%
\pgfsys@useobject{currentmarker}{}%
\end{pgfscope}%
\end{pgfscope}%
\begin{pgfscope}%
\pgftext[x=1.660261in,y=0.266357in,,top]{{\rmfamily\fontsize{8.000000}{9.600000}\selectfont \(\displaystyle -0.5\)}}%
\end{pgfscope}%
\begin{pgfscope}%
\pgfsetbuttcap%
\pgfsetroundjoin%
\definecolor{currentfill}{rgb}{0.000000,0.000000,0.000000}%
\pgfsetfillcolor{currentfill}%
\pgfsetlinewidth{0.501875pt}%
\definecolor{currentstroke}{rgb}{0.000000,0.000000,0.000000}%
\pgfsetstrokecolor{currentstroke}%
\pgfsetdash{}{0pt}%
\pgfsys@defobject{currentmarker}{\pgfqpoint{0.000000in}{0.000000in}}{\pgfqpoint{0.000000in}{0.055556in}}{%
\pgfpathmoveto{\pgfqpoint{0.000000in}{0.000000in}}%
\pgfpathlineto{\pgfqpoint{0.000000in}{0.055556in}}%
\pgfusepath{stroke,fill}%
}%
\begin{pgfscope}%
\pgfsys@transformshift{2.669439in}{0.321913in}%
\pgfsys@useobject{currentmarker}{}%
\end{pgfscope}%
\end{pgfscope}%
\begin{pgfscope}%
\pgfsetbuttcap%
\pgfsetroundjoin%
\definecolor{currentfill}{rgb}{0.000000,0.000000,0.000000}%
\pgfsetfillcolor{currentfill}%
\pgfsetlinewidth{0.501875pt}%
\definecolor{currentstroke}{rgb}{0.000000,0.000000,0.000000}%
\pgfsetstrokecolor{currentstroke}%
\pgfsetdash{}{0pt}%
\pgfsys@defobject{currentmarker}{\pgfqpoint{0.000000in}{-0.055556in}}{\pgfqpoint{0.000000in}{0.000000in}}{%
\pgfpathmoveto{\pgfqpoint{0.000000in}{0.000000in}}%
\pgfpathlineto{\pgfqpoint{0.000000in}{-0.055556in}}%
\pgfusepath{stroke,fill}%
}%
\begin{pgfscope}%
\pgfsys@transformshift{2.669439in}{2.897217in}%
\pgfsys@useobject{currentmarker}{}%
\end{pgfscope}%
\end{pgfscope}%
\begin{pgfscope}%
\pgftext[x=2.669439in,y=0.266357in,,top]{{\rmfamily\fontsize{8.000000}{9.600000}\selectfont \(\displaystyle 0.0\)}}%
\end{pgfscope}%
\begin{pgfscope}%
\pgfsetbuttcap%
\pgfsetroundjoin%
\definecolor{currentfill}{rgb}{0.000000,0.000000,0.000000}%
\pgfsetfillcolor{currentfill}%
\pgfsetlinewidth{0.501875pt}%
\definecolor{currentstroke}{rgb}{0.000000,0.000000,0.000000}%
\pgfsetstrokecolor{currentstroke}%
\pgfsetdash{}{0pt}%
\pgfsys@defobject{currentmarker}{\pgfqpoint{0.000000in}{0.000000in}}{\pgfqpoint{0.000000in}{0.055556in}}{%
\pgfpathmoveto{\pgfqpoint{0.000000in}{0.000000in}}%
\pgfpathlineto{\pgfqpoint{0.000000in}{0.055556in}}%
\pgfusepath{stroke,fill}%
}%
\begin{pgfscope}%
\pgfsys@transformshift{3.678617in}{0.321913in}%
\pgfsys@useobject{currentmarker}{}%
\end{pgfscope}%
\end{pgfscope}%
\begin{pgfscope}%
\pgfsetbuttcap%
\pgfsetroundjoin%
\definecolor{currentfill}{rgb}{0.000000,0.000000,0.000000}%
\pgfsetfillcolor{currentfill}%
\pgfsetlinewidth{0.501875pt}%
\definecolor{currentstroke}{rgb}{0.000000,0.000000,0.000000}%
\pgfsetstrokecolor{currentstroke}%
\pgfsetdash{}{0pt}%
\pgfsys@defobject{currentmarker}{\pgfqpoint{0.000000in}{-0.055556in}}{\pgfqpoint{0.000000in}{0.000000in}}{%
\pgfpathmoveto{\pgfqpoint{0.000000in}{0.000000in}}%
\pgfpathlineto{\pgfqpoint{0.000000in}{-0.055556in}}%
\pgfusepath{stroke,fill}%
}%
\begin{pgfscope}%
\pgfsys@transformshift{3.678617in}{2.897217in}%
\pgfsys@useobject{currentmarker}{}%
\end{pgfscope}%
\end{pgfscope}%
\begin{pgfscope}%
\pgftext[x=3.678617in,y=0.266357in,,top]{{\rmfamily\fontsize{8.000000}{9.600000}\selectfont \(\displaystyle 0.5\)}}%
\end{pgfscope}%
\begin{pgfscope}%
\pgfsetbuttcap%
\pgfsetroundjoin%
\definecolor{currentfill}{rgb}{0.000000,0.000000,0.000000}%
\pgfsetfillcolor{currentfill}%
\pgfsetlinewidth{0.501875pt}%
\definecolor{currentstroke}{rgb}{0.000000,0.000000,0.000000}%
\pgfsetstrokecolor{currentstroke}%
\pgfsetdash{}{0pt}%
\pgfsys@defobject{currentmarker}{\pgfqpoint{0.000000in}{0.000000in}}{\pgfqpoint{0.000000in}{0.055556in}}{%
\pgfpathmoveto{\pgfqpoint{0.000000in}{0.000000in}}%
\pgfpathlineto{\pgfqpoint{0.000000in}{0.055556in}}%
\pgfusepath{stroke,fill}%
}%
\begin{pgfscope}%
\pgfsys@transformshift{4.687795in}{0.321913in}%
\pgfsys@useobject{currentmarker}{}%
\end{pgfscope}%
\end{pgfscope}%
\begin{pgfscope}%
\pgfsetbuttcap%
\pgfsetroundjoin%
\definecolor{currentfill}{rgb}{0.000000,0.000000,0.000000}%
\pgfsetfillcolor{currentfill}%
\pgfsetlinewidth{0.501875pt}%
\definecolor{currentstroke}{rgb}{0.000000,0.000000,0.000000}%
\pgfsetstrokecolor{currentstroke}%
\pgfsetdash{}{0pt}%
\pgfsys@defobject{currentmarker}{\pgfqpoint{0.000000in}{-0.055556in}}{\pgfqpoint{0.000000in}{0.000000in}}{%
\pgfpathmoveto{\pgfqpoint{0.000000in}{0.000000in}}%
\pgfpathlineto{\pgfqpoint{0.000000in}{-0.055556in}}%
\pgfusepath{stroke,fill}%
}%
\begin{pgfscope}%
\pgfsys@transformshift{4.687795in}{2.897217in}%
\pgfsys@useobject{currentmarker}{}%
\end{pgfscope}%
\end{pgfscope}%
\begin{pgfscope}%
\pgftext[x=4.687795in,y=0.266357in,,top]{{\rmfamily\fontsize{8.000000}{9.600000}\selectfont \(\displaystyle 1.0\)}}%
\end{pgfscope}%
\begin{pgfscope}%
\pgftext[x=2.669439in,y=0.098789in,,top]{{\rmfamily\fontsize{10.000000}{12.000000}\selectfont \(\displaystyle  \delta \)}}%
\end{pgfscope}%
\begin{pgfscope}%
\pgfsetbuttcap%
\pgfsetroundjoin%
\definecolor{currentfill}{rgb}{0.000000,0.000000,0.000000}%
\pgfsetfillcolor{currentfill}%
\pgfsetlinewidth{0.501875pt}%
\definecolor{currentstroke}{rgb}{0.000000,0.000000,0.000000}%
\pgfsetstrokecolor{currentstroke}%
\pgfsetdash{}{0pt}%
\pgfsys@defobject{currentmarker}{\pgfqpoint{0.000000in}{0.000000in}}{\pgfqpoint{0.055556in}{0.000000in}}{%
\pgfpathmoveto{\pgfqpoint{0.000000in}{0.000000in}}%
\pgfpathlineto{\pgfqpoint{0.055556in}{0.000000in}}%
\pgfusepath{stroke,fill}%
}%
\begin{pgfscope}%
\pgfsys@transformshift{0.651083in}{0.321913in}%
\pgfsys@useobject{currentmarker}{}%
\end{pgfscope}%
\end{pgfscope}%
\begin{pgfscope}%
\pgfsetbuttcap%
\pgfsetroundjoin%
\definecolor{currentfill}{rgb}{0.000000,0.000000,0.000000}%
\pgfsetfillcolor{currentfill}%
\pgfsetlinewidth{0.501875pt}%
\definecolor{currentstroke}{rgb}{0.000000,0.000000,0.000000}%
\pgfsetstrokecolor{currentstroke}%
\pgfsetdash{}{0pt}%
\pgfsys@defobject{currentmarker}{\pgfqpoint{-0.055556in}{0.000000in}}{\pgfqpoint{0.000000in}{0.000000in}}{%
\pgfpathmoveto{\pgfqpoint{0.000000in}{0.000000in}}%
\pgfpathlineto{\pgfqpoint{-0.055556in}{0.000000in}}%
\pgfusepath{stroke,fill}%
}%
\begin{pgfscope}%
\pgfsys@transformshift{4.687795in}{0.321913in}%
\pgfsys@useobject{currentmarker}{}%
\end{pgfscope}%
\end{pgfscope}%
\begin{pgfscope}%
\pgftext[x=0.595527in,y=0.321913in,right,]{{\rmfamily\fontsize{8.000000}{9.600000}\selectfont \(\displaystyle 0.0\)}}%
\end{pgfscope}%
\begin{pgfscope}%
\pgfsetbuttcap%
\pgfsetroundjoin%
\definecolor{currentfill}{rgb}{0.000000,0.000000,0.000000}%
\pgfsetfillcolor{currentfill}%
\pgfsetlinewidth{0.501875pt}%
\definecolor{currentstroke}{rgb}{0.000000,0.000000,0.000000}%
\pgfsetstrokecolor{currentstroke}%
\pgfsetdash{}{0pt}%
\pgfsys@defobject{currentmarker}{\pgfqpoint{0.000000in}{0.000000in}}{\pgfqpoint{0.055556in}{0.000000in}}{%
\pgfpathmoveto{\pgfqpoint{0.000000in}{0.000000in}}%
\pgfpathlineto{\pgfqpoint{0.055556in}{0.000000in}}%
\pgfusepath{stroke,fill}%
}%
\begin{pgfscope}%
\pgfsys@transformshift{0.651083in}{0.965739in}%
\pgfsys@useobject{currentmarker}{}%
\end{pgfscope}%
\end{pgfscope}%
\begin{pgfscope}%
\pgfsetbuttcap%
\pgfsetroundjoin%
\definecolor{currentfill}{rgb}{0.000000,0.000000,0.000000}%
\pgfsetfillcolor{currentfill}%
\pgfsetlinewidth{0.501875pt}%
\definecolor{currentstroke}{rgb}{0.000000,0.000000,0.000000}%
\pgfsetstrokecolor{currentstroke}%
\pgfsetdash{}{0pt}%
\pgfsys@defobject{currentmarker}{\pgfqpoint{-0.055556in}{0.000000in}}{\pgfqpoint{0.000000in}{0.000000in}}{%
\pgfpathmoveto{\pgfqpoint{0.000000in}{0.000000in}}%
\pgfpathlineto{\pgfqpoint{-0.055556in}{0.000000in}}%
\pgfusepath{stroke,fill}%
}%
\begin{pgfscope}%
\pgfsys@transformshift{4.687795in}{0.965739in}%
\pgfsys@useobject{currentmarker}{}%
\end{pgfscope}%
\end{pgfscope}%
\begin{pgfscope}%
\pgftext[x=0.595527in,y=0.965739in,right,]{{\rmfamily\fontsize{8.000000}{9.600000}\selectfont \(\displaystyle 0.5\)}}%
\end{pgfscope}%
\begin{pgfscope}%
\pgfsetbuttcap%
\pgfsetroundjoin%
\definecolor{currentfill}{rgb}{0.000000,0.000000,0.000000}%
\pgfsetfillcolor{currentfill}%
\pgfsetlinewidth{0.501875pt}%
\definecolor{currentstroke}{rgb}{0.000000,0.000000,0.000000}%
\pgfsetstrokecolor{currentstroke}%
\pgfsetdash{}{0pt}%
\pgfsys@defobject{currentmarker}{\pgfqpoint{0.000000in}{0.000000in}}{\pgfqpoint{0.055556in}{0.000000in}}{%
\pgfpathmoveto{\pgfqpoint{0.000000in}{0.000000in}}%
\pgfpathlineto{\pgfqpoint{0.055556in}{0.000000in}}%
\pgfusepath{stroke,fill}%
}%
\begin{pgfscope}%
\pgfsys@transformshift{0.651083in}{1.609565in}%
\pgfsys@useobject{currentmarker}{}%
\end{pgfscope}%
\end{pgfscope}%
\begin{pgfscope}%
\pgfsetbuttcap%
\pgfsetroundjoin%
\definecolor{currentfill}{rgb}{0.000000,0.000000,0.000000}%
\pgfsetfillcolor{currentfill}%
\pgfsetlinewidth{0.501875pt}%
\definecolor{currentstroke}{rgb}{0.000000,0.000000,0.000000}%
\pgfsetstrokecolor{currentstroke}%
\pgfsetdash{}{0pt}%
\pgfsys@defobject{currentmarker}{\pgfqpoint{-0.055556in}{0.000000in}}{\pgfqpoint{0.000000in}{0.000000in}}{%
\pgfpathmoveto{\pgfqpoint{0.000000in}{0.000000in}}%
\pgfpathlineto{\pgfqpoint{-0.055556in}{0.000000in}}%
\pgfusepath{stroke,fill}%
}%
\begin{pgfscope}%
\pgfsys@transformshift{4.687795in}{1.609565in}%
\pgfsys@useobject{currentmarker}{}%
\end{pgfscope}%
\end{pgfscope}%
\begin{pgfscope}%
\pgftext[x=0.595527in,y=1.609565in,right,]{{\rmfamily\fontsize{8.000000}{9.600000}\selectfont \(\displaystyle 1.0\)}}%
\end{pgfscope}%
\begin{pgfscope}%
\pgfsetbuttcap%
\pgfsetroundjoin%
\definecolor{currentfill}{rgb}{0.000000,0.000000,0.000000}%
\pgfsetfillcolor{currentfill}%
\pgfsetlinewidth{0.501875pt}%
\definecolor{currentstroke}{rgb}{0.000000,0.000000,0.000000}%
\pgfsetstrokecolor{currentstroke}%
\pgfsetdash{}{0pt}%
\pgfsys@defobject{currentmarker}{\pgfqpoint{0.000000in}{0.000000in}}{\pgfqpoint{0.055556in}{0.000000in}}{%
\pgfpathmoveto{\pgfqpoint{0.000000in}{0.000000in}}%
\pgfpathlineto{\pgfqpoint{0.055556in}{0.000000in}}%
\pgfusepath{stroke,fill}%
}%
\begin{pgfscope}%
\pgfsys@transformshift{0.651083in}{2.253391in}%
\pgfsys@useobject{currentmarker}{}%
\end{pgfscope}%
\end{pgfscope}%
\begin{pgfscope}%
\pgfsetbuttcap%
\pgfsetroundjoin%
\definecolor{currentfill}{rgb}{0.000000,0.000000,0.000000}%
\pgfsetfillcolor{currentfill}%
\pgfsetlinewidth{0.501875pt}%
\definecolor{currentstroke}{rgb}{0.000000,0.000000,0.000000}%
\pgfsetstrokecolor{currentstroke}%
\pgfsetdash{}{0pt}%
\pgfsys@defobject{currentmarker}{\pgfqpoint{-0.055556in}{0.000000in}}{\pgfqpoint{0.000000in}{0.000000in}}{%
\pgfpathmoveto{\pgfqpoint{0.000000in}{0.000000in}}%
\pgfpathlineto{\pgfqpoint{-0.055556in}{0.000000in}}%
\pgfusepath{stroke,fill}%
}%
\begin{pgfscope}%
\pgfsys@transformshift{4.687795in}{2.253391in}%
\pgfsys@useobject{currentmarker}{}%
\end{pgfscope}%
\end{pgfscope}%
\begin{pgfscope}%
\pgftext[x=0.595527in,y=2.253391in,right,]{{\rmfamily\fontsize{8.000000}{9.600000}\selectfont \(\displaystyle 1.5\)}}%
\end{pgfscope}%
\begin{pgfscope}%
\pgfsetbuttcap%
\pgfsetroundjoin%
\definecolor{currentfill}{rgb}{0.000000,0.000000,0.000000}%
\pgfsetfillcolor{currentfill}%
\pgfsetlinewidth{0.501875pt}%
\definecolor{currentstroke}{rgb}{0.000000,0.000000,0.000000}%
\pgfsetstrokecolor{currentstroke}%
\pgfsetdash{}{0pt}%
\pgfsys@defobject{currentmarker}{\pgfqpoint{0.000000in}{0.000000in}}{\pgfqpoint{0.055556in}{0.000000in}}{%
\pgfpathmoveto{\pgfqpoint{0.000000in}{0.000000in}}%
\pgfpathlineto{\pgfqpoint{0.055556in}{0.000000in}}%
\pgfusepath{stroke,fill}%
}%
\begin{pgfscope}%
\pgfsys@transformshift{0.651083in}{2.897217in}%
\pgfsys@useobject{currentmarker}{}%
\end{pgfscope}%
\end{pgfscope}%
\begin{pgfscope}%
\pgfsetbuttcap%
\pgfsetroundjoin%
\definecolor{currentfill}{rgb}{0.000000,0.000000,0.000000}%
\pgfsetfillcolor{currentfill}%
\pgfsetlinewidth{0.501875pt}%
\definecolor{currentstroke}{rgb}{0.000000,0.000000,0.000000}%
\pgfsetstrokecolor{currentstroke}%
\pgfsetdash{}{0pt}%
\pgfsys@defobject{currentmarker}{\pgfqpoint{-0.055556in}{0.000000in}}{\pgfqpoint{0.000000in}{0.000000in}}{%
\pgfpathmoveto{\pgfqpoint{0.000000in}{0.000000in}}%
\pgfpathlineto{\pgfqpoint{-0.055556in}{0.000000in}}%
\pgfusepath{stroke,fill}%
}%
\begin{pgfscope}%
\pgfsys@transformshift{4.687795in}{2.897217in}%
\pgfsys@useobject{currentmarker}{}%
\end{pgfscope}%
\end{pgfscope}%
\begin{pgfscope}%
\pgftext[x=0.595527in,y=2.897217in,right,]{{\rmfamily\fontsize{8.000000}{9.600000}\selectfont \(\displaystyle 2.0\)}}%
\end{pgfscope}%
\begin{pgfscope}%
\pgftext[x=0.375232in,y=1.609565in,,bottom,rotate=90.000000]{{\rmfamily\fontsize{10.000000}{12.000000}\selectfont \(\displaystyle  f_j(\delta) = \Pi_j(\delta) \)}}%
\end{pgfscope}%
\begin{pgfscope}%
\pgfsetbuttcap%
\pgfsetroundjoin%
\pgfsetlinewidth{1.003750pt}%
\definecolor{currentstroke}{rgb}{0.000000,0.000000,0.000000}%
\pgfsetstrokecolor{currentstroke}%
\pgfsetdash{}{0pt}%
\pgfpathmoveto{\pgfqpoint{0.651083in}{2.897217in}}%
\pgfpathlineto{\pgfqpoint{4.687795in}{2.897217in}}%
\pgfusepath{stroke}%
\end{pgfscope}%
\begin{pgfscope}%
\pgfsetbuttcap%
\pgfsetroundjoin%
\pgfsetlinewidth{1.003750pt}%
\definecolor{currentstroke}{rgb}{0.000000,0.000000,0.000000}%
\pgfsetstrokecolor{currentstroke}%
\pgfsetdash{}{0pt}%
\pgfpathmoveto{\pgfqpoint{4.687795in}{0.321913in}}%
\pgfpathlineto{\pgfqpoint{4.687795in}{2.897217in}}%
\pgfusepath{stroke}%
\end{pgfscope}%
\begin{pgfscope}%
\pgfsetbuttcap%
\pgfsetroundjoin%
\pgfsetlinewidth{1.003750pt}%
\definecolor{currentstroke}{rgb}{0.000000,0.000000,0.000000}%
\pgfsetstrokecolor{currentstroke}%
\pgfsetdash{}{0pt}%
\pgfpathmoveto{\pgfqpoint{0.651083in}{0.321913in}}%
\pgfpathlineto{\pgfqpoint{4.687795in}{0.321913in}}%
\pgfusepath{stroke}%
\end{pgfscope}%
\begin{pgfscope}%
\pgfsetbuttcap%
\pgfsetroundjoin%
\pgfsetlinewidth{1.003750pt}%
\definecolor{currentstroke}{rgb}{0.000000,0.000000,0.000000}%
\pgfsetstrokecolor{currentstroke}%
\pgfsetdash{}{0pt}%
\pgfpathmoveto{\pgfqpoint{0.651083in}{0.321913in}}%
\pgfpathlineto{\pgfqpoint{0.651083in}{2.897217in}}%
\pgfusepath{stroke}%
\end{pgfscope}%
\begin{pgfscope}%
\pgfsetbuttcap%
\pgfsetroundjoin%
\definecolor{currentfill}{rgb}{0.300000,0.300000,0.300000}%
\pgfsetfillcolor{currentfill}%
\pgfsetfillopacity{0.500000}%
\pgfsetlinewidth{1.003750pt}%
\definecolor{currentstroke}{rgb}{0.300000,0.300000,0.300000}%
\pgfsetstrokecolor{currentstroke}%
\pgfsetstrokeopacity{0.500000}%
\pgfsetdash{}{0pt}%
\pgfpathmoveto{\pgfqpoint{1.705792in}{2.341919in}}%
\pgfpathlineto{\pgfqpoint{3.688641in}{2.341919in}}%
\pgfpathquadraticcurveto{\pgfqpoint{3.710864in}{2.341919in}}{\pgfqpoint{3.710864in}{2.364141in}}%
\pgfpathlineto{\pgfqpoint{3.710864in}{2.662896in}}%
\pgfpathquadraticcurveto{\pgfqpoint{3.710864in}{2.685118in}}{\pgfqpoint{3.688641in}{2.685118in}}%
\pgfpathlineto{\pgfqpoint{1.705792in}{2.685118in}}%
\pgfpathquadraticcurveto{\pgfqpoint{1.683570in}{2.685118in}}{\pgfqpoint{1.683570in}{2.662896in}}%
\pgfpathlineto{\pgfqpoint{1.683570in}{2.364141in}}%
\pgfpathquadraticcurveto{\pgfqpoint{1.683570in}{2.341919in}}{\pgfqpoint{1.705792in}{2.341919in}}%
\pgfpathclose%
\pgfusepath{stroke,fill}%
\end{pgfscope}%
\begin{pgfscope}%
\pgfsetbuttcap%
\pgfsetroundjoin%
\definecolor{currentfill}{rgb}{1.000000,1.000000,1.000000}%
\pgfsetfillcolor{currentfill}%
\pgfsetlinewidth{1.003750pt}%
\definecolor{currentstroke}{rgb}{0.000000,0.000000,0.000000}%
\pgfsetstrokecolor{currentstroke}%
\pgfsetdash{}{0pt}%
\pgfpathmoveto{\pgfqpoint{1.678014in}{2.369697in}}%
\pgfpathlineto{\pgfqpoint{3.660864in}{2.369697in}}%
\pgfpathquadraticcurveto{\pgfqpoint{3.683086in}{2.369697in}}{\pgfqpoint{3.683086in}{2.391919in}}%
\pgfpathlineto{\pgfqpoint{3.683086in}{2.690674in}}%
\pgfpathquadraticcurveto{\pgfqpoint{3.683086in}{2.712896in}}{\pgfqpoint{3.660864in}{2.712896in}}%
\pgfpathlineto{\pgfqpoint{1.678014in}{2.712896in}}%
\pgfpathquadraticcurveto{\pgfqpoint{1.655792in}{2.712896in}}{\pgfqpoint{1.655792in}{2.690674in}}%
\pgfpathlineto{\pgfqpoint{1.655792in}{2.391919in}}%
\pgfpathquadraticcurveto{\pgfqpoint{1.655792in}{2.369697in}}{\pgfqpoint{1.678014in}{2.369697in}}%
\pgfpathclose%
\pgfusepath{stroke,fill}%
\end{pgfscope}%
\begin{pgfscope}%
\pgfsetrectcap%
\pgfsetroundjoin%
\pgfsetlinewidth{1.003750pt}%
\definecolor{currentstroke}{rgb}{0.000000,0.000000,0.000000}%
\pgfsetstrokecolor{currentstroke}%
\pgfsetdash{}{0pt}%
\pgfpathmoveto{\pgfqpoint{1.733570in}{2.629563in}}%
\pgfpathlineto{\pgfqpoint{1.889125in}{2.629563in}}%
\pgfusepath{stroke}%
\end{pgfscope}%
\begin{pgfscope}%
\pgftext[x=2.011348in,y=2.590674in,left,base]{{\rmfamily\fontsize{8.000000}{9.600000}\selectfont j=0}}%
\end{pgfscope}%
\begin{pgfscope}%
\pgfsetrectcap%
\pgfsetroundjoin%
\pgfsetlinewidth{1.003750pt}%
\definecolor{currentstroke}{rgb}{0.000000,0.000000,1.000000}%
\pgfsetstrokecolor{currentstroke}%
\pgfsetdash{}{0pt}%
\pgfpathmoveto{\pgfqpoint{1.733570in}{2.474630in}}%
\pgfpathlineto{\pgfqpoint{1.889125in}{2.474630in}}%
\pgfusepath{stroke}%
\end{pgfscope}%
\begin{pgfscope}%
\pgftext[x=2.011348in,y=2.435741in,left,base]{{\rmfamily\fontsize{8.000000}{9.600000}\selectfont j=1}}%
\end{pgfscope}%
\begin{pgfscope}%
\pgfsetrectcap%
\pgfsetroundjoin%
\pgfsetlinewidth{1.003750pt}%
\definecolor{currentstroke}{rgb}{0.000000,0.500000,0.000000}%
\pgfsetstrokecolor{currentstroke}%
\pgfsetdash{}{0pt}%
\pgfpathmoveto{\pgfqpoint{2.453779in}{2.629563in}}%
\pgfpathlineto{\pgfqpoint{2.609334in}{2.629563in}}%
\pgfusepath{stroke}%
\end{pgfscope}%
\begin{pgfscope}%
\pgftext[x=2.731557in,y=2.590674in,left,base]{{\rmfamily\fontsize{8.000000}{9.600000}\selectfont j=2}}%
\end{pgfscope}%
\begin{pgfscope}%
\pgfsetrectcap%
\pgfsetroundjoin%
\pgfsetlinewidth{1.003750pt}%
\definecolor{currentstroke}{rgb}{1.000000,0.000000,0.000000}%
\pgfsetstrokecolor{currentstroke}%
\pgfsetdash{}{0pt}%
\pgfpathmoveto{\pgfqpoint{2.453779in}{2.474630in}}%
\pgfpathlineto{\pgfqpoint{2.609334in}{2.474630in}}%
\pgfusepath{stroke}%
\end{pgfscope}%
\begin{pgfscope}%
\pgftext[x=2.731557in,y=2.435741in,left,base]{{\rmfamily\fontsize{8.000000}{9.600000}\selectfont j=3}}%
\end{pgfscope}%
\begin{pgfscope}%
\pgfsetrectcap%
\pgfsetroundjoin%
\pgfsetlinewidth{1.003750pt}%
\definecolor{currentstroke}{rgb}{0.000000,0.750000,0.750000}%
\pgfsetstrokecolor{currentstroke}%
\pgfsetdash{}{0pt}%
\pgfpathmoveto{\pgfqpoint{3.173988in}{2.629563in}}%
\pgfpathlineto{\pgfqpoint{3.329543in}{2.629563in}}%
\pgfusepath{stroke}%
\end{pgfscope}%
\begin{pgfscope}%
\pgftext[x=3.451766in,y=2.590674in,left,base]{{\rmfamily\fontsize{8.000000}{9.600000}\selectfont j=4}}%
\end{pgfscope}%
\end{pgfpicture}%
\makeatother%
\endgroup%

        \caption{Erster Faktor der Betragssumme der $j$-ten Zeile
                 von $\Vand{\delta}^{-1}$ in Abhängigkeit der Auslenkung
                 $\delta \in (-0.9, 0.9)$
                für den Fall $n=5$.}
        \label{fig:pi_j}
    \end{figure}

    \begin{figure}[H]
        \centering
        %% Creator: Matplotlib, PGF backend
%%
%% To include the figure in your LaTeX document, write
%%   \input{<filename>.pgf}
%%
%% Make sure the required packages are loaded in your preamble
%%   \usepackage{pgf}
%%
%% Figures using additional raster images can only be included by \input if
%% they are in the same directory as the main LaTeX file. For loading figures
%% from other directories you can use the `import` package
%%   \usepackage{import}
%% and then include the figures with
%%   \import{<path to file>}{<filename>.pgf}
%%
%% Matplotlib used the following preamble
%%   \usepackage[utf8x]{inputenc}
%%   \usepackage[T1]{fontenc}
%%
\begingroup%
\makeatletter%
\begin{pgfpicture}%
\pgfpathrectangle{\pgfpointorigin}{\pgfqpoint{5.208661in}{3.219130in}}%
\pgfusepath{use as bounding box}%
\begin{pgfscope}%
\pgfsetbuttcap%
\pgfsetroundjoin%
\definecolor{currentfill}{rgb}{1.000000,1.000000,1.000000}%
\pgfsetfillcolor{currentfill}%
\pgfsetlinewidth{0.000000pt}%
\definecolor{currentstroke}{rgb}{1.000000,1.000000,1.000000}%
\pgfsetstrokecolor{currentstroke}%
\pgfsetdash{}{0pt}%
\pgfpathmoveto{\pgfqpoint{0.000000in}{0.000000in}}%
\pgfpathlineto{\pgfqpoint{5.208661in}{0.000000in}}%
\pgfpathlineto{\pgfqpoint{5.208661in}{3.219130in}}%
\pgfpathlineto{\pgfqpoint{0.000000in}{3.219130in}}%
\pgfpathclose%
\pgfusepath{fill}%
\end{pgfscope}%
\begin{pgfscope}%
\pgfsetbuttcap%
\pgfsetroundjoin%
\definecolor{currentfill}{rgb}{1.000000,1.000000,1.000000}%
\pgfsetfillcolor{currentfill}%
\pgfsetlinewidth{0.000000pt}%
\definecolor{currentstroke}{rgb}{0.000000,0.000000,0.000000}%
\pgfsetstrokecolor{currentstroke}%
\pgfsetstrokeopacity{0.000000}%
\pgfsetdash{}{0pt}%
\pgfpathmoveto{\pgfqpoint{0.651083in}{0.321913in}}%
\pgfpathlineto{\pgfqpoint{4.687795in}{0.321913in}}%
\pgfpathlineto{\pgfqpoint{4.687795in}{2.897217in}}%
\pgfpathlineto{\pgfqpoint{0.651083in}{2.897217in}}%
\pgfpathclose%
\pgfusepath{fill}%
\end{pgfscope}%
\begin{pgfscope}%
\pgfpathrectangle{\pgfqpoint{0.651083in}{0.321913in}}{\pgfqpoint{4.036712in}{2.575304in}} %
\pgfusepath{clip}%
\pgfsetrectcap%
\pgfsetroundjoin%
\pgfsetlinewidth{1.003750pt}%
\definecolor{currentstroke}{rgb}{0.000000,0.000000,0.000000}%
\pgfsetstrokecolor{currentstroke}%
\pgfsetdash{}{0pt}%
\pgfpathmoveto{\pgfqpoint{0.852918in}{1.180348in}}%
\pgfpathlineto{\pgfqpoint{4.485960in}{1.180348in}}%
\pgfpathlineto{\pgfqpoint{4.485960in}{1.180348in}}%
\pgfusepath{stroke}%
\end{pgfscope}%
\begin{pgfscope}%
\pgfpathrectangle{\pgfqpoint{0.651083in}{0.321913in}}{\pgfqpoint{4.036712in}{2.575304in}} %
\pgfusepath{clip}%
\pgfsetrectcap%
\pgfsetroundjoin%
\pgfsetlinewidth{1.003750pt}%
\definecolor{currentstroke}{rgb}{0.000000,0.000000,1.000000}%
\pgfsetstrokecolor{currentstroke}%
\pgfsetdash{}{0pt}%
\pgfpathmoveto{\pgfqpoint{0.852918in}{2.727293in}}%
\pgfpathlineto{\pgfqpoint{0.953836in}{2.656513in}}%
\pgfpathlineto{\pgfqpoint{1.054754in}{2.583281in}}%
\pgfpathlineto{\pgfqpoint{1.155672in}{2.507690in}}%
\pgfpathlineto{\pgfqpoint{1.276773in}{2.414010in}}%
\pgfpathlineto{\pgfqpoint{1.397874in}{2.317262in}}%
\pgfpathlineto{\pgfqpoint{1.518976in}{2.217647in}}%
\pgfpathlineto{\pgfqpoint{1.640077in}{2.115378in}}%
\pgfpathlineto{\pgfqpoint{1.781362in}{1.993025in}}%
\pgfpathlineto{\pgfqpoint{1.922647in}{1.867790in}}%
\pgfpathlineto{\pgfqpoint{2.084116in}{1.721734in}}%
\pgfpathlineto{\pgfqpoint{2.285951in}{1.535966in}}%
\pgfpathlineto{\pgfqpoint{2.770357in}{1.088109in}}%
\pgfpathlineto{\pgfqpoint{2.891458in}{0.979443in}}%
\pgfpathlineto{\pgfqpoint{2.992376in}{0.891330in}}%
\pgfpathlineto{\pgfqpoint{3.093294in}{0.806262in}}%
\pgfpathlineto{\pgfqpoint{3.174028in}{0.741002in}}%
\pgfpathlineto{\pgfqpoint{3.254762in}{0.678783in}}%
\pgfpathlineto{\pgfqpoint{3.315313in}{0.634446in}}%
\pgfpathlineto{\pgfqpoint{3.375864in}{0.592369in}}%
\pgfpathlineto{\pgfqpoint{3.436414in}{0.552784in}}%
\pgfpathlineto{\pgfqpoint{3.496965in}{0.515909in}}%
\pgfpathlineto{\pgfqpoint{3.557516in}{0.481935in}}%
\pgfpathlineto{\pgfqpoint{3.618066in}{0.451012in}}%
\pgfpathlineto{\pgfqpoint{3.678617in}{0.423237in}}%
\pgfpathlineto{\pgfqpoint{3.739168in}{0.451012in}}%
\pgfpathlineto{\pgfqpoint{3.799718in}{0.481935in}}%
\pgfpathlineto{\pgfqpoint{3.860269in}{0.515909in}}%
\pgfpathlineto{\pgfqpoint{3.920820in}{0.552784in}}%
\pgfpathlineto{\pgfqpoint{3.981370in}{0.592369in}}%
\pgfpathlineto{\pgfqpoint{4.041921in}{0.634446in}}%
\pgfpathlineto{\pgfqpoint{4.102472in}{0.678783in}}%
\pgfpathlineto{\pgfqpoint{4.183206in}{0.741002in}}%
\pgfpathlineto{\pgfqpoint{4.263940in}{0.806262in}}%
\pgfpathlineto{\pgfqpoint{4.364858in}{0.891330in}}%
\pgfpathlineto{\pgfqpoint{4.465776in}{0.979443in}}%
\pgfpathlineto{\pgfqpoint{4.485960in}{0.997360in}}%
\pgfpathlineto{\pgfqpoint{4.485960in}{0.997360in}}%
\pgfusepath{stroke}%
\end{pgfscope}%
\begin{pgfscope}%
\pgfpathrectangle{\pgfqpoint{0.651083in}{0.321913in}}{\pgfqpoint{4.036712in}{2.575304in}} %
\pgfusepath{clip}%
\pgfsetrectcap%
\pgfsetroundjoin%
\pgfsetlinewidth{1.003750pt}%
\definecolor{currentstroke}{rgb}{0.000000,0.500000,0.000000}%
\pgfsetstrokecolor{currentstroke}%
\pgfsetdash{}{0pt}%
\pgfpathmoveto{\pgfqpoint{0.852918in}{1.385067in}}%
\pgfpathlineto{\pgfqpoint{0.974020in}{1.384549in}}%
\pgfpathlineto{\pgfqpoint{1.095121in}{1.382052in}}%
\pgfpathlineto{\pgfqpoint{1.216222in}{1.377586in}}%
\pgfpathlineto{\pgfqpoint{1.337324in}{1.371167in}}%
\pgfpathlineto{\pgfqpoint{1.458425in}{1.362813in}}%
\pgfpathlineto{\pgfqpoint{1.599710in}{1.350652in}}%
\pgfpathlineto{\pgfqpoint{1.740995in}{1.335932in}}%
\pgfpathlineto{\pgfqpoint{1.882280in}{1.318706in}}%
\pgfpathlineto{\pgfqpoint{2.023565in}{1.299036in}}%
\pgfpathlineto{\pgfqpoint{2.164850in}{1.276996in}}%
\pgfpathlineto{\pgfqpoint{2.306135in}{1.252669in}}%
\pgfpathlineto{\pgfqpoint{2.447420in}{1.226154in}}%
\pgfpathlineto{\pgfqpoint{2.588705in}{1.197565in}}%
\pgfpathlineto{\pgfqpoint{2.750173in}{1.162524in}}%
\pgfpathlineto{\pgfqpoint{2.911642in}{1.125187in}}%
\pgfpathlineto{\pgfqpoint{3.093294in}{1.080798in}}%
\pgfpathlineto{\pgfqpoint{3.295129in}{1.029135in}}%
\pgfpathlineto{\pgfqpoint{3.638250in}{0.938459in}}%
\pgfpathlineto{\pgfqpoint{3.860269in}{0.880872in}}%
\pgfpathlineto{\pgfqpoint{4.001554in}{0.846373in}}%
\pgfpathlineto{\pgfqpoint{4.122655in}{0.819176in}}%
\pgfpathlineto{\pgfqpoint{4.223573in}{0.798883in}}%
\pgfpathlineto{\pgfqpoint{4.304307in}{0.784637in}}%
\pgfpathlineto{\pgfqpoint{4.385042in}{0.772519in}}%
\pgfpathlineto{\pgfqpoint{4.465776in}{0.762855in}}%
\pgfpathlineto{\pgfqpoint{4.485960in}{0.760859in}}%
\pgfpathlineto{\pgfqpoint{4.485960in}{0.760859in}}%
\pgfusepath{stroke}%
\end{pgfscope}%
\begin{pgfscope}%
\pgfpathrectangle{\pgfqpoint{0.651083in}{0.321913in}}{\pgfqpoint{4.036712in}{2.575304in}} %
\pgfusepath{clip}%
\pgfsetrectcap%
\pgfsetroundjoin%
\pgfsetlinewidth{1.003750pt}%
\definecolor{currentstroke}{rgb}{1.000000,0.000000,0.000000}%
\pgfsetstrokecolor{currentstroke}%
\pgfsetdash{}{0pt}%
\pgfpathmoveto{\pgfqpoint{0.852918in}{0.760859in}}%
\pgfpathlineto{\pgfqpoint{0.933653in}{0.769860in}}%
\pgfpathlineto{\pgfqpoint{1.014387in}{0.781395in}}%
\pgfpathlineto{\pgfqpoint{1.095121in}{0.795140in}}%
\pgfpathlineto{\pgfqpoint{1.196039in}{0.814923in}}%
\pgfpathlineto{\pgfqpoint{1.296957in}{0.837017in}}%
\pgfpathlineto{\pgfqpoint{1.418058in}{0.865797in}}%
\pgfpathlineto{\pgfqpoint{1.579527in}{0.906706in}}%
\pgfpathlineto{\pgfqpoint{1.862096in}{0.981286in}}%
\pgfpathlineto{\pgfqpoint{2.144666in}{1.055226in}}%
\pgfpathlineto{\pgfqpoint{2.346502in}{1.105744in}}%
\pgfpathlineto{\pgfqpoint{2.528154in}{1.148778in}}%
\pgfpathlineto{\pgfqpoint{2.689622in}{1.184710in}}%
\pgfpathlineto{\pgfqpoint{2.851091in}{1.218192in}}%
\pgfpathlineto{\pgfqpoint{2.992376in}{1.245312in}}%
\pgfpathlineto{\pgfqpoint{3.133661in}{1.270274in}}%
\pgfpathlineto{\pgfqpoint{3.274946in}{1.292978in}}%
\pgfpathlineto{\pgfqpoint{3.416231in}{1.313333in}}%
\pgfpathlineto{\pgfqpoint{3.557516in}{1.331264in}}%
\pgfpathlineto{\pgfqpoint{3.698801in}{1.346705in}}%
\pgfpathlineto{\pgfqpoint{3.840086in}{1.359602in}}%
\pgfpathlineto{\pgfqpoint{3.961187in}{1.368597in}}%
\pgfpathlineto{\pgfqpoint{4.082288in}{1.375663in}}%
\pgfpathlineto{\pgfqpoint{4.203390in}{1.380781in}}%
\pgfpathlineto{\pgfqpoint{4.324491in}{1.383936in}}%
\pgfpathlineto{\pgfqpoint{4.445592in}{1.385115in}}%
\pgfpathlineto{\pgfqpoint{4.485960in}{1.385067in}}%
\pgfpathlineto{\pgfqpoint{4.485960in}{1.385067in}}%
\pgfusepath{stroke}%
\end{pgfscope}%
\begin{pgfscope}%
\pgfpathrectangle{\pgfqpoint{0.651083in}{0.321913in}}{\pgfqpoint{4.036712in}{2.575304in}} %
\pgfusepath{clip}%
\pgfsetrectcap%
\pgfsetroundjoin%
\pgfsetlinewidth{1.003750pt}%
\definecolor{currentstroke}{rgb}{0.000000,0.750000,0.750000}%
\pgfsetstrokecolor{currentstroke}%
\pgfsetdash{}{0pt}%
\pgfpathmoveto{\pgfqpoint{0.852918in}{0.997360in}}%
\pgfpathlineto{\pgfqpoint{0.974020in}{0.891330in}}%
\pgfpathlineto{\pgfqpoint{1.074937in}{0.806262in}}%
\pgfpathlineto{\pgfqpoint{1.155672in}{0.741002in}}%
\pgfpathlineto{\pgfqpoint{1.236406in}{0.678783in}}%
\pgfpathlineto{\pgfqpoint{1.296957in}{0.634446in}}%
\pgfpathlineto{\pgfqpoint{1.357507in}{0.592369in}}%
\pgfpathlineto{\pgfqpoint{1.418058in}{0.552784in}}%
\pgfpathlineto{\pgfqpoint{1.478609in}{0.515909in}}%
\pgfpathlineto{\pgfqpoint{1.539159in}{0.481935in}}%
\pgfpathlineto{\pgfqpoint{1.599710in}{0.451012in}}%
\pgfpathlineto{\pgfqpoint{1.660261in}{0.423237in}}%
\pgfpathlineto{\pgfqpoint{1.720811in}{0.451012in}}%
\pgfpathlineto{\pgfqpoint{1.781362in}{0.481935in}}%
\pgfpathlineto{\pgfqpoint{1.841913in}{0.515909in}}%
\pgfpathlineto{\pgfqpoint{1.902464in}{0.552784in}}%
\pgfpathlineto{\pgfqpoint{1.963014in}{0.592369in}}%
\pgfpathlineto{\pgfqpoint{2.023565in}{0.634446in}}%
\pgfpathlineto{\pgfqpoint{2.084116in}{0.678783in}}%
\pgfpathlineto{\pgfqpoint{2.164850in}{0.741002in}}%
\pgfpathlineto{\pgfqpoint{2.245584in}{0.806262in}}%
\pgfpathlineto{\pgfqpoint{2.346502in}{0.891330in}}%
\pgfpathlineto{\pgfqpoint{2.447420in}{0.979443in}}%
\pgfpathlineto{\pgfqpoint{2.568521in}{1.088109in}}%
\pgfpathlineto{\pgfqpoint{2.729990in}{1.236179in}}%
\pgfpathlineto{\pgfqpoint{3.335496in}{1.795109in}}%
\pgfpathlineto{\pgfqpoint{3.496965in}{1.939680in}}%
\pgfpathlineto{\pgfqpoint{3.638250in}{2.063321in}}%
\pgfpathlineto{\pgfqpoint{3.779535in}{2.183841in}}%
\pgfpathlineto{\pgfqpoint{3.900636in}{2.284366in}}%
\pgfpathlineto{\pgfqpoint{4.021738in}{2.382092in}}%
\pgfpathlineto{\pgfqpoint{4.142839in}{2.476816in}}%
\pgfpathlineto{\pgfqpoint{4.263940in}{2.568348in}}%
\pgfpathlineto{\pgfqpoint{4.364858in}{2.642060in}}%
\pgfpathlineto{\pgfqpoint{4.465776in}{2.713338in}}%
\pgfpathlineto{\pgfqpoint{4.485960in}{2.727293in}}%
\pgfpathlineto{\pgfqpoint{4.485960in}{2.727293in}}%
\pgfusepath{stroke}%
\end{pgfscope}%
\begin{pgfscope}%
\pgfsetbuttcap%
\pgfsetroundjoin%
\definecolor{currentfill}{rgb}{0.000000,0.000000,0.000000}%
\pgfsetfillcolor{currentfill}%
\pgfsetlinewidth{0.501875pt}%
\definecolor{currentstroke}{rgb}{0.000000,0.000000,0.000000}%
\pgfsetstrokecolor{currentstroke}%
\pgfsetdash{}{0pt}%
\pgfsys@defobject{currentmarker}{\pgfqpoint{0.000000in}{0.000000in}}{\pgfqpoint{0.000000in}{0.055556in}}{%
\pgfpathmoveto{\pgfqpoint{0.000000in}{0.000000in}}%
\pgfpathlineto{\pgfqpoint{0.000000in}{0.055556in}}%
\pgfusepath{stroke,fill}%
}%
\begin{pgfscope}%
\pgfsys@transformshift{0.651083in}{0.321913in}%
\pgfsys@useobject{currentmarker}{}%
\end{pgfscope}%
\end{pgfscope}%
\begin{pgfscope}%
\pgfsetbuttcap%
\pgfsetroundjoin%
\definecolor{currentfill}{rgb}{0.000000,0.000000,0.000000}%
\pgfsetfillcolor{currentfill}%
\pgfsetlinewidth{0.501875pt}%
\definecolor{currentstroke}{rgb}{0.000000,0.000000,0.000000}%
\pgfsetstrokecolor{currentstroke}%
\pgfsetdash{}{0pt}%
\pgfsys@defobject{currentmarker}{\pgfqpoint{0.000000in}{-0.055556in}}{\pgfqpoint{0.000000in}{0.000000in}}{%
\pgfpathmoveto{\pgfqpoint{0.000000in}{0.000000in}}%
\pgfpathlineto{\pgfqpoint{0.000000in}{-0.055556in}}%
\pgfusepath{stroke,fill}%
}%
\begin{pgfscope}%
\pgfsys@transformshift{0.651083in}{2.897217in}%
\pgfsys@useobject{currentmarker}{}%
\end{pgfscope}%
\end{pgfscope}%
\begin{pgfscope}%
\pgftext[x=0.651083in,y=0.266357in,,top]{{\rmfamily\fontsize{8.000000}{9.600000}\selectfont \(\displaystyle -1.0\)}}%
\end{pgfscope}%
\begin{pgfscope}%
\pgfsetbuttcap%
\pgfsetroundjoin%
\definecolor{currentfill}{rgb}{0.000000,0.000000,0.000000}%
\pgfsetfillcolor{currentfill}%
\pgfsetlinewidth{0.501875pt}%
\definecolor{currentstroke}{rgb}{0.000000,0.000000,0.000000}%
\pgfsetstrokecolor{currentstroke}%
\pgfsetdash{}{0pt}%
\pgfsys@defobject{currentmarker}{\pgfqpoint{0.000000in}{0.000000in}}{\pgfqpoint{0.000000in}{0.055556in}}{%
\pgfpathmoveto{\pgfqpoint{0.000000in}{0.000000in}}%
\pgfpathlineto{\pgfqpoint{0.000000in}{0.055556in}}%
\pgfusepath{stroke,fill}%
}%
\begin{pgfscope}%
\pgfsys@transformshift{1.660261in}{0.321913in}%
\pgfsys@useobject{currentmarker}{}%
\end{pgfscope}%
\end{pgfscope}%
\begin{pgfscope}%
\pgfsetbuttcap%
\pgfsetroundjoin%
\definecolor{currentfill}{rgb}{0.000000,0.000000,0.000000}%
\pgfsetfillcolor{currentfill}%
\pgfsetlinewidth{0.501875pt}%
\definecolor{currentstroke}{rgb}{0.000000,0.000000,0.000000}%
\pgfsetstrokecolor{currentstroke}%
\pgfsetdash{}{0pt}%
\pgfsys@defobject{currentmarker}{\pgfqpoint{0.000000in}{-0.055556in}}{\pgfqpoint{0.000000in}{0.000000in}}{%
\pgfpathmoveto{\pgfqpoint{0.000000in}{0.000000in}}%
\pgfpathlineto{\pgfqpoint{0.000000in}{-0.055556in}}%
\pgfusepath{stroke,fill}%
}%
\begin{pgfscope}%
\pgfsys@transformshift{1.660261in}{2.897217in}%
\pgfsys@useobject{currentmarker}{}%
\end{pgfscope}%
\end{pgfscope}%
\begin{pgfscope}%
\pgftext[x=1.660261in,y=0.266357in,,top]{{\rmfamily\fontsize{8.000000}{9.600000}\selectfont \(\displaystyle -0.5\)}}%
\end{pgfscope}%
\begin{pgfscope}%
\pgfsetbuttcap%
\pgfsetroundjoin%
\definecolor{currentfill}{rgb}{0.000000,0.000000,0.000000}%
\pgfsetfillcolor{currentfill}%
\pgfsetlinewidth{0.501875pt}%
\definecolor{currentstroke}{rgb}{0.000000,0.000000,0.000000}%
\pgfsetstrokecolor{currentstroke}%
\pgfsetdash{}{0pt}%
\pgfsys@defobject{currentmarker}{\pgfqpoint{0.000000in}{0.000000in}}{\pgfqpoint{0.000000in}{0.055556in}}{%
\pgfpathmoveto{\pgfqpoint{0.000000in}{0.000000in}}%
\pgfpathlineto{\pgfqpoint{0.000000in}{0.055556in}}%
\pgfusepath{stroke,fill}%
}%
\begin{pgfscope}%
\pgfsys@transformshift{2.669439in}{0.321913in}%
\pgfsys@useobject{currentmarker}{}%
\end{pgfscope}%
\end{pgfscope}%
\begin{pgfscope}%
\pgfsetbuttcap%
\pgfsetroundjoin%
\definecolor{currentfill}{rgb}{0.000000,0.000000,0.000000}%
\pgfsetfillcolor{currentfill}%
\pgfsetlinewidth{0.501875pt}%
\definecolor{currentstroke}{rgb}{0.000000,0.000000,0.000000}%
\pgfsetstrokecolor{currentstroke}%
\pgfsetdash{}{0pt}%
\pgfsys@defobject{currentmarker}{\pgfqpoint{0.000000in}{-0.055556in}}{\pgfqpoint{0.000000in}{0.000000in}}{%
\pgfpathmoveto{\pgfqpoint{0.000000in}{0.000000in}}%
\pgfpathlineto{\pgfqpoint{0.000000in}{-0.055556in}}%
\pgfusepath{stroke,fill}%
}%
\begin{pgfscope}%
\pgfsys@transformshift{2.669439in}{2.897217in}%
\pgfsys@useobject{currentmarker}{}%
\end{pgfscope}%
\end{pgfscope}%
\begin{pgfscope}%
\pgftext[x=2.669439in,y=0.266357in,,top]{{\rmfamily\fontsize{8.000000}{9.600000}\selectfont \(\displaystyle 0.0\)}}%
\end{pgfscope}%
\begin{pgfscope}%
\pgfsetbuttcap%
\pgfsetroundjoin%
\definecolor{currentfill}{rgb}{0.000000,0.000000,0.000000}%
\pgfsetfillcolor{currentfill}%
\pgfsetlinewidth{0.501875pt}%
\definecolor{currentstroke}{rgb}{0.000000,0.000000,0.000000}%
\pgfsetstrokecolor{currentstroke}%
\pgfsetdash{}{0pt}%
\pgfsys@defobject{currentmarker}{\pgfqpoint{0.000000in}{0.000000in}}{\pgfqpoint{0.000000in}{0.055556in}}{%
\pgfpathmoveto{\pgfqpoint{0.000000in}{0.000000in}}%
\pgfpathlineto{\pgfqpoint{0.000000in}{0.055556in}}%
\pgfusepath{stroke,fill}%
}%
\begin{pgfscope}%
\pgfsys@transformshift{3.678617in}{0.321913in}%
\pgfsys@useobject{currentmarker}{}%
\end{pgfscope}%
\end{pgfscope}%
\begin{pgfscope}%
\pgfsetbuttcap%
\pgfsetroundjoin%
\definecolor{currentfill}{rgb}{0.000000,0.000000,0.000000}%
\pgfsetfillcolor{currentfill}%
\pgfsetlinewidth{0.501875pt}%
\definecolor{currentstroke}{rgb}{0.000000,0.000000,0.000000}%
\pgfsetstrokecolor{currentstroke}%
\pgfsetdash{}{0pt}%
\pgfsys@defobject{currentmarker}{\pgfqpoint{0.000000in}{-0.055556in}}{\pgfqpoint{0.000000in}{0.000000in}}{%
\pgfpathmoveto{\pgfqpoint{0.000000in}{0.000000in}}%
\pgfpathlineto{\pgfqpoint{0.000000in}{-0.055556in}}%
\pgfusepath{stroke,fill}%
}%
\begin{pgfscope}%
\pgfsys@transformshift{3.678617in}{2.897217in}%
\pgfsys@useobject{currentmarker}{}%
\end{pgfscope}%
\end{pgfscope}%
\begin{pgfscope}%
\pgftext[x=3.678617in,y=0.266357in,,top]{{\rmfamily\fontsize{8.000000}{9.600000}\selectfont \(\displaystyle 0.5\)}}%
\end{pgfscope}%
\begin{pgfscope}%
\pgfsetbuttcap%
\pgfsetroundjoin%
\definecolor{currentfill}{rgb}{0.000000,0.000000,0.000000}%
\pgfsetfillcolor{currentfill}%
\pgfsetlinewidth{0.501875pt}%
\definecolor{currentstroke}{rgb}{0.000000,0.000000,0.000000}%
\pgfsetstrokecolor{currentstroke}%
\pgfsetdash{}{0pt}%
\pgfsys@defobject{currentmarker}{\pgfqpoint{0.000000in}{0.000000in}}{\pgfqpoint{0.000000in}{0.055556in}}{%
\pgfpathmoveto{\pgfqpoint{0.000000in}{0.000000in}}%
\pgfpathlineto{\pgfqpoint{0.000000in}{0.055556in}}%
\pgfusepath{stroke,fill}%
}%
\begin{pgfscope}%
\pgfsys@transformshift{4.687795in}{0.321913in}%
\pgfsys@useobject{currentmarker}{}%
\end{pgfscope}%
\end{pgfscope}%
\begin{pgfscope}%
\pgfsetbuttcap%
\pgfsetroundjoin%
\definecolor{currentfill}{rgb}{0.000000,0.000000,0.000000}%
\pgfsetfillcolor{currentfill}%
\pgfsetlinewidth{0.501875pt}%
\definecolor{currentstroke}{rgb}{0.000000,0.000000,0.000000}%
\pgfsetstrokecolor{currentstroke}%
\pgfsetdash{}{0pt}%
\pgfsys@defobject{currentmarker}{\pgfqpoint{0.000000in}{-0.055556in}}{\pgfqpoint{0.000000in}{0.000000in}}{%
\pgfpathmoveto{\pgfqpoint{0.000000in}{0.000000in}}%
\pgfpathlineto{\pgfqpoint{0.000000in}{-0.055556in}}%
\pgfusepath{stroke,fill}%
}%
\begin{pgfscope}%
\pgfsys@transformshift{4.687795in}{2.897217in}%
\pgfsys@useobject{currentmarker}{}%
\end{pgfscope}%
\end{pgfscope}%
\begin{pgfscope}%
\pgftext[x=4.687795in,y=0.266357in,,top]{{\rmfamily\fontsize{8.000000}{9.600000}\selectfont \(\displaystyle 1.0\)}}%
\end{pgfscope}%
\begin{pgfscope}%
\pgftext[x=2.669439in,y=0.098789in,,top]{{\rmfamily\fontsize{10.000000}{12.000000}\selectfont \(\displaystyle  \delta \)}}%
\end{pgfscope}%
\begin{pgfscope}%
\pgfsetbuttcap%
\pgfsetroundjoin%
\definecolor{currentfill}{rgb}{0.000000,0.000000,0.000000}%
\pgfsetfillcolor{currentfill}%
\pgfsetlinewidth{0.501875pt}%
\definecolor{currentstroke}{rgb}{0.000000,0.000000,0.000000}%
\pgfsetstrokecolor{currentstroke}%
\pgfsetdash{}{0pt}%
\pgfsys@defobject{currentmarker}{\pgfqpoint{0.000000in}{0.000000in}}{\pgfqpoint{0.055556in}{0.000000in}}{%
\pgfpathmoveto{\pgfqpoint{0.000000in}{0.000000in}}%
\pgfpathlineto{\pgfqpoint{0.055556in}{0.000000in}}%
\pgfusepath{stroke,fill}%
}%
\begin{pgfscope}%
\pgfsys@transformshift{0.651083in}{0.321913in}%
\pgfsys@useobject{currentmarker}{}%
\end{pgfscope}%
\end{pgfscope}%
\begin{pgfscope}%
\pgfsetbuttcap%
\pgfsetroundjoin%
\definecolor{currentfill}{rgb}{0.000000,0.000000,0.000000}%
\pgfsetfillcolor{currentfill}%
\pgfsetlinewidth{0.501875pt}%
\definecolor{currentstroke}{rgb}{0.000000,0.000000,0.000000}%
\pgfsetstrokecolor{currentstroke}%
\pgfsetdash{}{0pt}%
\pgfsys@defobject{currentmarker}{\pgfqpoint{-0.055556in}{0.000000in}}{\pgfqpoint{0.000000in}{0.000000in}}{%
\pgfpathmoveto{\pgfqpoint{0.000000in}{0.000000in}}%
\pgfpathlineto{\pgfqpoint{-0.055556in}{0.000000in}}%
\pgfusepath{stroke,fill}%
}%
\begin{pgfscope}%
\pgfsys@transformshift{4.687795in}{0.321913in}%
\pgfsys@useobject{currentmarker}{}%
\end{pgfscope}%
\end{pgfscope}%
\begin{pgfscope}%
\pgftext[x=0.595527in,y=0.321913in,right,]{{\rmfamily\fontsize{8.000000}{9.600000}\selectfont \(\displaystyle 3\)}}%
\end{pgfscope}%
\begin{pgfscope}%
\pgfsetbuttcap%
\pgfsetroundjoin%
\definecolor{currentfill}{rgb}{0.000000,0.000000,0.000000}%
\pgfsetfillcolor{currentfill}%
\pgfsetlinewidth{0.501875pt}%
\definecolor{currentstroke}{rgb}{0.000000,0.000000,0.000000}%
\pgfsetstrokecolor{currentstroke}%
\pgfsetdash{}{0pt}%
\pgfsys@defobject{currentmarker}{\pgfqpoint{0.000000in}{0.000000in}}{\pgfqpoint{0.055556in}{0.000000in}}{%
\pgfpathmoveto{\pgfqpoint{0.000000in}{0.000000in}}%
\pgfpathlineto{\pgfqpoint{0.055556in}{0.000000in}}%
\pgfusepath{stroke,fill}%
}%
\begin{pgfscope}%
\pgfsys@transformshift{0.651083in}{0.751130in}%
\pgfsys@useobject{currentmarker}{}%
\end{pgfscope}%
\end{pgfscope}%
\begin{pgfscope}%
\pgfsetbuttcap%
\pgfsetroundjoin%
\definecolor{currentfill}{rgb}{0.000000,0.000000,0.000000}%
\pgfsetfillcolor{currentfill}%
\pgfsetlinewidth{0.501875pt}%
\definecolor{currentstroke}{rgb}{0.000000,0.000000,0.000000}%
\pgfsetstrokecolor{currentstroke}%
\pgfsetdash{}{0pt}%
\pgfsys@defobject{currentmarker}{\pgfqpoint{-0.055556in}{0.000000in}}{\pgfqpoint{0.000000in}{0.000000in}}{%
\pgfpathmoveto{\pgfqpoint{0.000000in}{0.000000in}}%
\pgfpathlineto{\pgfqpoint{-0.055556in}{0.000000in}}%
\pgfusepath{stroke,fill}%
}%
\begin{pgfscope}%
\pgfsys@transformshift{4.687795in}{0.751130in}%
\pgfsys@useobject{currentmarker}{}%
\end{pgfscope}%
\end{pgfscope}%
\begin{pgfscope}%
\pgftext[x=0.595527in,y=0.751130in,right,]{{\rmfamily\fontsize{8.000000}{9.600000}\selectfont \(\displaystyle 4\)}}%
\end{pgfscope}%
\begin{pgfscope}%
\pgfsetbuttcap%
\pgfsetroundjoin%
\definecolor{currentfill}{rgb}{0.000000,0.000000,0.000000}%
\pgfsetfillcolor{currentfill}%
\pgfsetlinewidth{0.501875pt}%
\definecolor{currentstroke}{rgb}{0.000000,0.000000,0.000000}%
\pgfsetstrokecolor{currentstroke}%
\pgfsetdash{}{0pt}%
\pgfsys@defobject{currentmarker}{\pgfqpoint{0.000000in}{0.000000in}}{\pgfqpoint{0.055556in}{0.000000in}}{%
\pgfpathmoveto{\pgfqpoint{0.000000in}{0.000000in}}%
\pgfpathlineto{\pgfqpoint{0.055556in}{0.000000in}}%
\pgfusepath{stroke,fill}%
}%
\begin{pgfscope}%
\pgfsys@transformshift{0.651083in}{1.180348in}%
\pgfsys@useobject{currentmarker}{}%
\end{pgfscope}%
\end{pgfscope}%
\begin{pgfscope}%
\pgfsetbuttcap%
\pgfsetroundjoin%
\definecolor{currentfill}{rgb}{0.000000,0.000000,0.000000}%
\pgfsetfillcolor{currentfill}%
\pgfsetlinewidth{0.501875pt}%
\definecolor{currentstroke}{rgb}{0.000000,0.000000,0.000000}%
\pgfsetstrokecolor{currentstroke}%
\pgfsetdash{}{0pt}%
\pgfsys@defobject{currentmarker}{\pgfqpoint{-0.055556in}{0.000000in}}{\pgfqpoint{0.000000in}{0.000000in}}{%
\pgfpathmoveto{\pgfqpoint{0.000000in}{0.000000in}}%
\pgfpathlineto{\pgfqpoint{-0.055556in}{0.000000in}}%
\pgfusepath{stroke,fill}%
}%
\begin{pgfscope}%
\pgfsys@transformshift{4.687795in}{1.180348in}%
\pgfsys@useobject{currentmarker}{}%
\end{pgfscope}%
\end{pgfscope}%
\begin{pgfscope}%
\pgftext[x=0.595527in,y=1.180348in,right,]{{\rmfamily\fontsize{8.000000}{9.600000}\selectfont \(\displaystyle 5\)}}%
\end{pgfscope}%
\begin{pgfscope}%
\pgfsetbuttcap%
\pgfsetroundjoin%
\definecolor{currentfill}{rgb}{0.000000,0.000000,0.000000}%
\pgfsetfillcolor{currentfill}%
\pgfsetlinewidth{0.501875pt}%
\definecolor{currentstroke}{rgb}{0.000000,0.000000,0.000000}%
\pgfsetstrokecolor{currentstroke}%
\pgfsetdash{}{0pt}%
\pgfsys@defobject{currentmarker}{\pgfqpoint{0.000000in}{0.000000in}}{\pgfqpoint{0.055556in}{0.000000in}}{%
\pgfpathmoveto{\pgfqpoint{0.000000in}{0.000000in}}%
\pgfpathlineto{\pgfqpoint{0.055556in}{0.000000in}}%
\pgfusepath{stroke,fill}%
}%
\begin{pgfscope}%
\pgfsys@transformshift{0.651083in}{1.609565in}%
\pgfsys@useobject{currentmarker}{}%
\end{pgfscope}%
\end{pgfscope}%
\begin{pgfscope}%
\pgfsetbuttcap%
\pgfsetroundjoin%
\definecolor{currentfill}{rgb}{0.000000,0.000000,0.000000}%
\pgfsetfillcolor{currentfill}%
\pgfsetlinewidth{0.501875pt}%
\definecolor{currentstroke}{rgb}{0.000000,0.000000,0.000000}%
\pgfsetstrokecolor{currentstroke}%
\pgfsetdash{}{0pt}%
\pgfsys@defobject{currentmarker}{\pgfqpoint{-0.055556in}{0.000000in}}{\pgfqpoint{0.000000in}{0.000000in}}{%
\pgfpathmoveto{\pgfqpoint{0.000000in}{0.000000in}}%
\pgfpathlineto{\pgfqpoint{-0.055556in}{0.000000in}}%
\pgfusepath{stroke,fill}%
}%
\begin{pgfscope}%
\pgfsys@transformshift{4.687795in}{1.609565in}%
\pgfsys@useobject{currentmarker}{}%
\end{pgfscope}%
\end{pgfscope}%
\begin{pgfscope}%
\pgftext[x=0.595527in,y=1.609565in,right,]{{\rmfamily\fontsize{8.000000}{9.600000}\selectfont \(\displaystyle 6\)}}%
\end{pgfscope}%
\begin{pgfscope}%
\pgfsetbuttcap%
\pgfsetroundjoin%
\definecolor{currentfill}{rgb}{0.000000,0.000000,0.000000}%
\pgfsetfillcolor{currentfill}%
\pgfsetlinewidth{0.501875pt}%
\definecolor{currentstroke}{rgb}{0.000000,0.000000,0.000000}%
\pgfsetstrokecolor{currentstroke}%
\pgfsetdash{}{0pt}%
\pgfsys@defobject{currentmarker}{\pgfqpoint{0.000000in}{0.000000in}}{\pgfqpoint{0.055556in}{0.000000in}}{%
\pgfpathmoveto{\pgfqpoint{0.000000in}{0.000000in}}%
\pgfpathlineto{\pgfqpoint{0.055556in}{0.000000in}}%
\pgfusepath{stroke,fill}%
}%
\begin{pgfscope}%
\pgfsys@transformshift{0.651083in}{2.038782in}%
\pgfsys@useobject{currentmarker}{}%
\end{pgfscope}%
\end{pgfscope}%
\begin{pgfscope}%
\pgfsetbuttcap%
\pgfsetroundjoin%
\definecolor{currentfill}{rgb}{0.000000,0.000000,0.000000}%
\pgfsetfillcolor{currentfill}%
\pgfsetlinewidth{0.501875pt}%
\definecolor{currentstroke}{rgb}{0.000000,0.000000,0.000000}%
\pgfsetstrokecolor{currentstroke}%
\pgfsetdash{}{0pt}%
\pgfsys@defobject{currentmarker}{\pgfqpoint{-0.055556in}{0.000000in}}{\pgfqpoint{0.000000in}{0.000000in}}{%
\pgfpathmoveto{\pgfqpoint{0.000000in}{0.000000in}}%
\pgfpathlineto{\pgfqpoint{-0.055556in}{0.000000in}}%
\pgfusepath{stroke,fill}%
}%
\begin{pgfscope}%
\pgfsys@transformshift{4.687795in}{2.038782in}%
\pgfsys@useobject{currentmarker}{}%
\end{pgfscope}%
\end{pgfscope}%
\begin{pgfscope}%
\pgftext[x=0.595527in,y=2.038782in,right,]{{\rmfamily\fontsize{8.000000}{9.600000}\selectfont \(\displaystyle 7\)}}%
\end{pgfscope}%
\begin{pgfscope}%
\pgfsetbuttcap%
\pgfsetroundjoin%
\definecolor{currentfill}{rgb}{0.000000,0.000000,0.000000}%
\pgfsetfillcolor{currentfill}%
\pgfsetlinewidth{0.501875pt}%
\definecolor{currentstroke}{rgb}{0.000000,0.000000,0.000000}%
\pgfsetstrokecolor{currentstroke}%
\pgfsetdash{}{0pt}%
\pgfsys@defobject{currentmarker}{\pgfqpoint{0.000000in}{0.000000in}}{\pgfqpoint{0.055556in}{0.000000in}}{%
\pgfpathmoveto{\pgfqpoint{0.000000in}{0.000000in}}%
\pgfpathlineto{\pgfqpoint{0.055556in}{0.000000in}}%
\pgfusepath{stroke,fill}%
}%
\begin{pgfscope}%
\pgfsys@transformshift{0.651083in}{2.467999in}%
\pgfsys@useobject{currentmarker}{}%
\end{pgfscope}%
\end{pgfscope}%
\begin{pgfscope}%
\pgfsetbuttcap%
\pgfsetroundjoin%
\definecolor{currentfill}{rgb}{0.000000,0.000000,0.000000}%
\pgfsetfillcolor{currentfill}%
\pgfsetlinewidth{0.501875pt}%
\definecolor{currentstroke}{rgb}{0.000000,0.000000,0.000000}%
\pgfsetstrokecolor{currentstroke}%
\pgfsetdash{}{0pt}%
\pgfsys@defobject{currentmarker}{\pgfqpoint{-0.055556in}{0.000000in}}{\pgfqpoint{0.000000in}{0.000000in}}{%
\pgfpathmoveto{\pgfqpoint{0.000000in}{0.000000in}}%
\pgfpathlineto{\pgfqpoint{-0.055556in}{0.000000in}}%
\pgfusepath{stroke,fill}%
}%
\begin{pgfscope}%
\pgfsys@transformshift{4.687795in}{2.467999in}%
\pgfsys@useobject{currentmarker}{}%
\end{pgfscope}%
\end{pgfscope}%
\begin{pgfscope}%
\pgftext[x=0.595527in,y=2.467999in,right,]{{\rmfamily\fontsize{8.000000}{9.600000}\selectfont \(\displaystyle 8\)}}%
\end{pgfscope}%
\begin{pgfscope}%
\pgfsetbuttcap%
\pgfsetroundjoin%
\definecolor{currentfill}{rgb}{0.000000,0.000000,0.000000}%
\pgfsetfillcolor{currentfill}%
\pgfsetlinewidth{0.501875pt}%
\definecolor{currentstroke}{rgb}{0.000000,0.000000,0.000000}%
\pgfsetstrokecolor{currentstroke}%
\pgfsetdash{}{0pt}%
\pgfsys@defobject{currentmarker}{\pgfqpoint{0.000000in}{0.000000in}}{\pgfqpoint{0.055556in}{0.000000in}}{%
\pgfpathmoveto{\pgfqpoint{0.000000in}{0.000000in}}%
\pgfpathlineto{\pgfqpoint{0.055556in}{0.000000in}}%
\pgfusepath{stroke,fill}%
}%
\begin{pgfscope}%
\pgfsys@transformshift{0.651083in}{2.897217in}%
\pgfsys@useobject{currentmarker}{}%
\end{pgfscope}%
\end{pgfscope}%
\begin{pgfscope}%
\pgfsetbuttcap%
\pgfsetroundjoin%
\definecolor{currentfill}{rgb}{0.000000,0.000000,0.000000}%
\pgfsetfillcolor{currentfill}%
\pgfsetlinewidth{0.501875pt}%
\definecolor{currentstroke}{rgb}{0.000000,0.000000,0.000000}%
\pgfsetstrokecolor{currentstroke}%
\pgfsetdash{}{0pt}%
\pgfsys@defobject{currentmarker}{\pgfqpoint{-0.055556in}{0.000000in}}{\pgfqpoint{0.000000in}{0.000000in}}{%
\pgfpathmoveto{\pgfqpoint{0.000000in}{0.000000in}}%
\pgfpathlineto{\pgfqpoint{-0.055556in}{0.000000in}}%
\pgfusepath{stroke,fill}%
}%
\begin{pgfscope}%
\pgfsys@transformshift{4.687795in}{2.897217in}%
\pgfsys@useobject{currentmarker}{}%
\end{pgfscope}%
\end{pgfscope}%
\begin{pgfscope}%
\pgftext[x=0.595527in,y=2.897217in,right,]{{\rmfamily\fontsize{8.000000}{9.600000}\selectfont \(\displaystyle 9\)}}%
\end{pgfscope}%
\begin{pgfscope}%
\pgftext[x=0.467054in,y=1.609565in,,bottom,rotate=90.000000]{{\rmfamily\fontsize{10.000000}{12.000000}\selectfont \(\displaystyle  g_j(\delta) = \sum_{r=0}^{n-1} | \sigma_r^j(z(\delta)) | \)}}%
\end{pgfscope}%
\begin{pgfscope}%
\pgfsetbuttcap%
\pgfsetroundjoin%
\pgfsetlinewidth{1.003750pt}%
\definecolor{currentstroke}{rgb}{0.000000,0.000000,0.000000}%
\pgfsetstrokecolor{currentstroke}%
\pgfsetdash{}{0pt}%
\pgfpathmoveto{\pgfqpoint{0.651083in}{2.897217in}}%
\pgfpathlineto{\pgfqpoint{4.687795in}{2.897217in}}%
\pgfusepath{stroke}%
\end{pgfscope}%
\begin{pgfscope}%
\pgfsetbuttcap%
\pgfsetroundjoin%
\pgfsetlinewidth{1.003750pt}%
\definecolor{currentstroke}{rgb}{0.000000,0.000000,0.000000}%
\pgfsetstrokecolor{currentstroke}%
\pgfsetdash{}{0pt}%
\pgfpathmoveto{\pgfqpoint{4.687795in}{0.321913in}}%
\pgfpathlineto{\pgfqpoint{4.687795in}{2.897217in}}%
\pgfusepath{stroke}%
\end{pgfscope}%
\begin{pgfscope}%
\pgfsetbuttcap%
\pgfsetroundjoin%
\pgfsetlinewidth{1.003750pt}%
\definecolor{currentstroke}{rgb}{0.000000,0.000000,0.000000}%
\pgfsetstrokecolor{currentstroke}%
\pgfsetdash{}{0pt}%
\pgfpathmoveto{\pgfqpoint{0.651083in}{0.321913in}}%
\pgfpathlineto{\pgfqpoint{4.687795in}{0.321913in}}%
\pgfusepath{stroke}%
\end{pgfscope}%
\begin{pgfscope}%
\pgfsetbuttcap%
\pgfsetroundjoin%
\pgfsetlinewidth{1.003750pt}%
\definecolor{currentstroke}{rgb}{0.000000,0.000000,0.000000}%
\pgfsetstrokecolor{currentstroke}%
\pgfsetdash{}{0pt}%
\pgfpathmoveto{\pgfqpoint{0.651083in}{0.321913in}}%
\pgfpathlineto{\pgfqpoint{0.651083in}{2.897217in}}%
\pgfusepath{stroke}%
\end{pgfscope}%
\begin{pgfscope}%
\pgfsetbuttcap%
\pgfsetroundjoin%
\definecolor{currentfill}{rgb}{0.300000,0.300000,0.300000}%
\pgfsetfillcolor{currentfill}%
\pgfsetfillopacity{0.500000}%
\pgfsetlinewidth{1.003750pt}%
\definecolor{currentstroke}{rgb}{0.300000,0.300000,0.300000}%
\pgfsetstrokecolor{currentstroke}%
\pgfsetstrokeopacity{0.500000}%
\pgfsetdash{}{0pt}%
\pgfpathmoveto{\pgfqpoint{1.705792in}{2.341919in}}%
\pgfpathlineto{\pgfqpoint{3.688641in}{2.341919in}}%
\pgfpathquadraticcurveto{\pgfqpoint{3.710864in}{2.341919in}}{\pgfqpoint{3.710864in}{2.364141in}}%
\pgfpathlineto{\pgfqpoint{3.710864in}{2.662896in}}%
\pgfpathquadraticcurveto{\pgfqpoint{3.710864in}{2.685118in}}{\pgfqpoint{3.688641in}{2.685118in}}%
\pgfpathlineto{\pgfqpoint{1.705792in}{2.685118in}}%
\pgfpathquadraticcurveto{\pgfqpoint{1.683570in}{2.685118in}}{\pgfqpoint{1.683570in}{2.662896in}}%
\pgfpathlineto{\pgfqpoint{1.683570in}{2.364141in}}%
\pgfpathquadraticcurveto{\pgfqpoint{1.683570in}{2.341919in}}{\pgfqpoint{1.705792in}{2.341919in}}%
\pgfpathclose%
\pgfusepath{stroke,fill}%
\end{pgfscope}%
\begin{pgfscope}%
\pgfsetbuttcap%
\pgfsetroundjoin%
\definecolor{currentfill}{rgb}{1.000000,1.000000,1.000000}%
\pgfsetfillcolor{currentfill}%
\pgfsetlinewidth{1.003750pt}%
\definecolor{currentstroke}{rgb}{0.000000,0.000000,0.000000}%
\pgfsetstrokecolor{currentstroke}%
\pgfsetdash{}{0pt}%
\pgfpathmoveto{\pgfqpoint{1.678014in}{2.369697in}}%
\pgfpathlineto{\pgfqpoint{3.660864in}{2.369697in}}%
\pgfpathquadraticcurveto{\pgfqpoint{3.683086in}{2.369697in}}{\pgfqpoint{3.683086in}{2.391919in}}%
\pgfpathlineto{\pgfqpoint{3.683086in}{2.690674in}}%
\pgfpathquadraticcurveto{\pgfqpoint{3.683086in}{2.712896in}}{\pgfqpoint{3.660864in}{2.712896in}}%
\pgfpathlineto{\pgfqpoint{1.678014in}{2.712896in}}%
\pgfpathquadraticcurveto{\pgfqpoint{1.655792in}{2.712896in}}{\pgfqpoint{1.655792in}{2.690674in}}%
\pgfpathlineto{\pgfqpoint{1.655792in}{2.391919in}}%
\pgfpathquadraticcurveto{\pgfqpoint{1.655792in}{2.369697in}}{\pgfqpoint{1.678014in}{2.369697in}}%
\pgfpathclose%
\pgfusepath{stroke,fill}%
\end{pgfscope}%
\begin{pgfscope}%
\pgfsetrectcap%
\pgfsetroundjoin%
\pgfsetlinewidth{1.003750pt}%
\definecolor{currentstroke}{rgb}{0.000000,0.000000,0.000000}%
\pgfsetstrokecolor{currentstroke}%
\pgfsetdash{}{0pt}%
\pgfpathmoveto{\pgfqpoint{1.733570in}{2.629563in}}%
\pgfpathlineto{\pgfqpoint{1.889125in}{2.629563in}}%
\pgfusepath{stroke}%
\end{pgfscope}%
\begin{pgfscope}%
\pgftext[x=2.011348in,y=2.590674in,left,base]{{\rmfamily\fontsize{8.000000}{9.600000}\selectfont j=0}}%
\end{pgfscope}%
\begin{pgfscope}%
\pgfsetrectcap%
\pgfsetroundjoin%
\pgfsetlinewidth{1.003750pt}%
\definecolor{currentstroke}{rgb}{0.000000,0.000000,1.000000}%
\pgfsetstrokecolor{currentstroke}%
\pgfsetdash{}{0pt}%
\pgfpathmoveto{\pgfqpoint{1.733570in}{2.474630in}}%
\pgfpathlineto{\pgfqpoint{1.889125in}{2.474630in}}%
\pgfusepath{stroke}%
\end{pgfscope}%
\begin{pgfscope}%
\pgftext[x=2.011348in,y=2.435741in,left,base]{{\rmfamily\fontsize{8.000000}{9.600000}\selectfont j=1}}%
\end{pgfscope}%
\begin{pgfscope}%
\pgfsetrectcap%
\pgfsetroundjoin%
\pgfsetlinewidth{1.003750pt}%
\definecolor{currentstroke}{rgb}{0.000000,0.500000,0.000000}%
\pgfsetstrokecolor{currentstroke}%
\pgfsetdash{}{0pt}%
\pgfpathmoveto{\pgfqpoint{2.453779in}{2.629563in}}%
\pgfpathlineto{\pgfqpoint{2.609334in}{2.629563in}}%
\pgfusepath{stroke}%
\end{pgfscope}%
\begin{pgfscope}%
\pgftext[x=2.731557in,y=2.590674in,left,base]{{\rmfamily\fontsize{8.000000}{9.600000}\selectfont j=2}}%
\end{pgfscope}%
\begin{pgfscope}%
\pgfsetrectcap%
\pgfsetroundjoin%
\pgfsetlinewidth{1.003750pt}%
\definecolor{currentstroke}{rgb}{1.000000,0.000000,0.000000}%
\pgfsetstrokecolor{currentstroke}%
\pgfsetdash{}{0pt}%
\pgfpathmoveto{\pgfqpoint{2.453779in}{2.474630in}}%
\pgfpathlineto{\pgfqpoint{2.609334in}{2.474630in}}%
\pgfusepath{stroke}%
\end{pgfscope}%
\begin{pgfscope}%
\pgftext[x=2.731557in,y=2.435741in,left,base]{{\rmfamily\fontsize{8.000000}{9.600000}\selectfont j=3}}%
\end{pgfscope}%
\begin{pgfscope}%
\pgfsetrectcap%
\pgfsetroundjoin%
\pgfsetlinewidth{1.003750pt}%
\definecolor{currentstroke}{rgb}{0.000000,0.750000,0.750000}%
\pgfsetstrokecolor{currentstroke}%
\pgfsetdash{}{0pt}%
\pgfpathmoveto{\pgfqpoint{3.173988in}{2.629563in}}%
\pgfpathlineto{\pgfqpoint{3.329543in}{2.629563in}}%
\pgfusepath{stroke}%
\end{pgfscope}%
\begin{pgfscope}%
\pgftext[x=3.451766in,y=2.590674in,left,base]{{\rmfamily\fontsize{8.000000}{9.600000}\selectfont j=4}}%
\end{pgfscope}%
\end{pgfpicture}%
\makeatother%
\endgroup%

        \caption{Zweiter Faktor
                der Betragssumme der $j$-ten Zeile
                von $\Vand{\delta}^{-1}$ in
                Abhängigkeit der Auslenkung $\delta \in (-0.9, 0.9)$
                für den Fall $n=5$.}
        \label{fig:sigma_row_sum}
    \end{figure}

    \begin{figure}[H]
        \centering
        %% Creator: Matplotlib, PGF backend
%%
%% To include the figure in your LaTeX document, write
%%   \input{<filename>.pgf}
%%
%% Make sure the required packages are loaded in your preamble
%%   \usepackage{pgf}
%%
%% Figures using additional raster images can only be included by \input if
%% they are in the same directory as the main LaTeX file. For loading figures
%% from other directories you can use the `import` package
%%   \usepackage{import}
%% and then include the figures with
%%   \import{<path to file>}{<filename>.pgf}
%%
%% Matplotlib used the following preamble
%%   \usepackage[utf8x]{inputenc}
%%   \usepackage[T1]{fontenc}
%%
\begingroup%
\makeatletter%
\begin{pgfpicture}%
\pgfpathrectangle{\pgfpointorigin}{\pgfqpoint{5.208661in}{3.219130in}}%
\pgfusepath{use as bounding box}%
\begin{pgfscope}%
\pgfsetbuttcap%
\pgfsetroundjoin%
\definecolor{currentfill}{rgb}{1.000000,1.000000,1.000000}%
\pgfsetfillcolor{currentfill}%
\pgfsetlinewidth{0.000000pt}%
\definecolor{currentstroke}{rgb}{1.000000,1.000000,1.000000}%
\pgfsetstrokecolor{currentstroke}%
\pgfsetdash{}{0pt}%
\pgfpathmoveto{\pgfqpoint{0.000000in}{0.000000in}}%
\pgfpathlineto{\pgfqpoint{5.208661in}{0.000000in}}%
\pgfpathlineto{\pgfqpoint{5.208661in}{3.219130in}}%
\pgfpathlineto{\pgfqpoint{0.000000in}{3.219130in}}%
\pgfpathclose%
\pgfusepath{fill}%
\end{pgfscope}%
\begin{pgfscope}%
\pgfsetbuttcap%
\pgfsetroundjoin%
\definecolor{currentfill}{rgb}{1.000000,1.000000,1.000000}%
\pgfsetfillcolor{currentfill}%
\pgfsetlinewidth{0.000000pt}%
\definecolor{currentstroke}{rgb}{0.000000,0.000000,0.000000}%
\pgfsetstrokecolor{currentstroke}%
\pgfsetstrokeopacity{0.000000}%
\pgfsetdash{}{0pt}%
\pgfpathmoveto{\pgfqpoint{0.651083in}{0.321913in}}%
\pgfpathlineto{\pgfqpoint{4.687795in}{0.321913in}}%
\pgfpathlineto{\pgfqpoint{4.687795in}{2.897217in}}%
\pgfpathlineto{\pgfqpoint{0.651083in}{2.897217in}}%
\pgfpathclose%
\pgfusepath{fill}%
\end{pgfscope}%
\begin{pgfscope}%
\pgfpathrectangle{\pgfqpoint{0.651083in}{0.321913in}}{\pgfqpoint{4.036712in}{2.575304in}} %
\pgfusepath{clip}%
\pgfsetrectcap%
\pgfsetroundjoin%
\pgfsetlinewidth{1.003750pt}%
\definecolor{currentstroke}{rgb}{0.000000,0.000000,0.000000}%
\pgfsetstrokecolor{currentstroke}%
\pgfsetdash{}{0pt}%
\pgfpathmoveto{\pgfqpoint{0.852918in}{2.802749in}}%
\pgfpathlineto{\pgfqpoint{0.873102in}{2.562628in}}%
\pgfpathlineto{\pgfqpoint{0.893285in}{2.363016in}}%
\pgfpathlineto{\pgfqpoint{0.913469in}{2.194554in}}%
\pgfpathlineto{\pgfqpoint{0.933653in}{2.050558in}}%
\pgfpathlineto{\pgfqpoint{0.953836in}{1.926127in}}%
\pgfpathlineto{\pgfqpoint{0.974020in}{1.817586in}}%
\pgfpathlineto{\pgfqpoint{0.994203in}{1.722124in}}%
\pgfpathlineto{\pgfqpoint{1.014387in}{1.637556in}}%
\pgfpathlineto{\pgfqpoint{1.034570in}{1.562157in}}%
\pgfpathlineto{\pgfqpoint{1.054754in}{1.494546in}}%
\pgfpathlineto{\pgfqpoint{1.074937in}{1.433608in}}%
\pgfpathlineto{\pgfqpoint{1.095121in}{1.378427in}}%
\pgfpathlineto{\pgfqpoint{1.115305in}{1.328251in}}%
\pgfpathlineto{\pgfqpoint{1.135488in}{1.282449in}}%
\pgfpathlineto{\pgfqpoint{1.155672in}{1.240495in}}%
\pgfpathlineto{\pgfqpoint{1.175855in}{1.201941in}}%
\pgfpathlineto{\pgfqpoint{1.196039in}{1.166408in}}%
\pgfpathlineto{\pgfqpoint{1.216222in}{1.133570in}}%
\pgfpathlineto{\pgfqpoint{1.236406in}{1.103145in}}%
\pgfpathlineto{\pgfqpoint{1.256590in}{1.074892in}}%
\pgfpathlineto{\pgfqpoint{1.296957in}{1.024078in}}%
\pgfpathlineto{\pgfqpoint{1.337324in}{0.979728in}}%
\pgfpathlineto{\pgfqpoint{1.377691in}{0.940758in}}%
\pgfpathlineto{\pgfqpoint{1.418058in}{0.906311in}}%
\pgfpathlineto{\pgfqpoint{1.458425in}{0.875702in}}%
\pgfpathlineto{\pgfqpoint{1.498792in}{0.848380in}}%
\pgfpathlineto{\pgfqpoint{1.539159in}{0.823893in}}%
\pgfpathlineto{\pgfqpoint{1.579527in}{0.801871in}}%
\pgfpathlineto{\pgfqpoint{1.619894in}{0.782003in}}%
\pgfpathlineto{\pgfqpoint{1.660261in}{0.764031in}}%
\pgfpathlineto{\pgfqpoint{1.720811in}{0.740162in}}%
\pgfpathlineto{\pgfqpoint{1.781362in}{0.719474in}}%
\pgfpathlineto{\pgfqpoint{1.841913in}{0.701491in}}%
\pgfpathlineto{\pgfqpoint{1.902464in}{0.685834in}}%
\pgfpathlineto{\pgfqpoint{1.963014in}{0.672194in}}%
\pgfpathlineto{\pgfqpoint{2.043748in}{0.656724in}}%
\pgfpathlineto{\pgfqpoint{2.124483in}{0.643937in}}%
\pgfpathlineto{\pgfqpoint{2.205217in}{0.633475in}}%
\pgfpathlineto{\pgfqpoint{2.285951in}{0.625062in}}%
\pgfpathlineto{\pgfqpoint{2.386869in}{0.617115in}}%
\pgfpathlineto{\pgfqpoint{2.487787in}{0.611752in}}%
\pgfpathlineto{\pgfqpoint{2.588705in}{0.608782in}}%
\pgfpathlineto{\pgfqpoint{2.689622in}{0.608103in}}%
\pgfpathlineto{\pgfqpoint{2.790540in}{0.609691in}}%
\pgfpathlineto{\pgfqpoint{2.891458in}{0.613601in}}%
\pgfpathlineto{\pgfqpoint{2.992376in}{0.619970in}}%
\pgfpathlineto{\pgfqpoint{3.073110in}{0.626986in}}%
\pgfpathlineto{\pgfqpoint{3.153844in}{0.635889in}}%
\pgfpathlineto{\pgfqpoint{3.234579in}{0.646904in}}%
\pgfpathlineto{\pgfqpoint{3.315313in}{0.660324in}}%
\pgfpathlineto{\pgfqpoint{3.396047in}{0.676533in}}%
\pgfpathlineto{\pgfqpoint{3.456598in}{0.690815in}}%
\pgfpathlineto{\pgfqpoint{3.517149in}{0.707210in}}%
\pgfpathlineto{\pgfqpoint{3.577699in}{0.726048in}}%
\pgfpathlineto{\pgfqpoint{3.638250in}{0.747738in}}%
\pgfpathlineto{\pgfqpoint{3.678617in}{0.764031in}}%
\pgfpathlineto{\pgfqpoint{3.718984in}{0.782003in}}%
\pgfpathlineto{\pgfqpoint{3.759351in}{0.801871in}}%
\pgfpathlineto{\pgfqpoint{3.799718in}{0.823893in}}%
\pgfpathlineto{\pgfqpoint{3.840086in}{0.848380in}}%
\pgfpathlineto{\pgfqpoint{3.880453in}{0.875702in}}%
\pgfpathlineto{\pgfqpoint{3.920820in}{0.906311in}}%
\pgfpathlineto{\pgfqpoint{3.961187in}{0.940758in}}%
\pgfpathlineto{\pgfqpoint{4.001554in}{0.979728in}}%
\pgfpathlineto{\pgfqpoint{4.041921in}{1.024078in}}%
\pgfpathlineto{\pgfqpoint{4.062105in}{1.048598in}}%
\pgfpathlineto{\pgfqpoint{4.082288in}{1.074892in}}%
\pgfpathlineto{\pgfqpoint{4.102472in}{1.103145in}}%
\pgfpathlineto{\pgfqpoint{4.122655in}{1.133570in}}%
\pgfpathlineto{\pgfqpoint{4.142839in}{1.166408in}}%
\pgfpathlineto{\pgfqpoint{4.163023in}{1.201941in}}%
\pgfpathlineto{\pgfqpoint{4.183206in}{1.240495in}}%
\pgfpathlineto{\pgfqpoint{4.203390in}{1.282449in}}%
\pgfpathlineto{\pgfqpoint{4.223573in}{1.328251in}}%
\pgfpathlineto{\pgfqpoint{4.243757in}{1.378427in}}%
\pgfpathlineto{\pgfqpoint{4.263940in}{1.433608in}}%
\pgfpathlineto{\pgfqpoint{4.284124in}{1.494546in}}%
\pgfpathlineto{\pgfqpoint{4.304307in}{1.562157in}}%
\pgfpathlineto{\pgfqpoint{4.324491in}{1.637556in}}%
\pgfpathlineto{\pgfqpoint{4.344675in}{1.722124in}}%
\pgfpathlineto{\pgfqpoint{4.364858in}{1.817586in}}%
\pgfpathlineto{\pgfqpoint{4.385042in}{1.926127in}}%
\pgfpathlineto{\pgfqpoint{4.405225in}{2.050558in}}%
\pgfpathlineto{\pgfqpoint{4.425409in}{2.194554in}}%
\pgfpathlineto{\pgfqpoint{4.445592in}{2.363016in}}%
\pgfpathlineto{\pgfqpoint{4.465776in}{2.562628in}}%
\pgfpathlineto{\pgfqpoint{4.485960in}{2.802749in}}%
\pgfpathlineto{\pgfqpoint{4.485960in}{2.802749in}}%
\pgfusepath{stroke}%
\end{pgfscope}%
\begin{pgfscope}%
\pgfpathrectangle{\pgfqpoint{0.651083in}{0.321913in}}{\pgfqpoint{4.036712in}{2.575304in}} %
\pgfusepath{clip}%
\pgfsetrectcap%
\pgfsetroundjoin%
\pgfsetlinewidth{1.003750pt}%
\definecolor{currentstroke}{rgb}{0.000000,0.000000,1.000000}%
\pgfsetstrokecolor{currentstroke}%
\pgfsetdash{}{0pt}%
\pgfpathmoveto{\pgfqpoint{0.852918in}{0.633201in}}%
\pgfpathlineto{\pgfqpoint{1.579527in}{0.620023in}}%
\pgfpathlineto{\pgfqpoint{2.144666in}{0.611817in}}%
\pgfpathlineto{\pgfqpoint{2.507970in}{0.608516in}}%
\pgfpathlineto{\pgfqpoint{2.770357in}{0.608281in}}%
\pgfpathlineto{\pgfqpoint{2.952009in}{0.610146in}}%
\pgfpathlineto{\pgfqpoint{3.093294in}{0.613519in}}%
\pgfpathlineto{\pgfqpoint{3.214395in}{0.618438in}}%
\pgfpathlineto{\pgfqpoint{3.315313in}{0.624558in}}%
\pgfpathlineto{\pgfqpoint{3.396047in}{0.631219in}}%
\pgfpathlineto{\pgfqpoint{3.476781in}{0.639896in}}%
\pgfpathlineto{\pgfqpoint{3.557516in}{0.651112in}}%
\pgfpathlineto{\pgfqpoint{3.618066in}{0.661579in}}%
\pgfpathlineto{\pgfqpoint{3.678617in}{0.674178in}}%
\pgfpathlineto{\pgfqpoint{3.718984in}{0.693172in}}%
\pgfpathlineto{\pgfqpoint{3.759351in}{0.714214in}}%
\pgfpathlineto{\pgfqpoint{3.799718in}{0.737577in}}%
\pgfpathlineto{\pgfqpoint{3.840086in}{0.763581in}}%
\pgfpathlineto{\pgfqpoint{3.880453in}{0.792610in}}%
\pgfpathlineto{\pgfqpoint{3.920820in}{0.825123in}}%
\pgfpathlineto{\pgfqpoint{3.961187in}{0.861682in}}%
\pgfpathlineto{\pgfqpoint{4.001554in}{0.902976in}}%
\pgfpathlineto{\pgfqpoint{4.041921in}{0.949867in}}%
\pgfpathlineto{\pgfqpoint{4.062105in}{0.975742in}}%
\pgfpathlineto{\pgfqpoint{4.082288in}{1.003447in}}%
\pgfpathlineto{\pgfqpoint{4.102472in}{1.033168in}}%
\pgfpathlineto{\pgfqpoint{4.122655in}{1.065117in}}%
\pgfpathlineto{\pgfqpoint{4.142839in}{1.099538in}}%
\pgfpathlineto{\pgfqpoint{4.163023in}{1.136712in}}%
\pgfpathlineto{\pgfqpoint{4.183206in}{1.176965in}}%
\pgfpathlineto{\pgfqpoint{4.203390in}{1.220678in}}%
\pgfpathlineto{\pgfqpoint{4.223573in}{1.268298in}}%
\pgfpathlineto{\pgfqpoint{4.243757in}{1.320354in}}%
\pgfpathlineto{\pgfqpoint{4.263940in}{1.377473in}}%
\pgfpathlineto{\pgfqpoint{4.284124in}{1.440413in}}%
\pgfpathlineto{\pgfqpoint{4.304307in}{1.510085in}}%
\pgfpathlineto{\pgfqpoint{4.324491in}{1.587609in}}%
\pgfpathlineto{\pgfqpoint{4.344675in}{1.674365in}}%
\pgfpathlineto{\pgfqpoint{4.364858in}{1.772078in}}%
\pgfpathlineto{\pgfqpoint{4.385042in}{1.882935in}}%
\pgfpathlineto{\pgfqpoint{4.405225in}{2.009747in}}%
\pgfpathlineto{\pgfqpoint{4.425409in}{2.156191in}}%
\pgfpathlineto{\pgfqpoint{4.445592in}{2.327168in}}%
\pgfpathlineto{\pgfqpoint{4.465776in}{2.529363in}}%
\pgfpathlineto{\pgfqpoint{4.485960in}{2.772135in}}%
\pgfpathlineto{\pgfqpoint{4.485960in}{2.772135in}}%
\pgfusepath{stroke}%
\end{pgfscope}%
\begin{pgfscope}%
\pgfpathrectangle{\pgfqpoint{0.651083in}{0.321913in}}{\pgfqpoint{4.036712in}{2.575304in}} %
\pgfusepath{clip}%
\pgfsetrectcap%
\pgfsetroundjoin%
\pgfsetlinewidth{1.003750pt}%
\definecolor{currentstroke}{rgb}{0.000000,0.500000,0.000000}%
\pgfsetstrokecolor{currentstroke}%
\pgfsetdash{}{0pt}%
\pgfpathmoveto{\pgfqpoint{0.852918in}{0.629682in}}%
\pgfpathlineto{\pgfqpoint{1.236406in}{0.621550in}}%
\pgfpathlineto{\pgfqpoint{1.660261in}{0.614888in}}%
\pgfpathlineto{\pgfqpoint{2.084116in}{0.610450in}}%
\pgfpathlineto{\pgfqpoint{2.507970in}{0.608252in}}%
\pgfpathlineto{\pgfqpoint{2.891458in}{0.608455in}}%
\pgfpathlineto{\pgfqpoint{3.234579in}{0.610885in}}%
\pgfpathlineto{\pgfqpoint{3.517149in}{0.615084in}}%
\pgfpathlineto{\pgfqpoint{3.759351in}{0.620960in}}%
\pgfpathlineto{\pgfqpoint{3.961187in}{0.628187in}}%
\pgfpathlineto{\pgfqpoint{4.122655in}{0.636140in}}%
\pgfpathlineto{\pgfqpoint{4.263940in}{0.645275in}}%
\pgfpathlineto{\pgfqpoint{4.385042in}{0.655226in}}%
\pgfpathlineto{\pgfqpoint{4.485960in}{0.665398in}}%
\pgfpathlineto{\pgfqpoint{4.485960in}{0.665398in}}%
\pgfusepath{stroke}%
\end{pgfscope}%
\begin{pgfscope}%
\pgfpathrectangle{\pgfqpoint{0.651083in}{0.321913in}}{\pgfqpoint{4.036712in}{2.575304in}} %
\pgfusepath{clip}%
\pgfsetrectcap%
\pgfsetroundjoin%
\pgfsetlinewidth{1.003750pt}%
\definecolor{currentstroke}{rgb}{1.000000,0.000000,0.000000}%
\pgfsetstrokecolor{currentstroke}%
\pgfsetdash{}{0pt}%
\pgfpathmoveto{\pgfqpoint{0.852918in}{0.665398in}}%
\pgfpathlineto{\pgfqpoint{0.953836in}{0.655226in}}%
\pgfpathlineto{\pgfqpoint{1.074937in}{0.645275in}}%
\pgfpathlineto{\pgfqpoint{1.216222in}{0.636140in}}%
\pgfpathlineto{\pgfqpoint{1.377691in}{0.628187in}}%
\pgfpathlineto{\pgfqpoint{1.559343in}{0.621573in}}%
\pgfpathlineto{\pgfqpoint{1.781362in}{0.615894in}}%
\pgfpathlineto{\pgfqpoint{2.043748in}{0.611591in}}%
\pgfpathlineto{\pgfqpoint{2.346502in}{0.608919in}}%
\pgfpathlineto{\pgfqpoint{2.709806in}{0.608070in}}%
\pgfpathlineto{\pgfqpoint{3.113477in}{0.609460in}}%
\pgfpathlineto{\pgfqpoint{3.537332in}{0.613166in}}%
\pgfpathlineto{\pgfqpoint{3.961187in}{0.619073in}}%
\pgfpathlineto{\pgfqpoint{4.364858in}{0.626880in}}%
\pgfpathlineto{\pgfqpoint{4.485960in}{0.629682in}}%
\pgfpathlineto{\pgfqpoint{4.485960in}{0.629682in}}%
\pgfusepath{stroke}%
\end{pgfscope}%
\begin{pgfscope}%
\pgfpathrectangle{\pgfqpoint{0.651083in}{0.321913in}}{\pgfqpoint{4.036712in}{2.575304in}} %
\pgfusepath{clip}%
\pgfsetrectcap%
\pgfsetroundjoin%
\pgfsetlinewidth{1.003750pt}%
\definecolor{currentstroke}{rgb}{0.000000,0.750000,0.750000}%
\pgfsetstrokecolor{currentstroke}%
\pgfsetdash{}{0pt}%
\pgfpathmoveto{\pgfqpoint{0.852918in}{2.772135in}}%
\pgfpathlineto{\pgfqpoint{0.873102in}{2.529363in}}%
\pgfpathlineto{\pgfqpoint{0.893285in}{2.327168in}}%
\pgfpathlineto{\pgfqpoint{0.913469in}{2.156191in}}%
\pgfpathlineto{\pgfqpoint{0.933653in}{2.009747in}}%
\pgfpathlineto{\pgfqpoint{0.953836in}{1.882935in}}%
\pgfpathlineto{\pgfqpoint{0.974020in}{1.772078in}}%
\pgfpathlineto{\pgfqpoint{0.994203in}{1.674365in}}%
\pgfpathlineto{\pgfqpoint{1.014387in}{1.587609in}}%
\pgfpathlineto{\pgfqpoint{1.034570in}{1.510085in}}%
\pgfpathlineto{\pgfqpoint{1.054754in}{1.440413in}}%
\pgfpathlineto{\pgfqpoint{1.074937in}{1.377473in}}%
\pgfpathlineto{\pgfqpoint{1.095121in}{1.320354in}}%
\pgfpathlineto{\pgfqpoint{1.115305in}{1.268298in}}%
\pgfpathlineto{\pgfqpoint{1.135488in}{1.220678in}}%
\pgfpathlineto{\pgfqpoint{1.155672in}{1.176965in}}%
\pgfpathlineto{\pgfqpoint{1.175855in}{1.136712in}}%
\pgfpathlineto{\pgfqpoint{1.196039in}{1.099538in}}%
\pgfpathlineto{\pgfqpoint{1.216222in}{1.065117in}}%
\pgfpathlineto{\pgfqpoint{1.236406in}{1.033168in}}%
\pgfpathlineto{\pgfqpoint{1.256590in}{1.003447in}}%
\pgfpathlineto{\pgfqpoint{1.296957in}{0.949867in}}%
\pgfpathlineto{\pgfqpoint{1.337324in}{0.902976in}}%
\pgfpathlineto{\pgfqpoint{1.377691in}{0.861682in}}%
\pgfpathlineto{\pgfqpoint{1.418058in}{0.825123in}}%
\pgfpathlineto{\pgfqpoint{1.458425in}{0.792610in}}%
\pgfpathlineto{\pgfqpoint{1.498792in}{0.763581in}}%
\pgfpathlineto{\pgfqpoint{1.539159in}{0.737577in}}%
\pgfpathlineto{\pgfqpoint{1.579527in}{0.714214in}}%
\pgfpathlineto{\pgfqpoint{1.619894in}{0.693172in}}%
\pgfpathlineto{\pgfqpoint{1.660261in}{0.674178in}}%
\pgfpathlineto{\pgfqpoint{1.720811in}{0.661579in}}%
\pgfpathlineto{\pgfqpoint{1.781362in}{0.651112in}}%
\pgfpathlineto{\pgfqpoint{1.862096in}{0.639896in}}%
\pgfpathlineto{\pgfqpoint{1.942831in}{0.631219in}}%
\pgfpathlineto{\pgfqpoint{2.023565in}{0.624558in}}%
\pgfpathlineto{\pgfqpoint{2.124483in}{0.618438in}}%
\pgfpathlineto{\pgfqpoint{2.245584in}{0.613519in}}%
\pgfpathlineto{\pgfqpoint{2.386869in}{0.610146in}}%
\pgfpathlineto{\pgfqpoint{2.568521in}{0.608281in}}%
\pgfpathlineto{\pgfqpoint{2.790540in}{0.608324in}}%
\pgfpathlineto{\pgfqpoint{3.093294in}{0.610675in}}%
\pgfpathlineto{\pgfqpoint{3.537332in}{0.616514in}}%
\pgfpathlineto{\pgfqpoint{4.163023in}{0.627059in}}%
\pgfpathlineto{\pgfqpoint{4.485960in}{0.633201in}}%
\pgfpathlineto{\pgfqpoint{4.485960in}{0.633201in}}%
\pgfusepath{stroke}%
\end{pgfscope}%
\begin{pgfscope}%
\pgfsetbuttcap%
\pgfsetroundjoin%
\definecolor{currentfill}{rgb}{0.000000,0.000000,0.000000}%
\pgfsetfillcolor{currentfill}%
\pgfsetlinewidth{0.501875pt}%
\definecolor{currentstroke}{rgb}{0.000000,0.000000,0.000000}%
\pgfsetstrokecolor{currentstroke}%
\pgfsetdash{}{0pt}%
\pgfsys@defobject{currentmarker}{\pgfqpoint{0.000000in}{0.000000in}}{\pgfqpoint{0.000000in}{0.055556in}}{%
\pgfpathmoveto{\pgfqpoint{0.000000in}{0.000000in}}%
\pgfpathlineto{\pgfqpoint{0.000000in}{0.055556in}}%
\pgfusepath{stroke,fill}%
}%
\begin{pgfscope}%
\pgfsys@transformshift{0.651083in}{0.321913in}%
\pgfsys@useobject{currentmarker}{}%
\end{pgfscope}%
\end{pgfscope}%
\begin{pgfscope}%
\pgfsetbuttcap%
\pgfsetroundjoin%
\definecolor{currentfill}{rgb}{0.000000,0.000000,0.000000}%
\pgfsetfillcolor{currentfill}%
\pgfsetlinewidth{0.501875pt}%
\definecolor{currentstroke}{rgb}{0.000000,0.000000,0.000000}%
\pgfsetstrokecolor{currentstroke}%
\pgfsetdash{}{0pt}%
\pgfsys@defobject{currentmarker}{\pgfqpoint{0.000000in}{-0.055556in}}{\pgfqpoint{0.000000in}{0.000000in}}{%
\pgfpathmoveto{\pgfqpoint{0.000000in}{0.000000in}}%
\pgfpathlineto{\pgfqpoint{0.000000in}{-0.055556in}}%
\pgfusepath{stroke,fill}%
}%
\begin{pgfscope}%
\pgfsys@transformshift{0.651083in}{2.897217in}%
\pgfsys@useobject{currentmarker}{}%
\end{pgfscope}%
\end{pgfscope}%
\begin{pgfscope}%
\pgftext[x=0.651083in,y=0.266357in,,top]{{\rmfamily\fontsize{8.000000}{9.600000}\selectfont \(\displaystyle -1.0\)}}%
\end{pgfscope}%
\begin{pgfscope}%
\pgfsetbuttcap%
\pgfsetroundjoin%
\definecolor{currentfill}{rgb}{0.000000,0.000000,0.000000}%
\pgfsetfillcolor{currentfill}%
\pgfsetlinewidth{0.501875pt}%
\definecolor{currentstroke}{rgb}{0.000000,0.000000,0.000000}%
\pgfsetstrokecolor{currentstroke}%
\pgfsetdash{}{0pt}%
\pgfsys@defobject{currentmarker}{\pgfqpoint{0.000000in}{0.000000in}}{\pgfqpoint{0.000000in}{0.055556in}}{%
\pgfpathmoveto{\pgfqpoint{0.000000in}{0.000000in}}%
\pgfpathlineto{\pgfqpoint{0.000000in}{0.055556in}}%
\pgfusepath{stroke,fill}%
}%
\begin{pgfscope}%
\pgfsys@transformshift{1.660261in}{0.321913in}%
\pgfsys@useobject{currentmarker}{}%
\end{pgfscope}%
\end{pgfscope}%
\begin{pgfscope}%
\pgfsetbuttcap%
\pgfsetroundjoin%
\definecolor{currentfill}{rgb}{0.000000,0.000000,0.000000}%
\pgfsetfillcolor{currentfill}%
\pgfsetlinewidth{0.501875pt}%
\definecolor{currentstroke}{rgb}{0.000000,0.000000,0.000000}%
\pgfsetstrokecolor{currentstroke}%
\pgfsetdash{}{0pt}%
\pgfsys@defobject{currentmarker}{\pgfqpoint{0.000000in}{-0.055556in}}{\pgfqpoint{0.000000in}{0.000000in}}{%
\pgfpathmoveto{\pgfqpoint{0.000000in}{0.000000in}}%
\pgfpathlineto{\pgfqpoint{0.000000in}{-0.055556in}}%
\pgfusepath{stroke,fill}%
}%
\begin{pgfscope}%
\pgfsys@transformshift{1.660261in}{2.897217in}%
\pgfsys@useobject{currentmarker}{}%
\end{pgfscope}%
\end{pgfscope}%
\begin{pgfscope}%
\pgftext[x=1.660261in,y=0.266357in,,top]{{\rmfamily\fontsize{8.000000}{9.600000}\selectfont \(\displaystyle -0.5\)}}%
\end{pgfscope}%
\begin{pgfscope}%
\pgfsetbuttcap%
\pgfsetroundjoin%
\definecolor{currentfill}{rgb}{0.000000,0.000000,0.000000}%
\pgfsetfillcolor{currentfill}%
\pgfsetlinewidth{0.501875pt}%
\definecolor{currentstroke}{rgb}{0.000000,0.000000,0.000000}%
\pgfsetstrokecolor{currentstroke}%
\pgfsetdash{}{0pt}%
\pgfsys@defobject{currentmarker}{\pgfqpoint{0.000000in}{0.000000in}}{\pgfqpoint{0.000000in}{0.055556in}}{%
\pgfpathmoveto{\pgfqpoint{0.000000in}{0.000000in}}%
\pgfpathlineto{\pgfqpoint{0.000000in}{0.055556in}}%
\pgfusepath{stroke,fill}%
}%
\begin{pgfscope}%
\pgfsys@transformshift{2.669439in}{0.321913in}%
\pgfsys@useobject{currentmarker}{}%
\end{pgfscope}%
\end{pgfscope}%
\begin{pgfscope}%
\pgfsetbuttcap%
\pgfsetroundjoin%
\definecolor{currentfill}{rgb}{0.000000,0.000000,0.000000}%
\pgfsetfillcolor{currentfill}%
\pgfsetlinewidth{0.501875pt}%
\definecolor{currentstroke}{rgb}{0.000000,0.000000,0.000000}%
\pgfsetstrokecolor{currentstroke}%
\pgfsetdash{}{0pt}%
\pgfsys@defobject{currentmarker}{\pgfqpoint{0.000000in}{-0.055556in}}{\pgfqpoint{0.000000in}{0.000000in}}{%
\pgfpathmoveto{\pgfqpoint{0.000000in}{0.000000in}}%
\pgfpathlineto{\pgfqpoint{0.000000in}{-0.055556in}}%
\pgfusepath{stroke,fill}%
}%
\begin{pgfscope}%
\pgfsys@transformshift{2.669439in}{2.897217in}%
\pgfsys@useobject{currentmarker}{}%
\end{pgfscope}%
\end{pgfscope}%
\begin{pgfscope}%
\pgftext[x=2.669439in,y=0.266357in,,top]{{\rmfamily\fontsize{8.000000}{9.600000}\selectfont \(\displaystyle 0.0\)}}%
\end{pgfscope}%
\begin{pgfscope}%
\pgfsetbuttcap%
\pgfsetroundjoin%
\definecolor{currentfill}{rgb}{0.000000,0.000000,0.000000}%
\pgfsetfillcolor{currentfill}%
\pgfsetlinewidth{0.501875pt}%
\definecolor{currentstroke}{rgb}{0.000000,0.000000,0.000000}%
\pgfsetstrokecolor{currentstroke}%
\pgfsetdash{}{0pt}%
\pgfsys@defobject{currentmarker}{\pgfqpoint{0.000000in}{0.000000in}}{\pgfqpoint{0.000000in}{0.055556in}}{%
\pgfpathmoveto{\pgfqpoint{0.000000in}{0.000000in}}%
\pgfpathlineto{\pgfqpoint{0.000000in}{0.055556in}}%
\pgfusepath{stroke,fill}%
}%
\begin{pgfscope}%
\pgfsys@transformshift{3.678617in}{0.321913in}%
\pgfsys@useobject{currentmarker}{}%
\end{pgfscope}%
\end{pgfscope}%
\begin{pgfscope}%
\pgfsetbuttcap%
\pgfsetroundjoin%
\definecolor{currentfill}{rgb}{0.000000,0.000000,0.000000}%
\pgfsetfillcolor{currentfill}%
\pgfsetlinewidth{0.501875pt}%
\definecolor{currentstroke}{rgb}{0.000000,0.000000,0.000000}%
\pgfsetstrokecolor{currentstroke}%
\pgfsetdash{}{0pt}%
\pgfsys@defobject{currentmarker}{\pgfqpoint{0.000000in}{-0.055556in}}{\pgfqpoint{0.000000in}{0.000000in}}{%
\pgfpathmoveto{\pgfqpoint{0.000000in}{0.000000in}}%
\pgfpathlineto{\pgfqpoint{0.000000in}{-0.055556in}}%
\pgfusepath{stroke,fill}%
}%
\begin{pgfscope}%
\pgfsys@transformshift{3.678617in}{2.897217in}%
\pgfsys@useobject{currentmarker}{}%
\end{pgfscope}%
\end{pgfscope}%
\begin{pgfscope}%
\pgftext[x=3.678617in,y=0.266357in,,top]{{\rmfamily\fontsize{8.000000}{9.600000}\selectfont \(\displaystyle 0.5\)}}%
\end{pgfscope}%
\begin{pgfscope}%
\pgfsetbuttcap%
\pgfsetroundjoin%
\definecolor{currentfill}{rgb}{0.000000,0.000000,0.000000}%
\pgfsetfillcolor{currentfill}%
\pgfsetlinewidth{0.501875pt}%
\definecolor{currentstroke}{rgb}{0.000000,0.000000,0.000000}%
\pgfsetstrokecolor{currentstroke}%
\pgfsetdash{}{0pt}%
\pgfsys@defobject{currentmarker}{\pgfqpoint{0.000000in}{0.000000in}}{\pgfqpoint{0.000000in}{0.055556in}}{%
\pgfpathmoveto{\pgfqpoint{0.000000in}{0.000000in}}%
\pgfpathlineto{\pgfqpoint{0.000000in}{0.055556in}}%
\pgfusepath{stroke,fill}%
}%
\begin{pgfscope}%
\pgfsys@transformshift{4.687795in}{0.321913in}%
\pgfsys@useobject{currentmarker}{}%
\end{pgfscope}%
\end{pgfscope}%
\begin{pgfscope}%
\pgfsetbuttcap%
\pgfsetroundjoin%
\definecolor{currentfill}{rgb}{0.000000,0.000000,0.000000}%
\pgfsetfillcolor{currentfill}%
\pgfsetlinewidth{0.501875pt}%
\definecolor{currentstroke}{rgb}{0.000000,0.000000,0.000000}%
\pgfsetstrokecolor{currentstroke}%
\pgfsetdash{}{0pt}%
\pgfsys@defobject{currentmarker}{\pgfqpoint{0.000000in}{-0.055556in}}{\pgfqpoint{0.000000in}{0.000000in}}{%
\pgfpathmoveto{\pgfqpoint{0.000000in}{0.000000in}}%
\pgfpathlineto{\pgfqpoint{0.000000in}{-0.055556in}}%
\pgfusepath{stroke,fill}%
}%
\begin{pgfscope}%
\pgfsys@transformshift{4.687795in}{2.897217in}%
\pgfsys@useobject{currentmarker}{}%
\end{pgfscope}%
\end{pgfscope}%
\begin{pgfscope}%
\pgftext[x=4.687795in,y=0.266357in,,top]{{\rmfamily\fontsize{8.000000}{9.600000}\selectfont \(\displaystyle 1.0\)}}%
\end{pgfscope}%
\begin{pgfscope}%
\pgftext[x=2.669439in,y=0.098789in,,top]{{\rmfamily\fontsize{10.000000}{12.000000}\selectfont \(\displaystyle  \delta \)}}%
\end{pgfscope}%
\begin{pgfscope}%
\pgfsetbuttcap%
\pgfsetroundjoin%
\definecolor{currentfill}{rgb}{0.000000,0.000000,0.000000}%
\pgfsetfillcolor{currentfill}%
\pgfsetlinewidth{0.501875pt}%
\definecolor{currentstroke}{rgb}{0.000000,0.000000,0.000000}%
\pgfsetstrokecolor{currentstroke}%
\pgfsetdash{}{0pt}%
\pgfsys@defobject{currentmarker}{\pgfqpoint{0.000000in}{0.000000in}}{\pgfqpoint{0.055556in}{0.000000in}}{%
\pgfpathmoveto{\pgfqpoint{0.000000in}{0.000000in}}%
\pgfpathlineto{\pgfqpoint{0.055556in}{0.000000in}}%
\pgfusepath{stroke,fill}%
}%
\begin{pgfscope}%
\pgfsys@transformshift{0.651083in}{0.321913in}%
\pgfsys@useobject{currentmarker}{}%
\end{pgfscope}%
\end{pgfscope}%
\begin{pgfscope}%
\pgfsetbuttcap%
\pgfsetroundjoin%
\definecolor{currentfill}{rgb}{0.000000,0.000000,0.000000}%
\pgfsetfillcolor{currentfill}%
\pgfsetlinewidth{0.501875pt}%
\definecolor{currentstroke}{rgb}{0.000000,0.000000,0.000000}%
\pgfsetstrokecolor{currentstroke}%
\pgfsetdash{}{0pt}%
\pgfsys@defobject{currentmarker}{\pgfqpoint{-0.055556in}{0.000000in}}{\pgfqpoint{0.000000in}{0.000000in}}{%
\pgfpathmoveto{\pgfqpoint{0.000000in}{0.000000in}}%
\pgfpathlineto{\pgfqpoint{-0.055556in}{0.000000in}}%
\pgfusepath{stroke,fill}%
}%
\begin{pgfscope}%
\pgfsys@transformshift{4.687795in}{0.321913in}%
\pgfsys@useobject{currentmarker}{}%
\end{pgfscope}%
\end{pgfscope}%
\begin{pgfscope}%
\pgftext[x=0.595527in,y=0.321913in,right,]{{\rmfamily\fontsize{8.000000}{9.600000}\selectfont \(\displaystyle 0\)}}%
\end{pgfscope}%
\begin{pgfscope}%
\pgfsetbuttcap%
\pgfsetroundjoin%
\definecolor{currentfill}{rgb}{0.000000,0.000000,0.000000}%
\pgfsetfillcolor{currentfill}%
\pgfsetlinewidth{0.501875pt}%
\definecolor{currentstroke}{rgb}{0.000000,0.000000,0.000000}%
\pgfsetstrokecolor{currentstroke}%
\pgfsetdash{}{0pt}%
\pgfsys@defobject{currentmarker}{\pgfqpoint{0.000000in}{0.000000in}}{\pgfqpoint{0.055556in}{0.000000in}}{%
\pgfpathmoveto{\pgfqpoint{0.000000in}{0.000000in}}%
\pgfpathlineto{\pgfqpoint{0.055556in}{0.000000in}}%
\pgfusepath{stroke,fill}%
}%
\begin{pgfscope}%
\pgfsys@transformshift{0.651083in}{0.608058in}%
\pgfsys@useobject{currentmarker}{}%
\end{pgfscope}%
\end{pgfscope}%
\begin{pgfscope}%
\pgfsetbuttcap%
\pgfsetroundjoin%
\definecolor{currentfill}{rgb}{0.000000,0.000000,0.000000}%
\pgfsetfillcolor{currentfill}%
\pgfsetlinewidth{0.501875pt}%
\definecolor{currentstroke}{rgb}{0.000000,0.000000,0.000000}%
\pgfsetstrokecolor{currentstroke}%
\pgfsetdash{}{0pt}%
\pgfsys@defobject{currentmarker}{\pgfqpoint{-0.055556in}{0.000000in}}{\pgfqpoint{0.000000in}{0.000000in}}{%
\pgfpathmoveto{\pgfqpoint{0.000000in}{0.000000in}}%
\pgfpathlineto{\pgfqpoint{-0.055556in}{0.000000in}}%
\pgfusepath{stroke,fill}%
}%
\begin{pgfscope}%
\pgfsys@transformshift{4.687795in}{0.608058in}%
\pgfsys@useobject{currentmarker}{}%
\end{pgfscope}%
\end{pgfscope}%
\begin{pgfscope}%
\pgftext[x=0.595527in,y=0.608058in,right,]{{\rmfamily\fontsize{8.000000}{9.600000}\selectfont \(\displaystyle 1\)}}%
\end{pgfscope}%
\begin{pgfscope}%
\pgfsetbuttcap%
\pgfsetroundjoin%
\definecolor{currentfill}{rgb}{0.000000,0.000000,0.000000}%
\pgfsetfillcolor{currentfill}%
\pgfsetlinewidth{0.501875pt}%
\definecolor{currentstroke}{rgb}{0.000000,0.000000,0.000000}%
\pgfsetstrokecolor{currentstroke}%
\pgfsetdash{}{0pt}%
\pgfsys@defobject{currentmarker}{\pgfqpoint{0.000000in}{0.000000in}}{\pgfqpoint{0.055556in}{0.000000in}}{%
\pgfpathmoveto{\pgfqpoint{0.000000in}{0.000000in}}%
\pgfpathlineto{\pgfqpoint{0.055556in}{0.000000in}}%
\pgfusepath{stroke,fill}%
}%
\begin{pgfscope}%
\pgfsys@transformshift{0.651083in}{0.894203in}%
\pgfsys@useobject{currentmarker}{}%
\end{pgfscope}%
\end{pgfscope}%
\begin{pgfscope}%
\pgfsetbuttcap%
\pgfsetroundjoin%
\definecolor{currentfill}{rgb}{0.000000,0.000000,0.000000}%
\pgfsetfillcolor{currentfill}%
\pgfsetlinewidth{0.501875pt}%
\definecolor{currentstroke}{rgb}{0.000000,0.000000,0.000000}%
\pgfsetstrokecolor{currentstroke}%
\pgfsetdash{}{0pt}%
\pgfsys@defobject{currentmarker}{\pgfqpoint{-0.055556in}{0.000000in}}{\pgfqpoint{0.000000in}{0.000000in}}{%
\pgfpathmoveto{\pgfqpoint{0.000000in}{0.000000in}}%
\pgfpathlineto{\pgfqpoint{-0.055556in}{0.000000in}}%
\pgfusepath{stroke,fill}%
}%
\begin{pgfscope}%
\pgfsys@transformshift{4.687795in}{0.894203in}%
\pgfsys@useobject{currentmarker}{}%
\end{pgfscope}%
\end{pgfscope}%
\begin{pgfscope}%
\pgftext[x=0.595527in,y=0.894203in,right,]{{\rmfamily\fontsize{8.000000}{9.600000}\selectfont \(\displaystyle 2\)}}%
\end{pgfscope}%
\begin{pgfscope}%
\pgfsetbuttcap%
\pgfsetroundjoin%
\definecolor{currentfill}{rgb}{0.000000,0.000000,0.000000}%
\pgfsetfillcolor{currentfill}%
\pgfsetlinewidth{0.501875pt}%
\definecolor{currentstroke}{rgb}{0.000000,0.000000,0.000000}%
\pgfsetstrokecolor{currentstroke}%
\pgfsetdash{}{0pt}%
\pgfsys@defobject{currentmarker}{\pgfqpoint{0.000000in}{0.000000in}}{\pgfqpoint{0.055556in}{0.000000in}}{%
\pgfpathmoveto{\pgfqpoint{0.000000in}{0.000000in}}%
\pgfpathlineto{\pgfqpoint{0.055556in}{0.000000in}}%
\pgfusepath{stroke,fill}%
}%
\begin{pgfscope}%
\pgfsys@transformshift{0.651083in}{1.180348in}%
\pgfsys@useobject{currentmarker}{}%
\end{pgfscope}%
\end{pgfscope}%
\begin{pgfscope}%
\pgfsetbuttcap%
\pgfsetroundjoin%
\definecolor{currentfill}{rgb}{0.000000,0.000000,0.000000}%
\pgfsetfillcolor{currentfill}%
\pgfsetlinewidth{0.501875pt}%
\definecolor{currentstroke}{rgb}{0.000000,0.000000,0.000000}%
\pgfsetstrokecolor{currentstroke}%
\pgfsetdash{}{0pt}%
\pgfsys@defobject{currentmarker}{\pgfqpoint{-0.055556in}{0.000000in}}{\pgfqpoint{0.000000in}{0.000000in}}{%
\pgfpathmoveto{\pgfqpoint{0.000000in}{0.000000in}}%
\pgfpathlineto{\pgfqpoint{-0.055556in}{0.000000in}}%
\pgfusepath{stroke,fill}%
}%
\begin{pgfscope}%
\pgfsys@transformshift{4.687795in}{1.180348in}%
\pgfsys@useobject{currentmarker}{}%
\end{pgfscope}%
\end{pgfscope}%
\begin{pgfscope}%
\pgftext[x=0.595527in,y=1.180348in,right,]{{\rmfamily\fontsize{8.000000}{9.600000}\selectfont \(\displaystyle 3\)}}%
\end{pgfscope}%
\begin{pgfscope}%
\pgfsetbuttcap%
\pgfsetroundjoin%
\definecolor{currentfill}{rgb}{0.000000,0.000000,0.000000}%
\pgfsetfillcolor{currentfill}%
\pgfsetlinewidth{0.501875pt}%
\definecolor{currentstroke}{rgb}{0.000000,0.000000,0.000000}%
\pgfsetstrokecolor{currentstroke}%
\pgfsetdash{}{0pt}%
\pgfsys@defobject{currentmarker}{\pgfqpoint{0.000000in}{0.000000in}}{\pgfqpoint{0.055556in}{0.000000in}}{%
\pgfpathmoveto{\pgfqpoint{0.000000in}{0.000000in}}%
\pgfpathlineto{\pgfqpoint{0.055556in}{0.000000in}}%
\pgfusepath{stroke,fill}%
}%
\begin{pgfscope}%
\pgfsys@transformshift{0.651083in}{1.466492in}%
\pgfsys@useobject{currentmarker}{}%
\end{pgfscope}%
\end{pgfscope}%
\begin{pgfscope}%
\pgfsetbuttcap%
\pgfsetroundjoin%
\definecolor{currentfill}{rgb}{0.000000,0.000000,0.000000}%
\pgfsetfillcolor{currentfill}%
\pgfsetlinewidth{0.501875pt}%
\definecolor{currentstroke}{rgb}{0.000000,0.000000,0.000000}%
\pgfsetstrokecolor{currentstroke}%
\pgfsetdash{}{0pt}%
\pgfsys@defobject{currentmarker}{\pgfqpoint{-0.055556in}{0.000000in}}{\pgfqpoint{0.000000in}{0.000000in}}{%
\pgfpathmoveto{\pgfqpoint{0.000000in}{0.000000in}}%
\pgfpathlineto{\pgfqpoint{-0.055556in}{0.000000in}}%
\pgfusepath{stroke,fill}%
}%
\begin{pgfscope}%
\pgfsys@transformshift{4.687795in}{1.466492in}%
\pgfsys@useobject{currentmarker}{}%
\end{pgfscope}%
\end{pgfscope}%
\begin{pgfscope}%
\pgftext[x=0.595527in,y=1.466492in,right,]{{\rmfamily\fontsize{8.000000}{9.600000}\selectfont \(\displaystyle 4\)}}%
\end{pgfscope}%
\begin{pgfscope}%
\pgfsetbuttcap%
\pgfsetroundjoin%
\definecolor{currentfill}{rgb}{0.000000,0.000000,0.000000}%
\pgfsetfillcolor{currentfill}%
\pgfsetlinewidth{0.501875pt}%
\definecolor{currentstroke}{rgb}{0.000000,0.000000,0.000000}%
\pgfsetstrokecolor{currentstroke}%
\pgfsetdash{}{0pt}%
\pgfsys@defobject{currentmarker}{\pgfqpoint{0.000000in}{0.000000in}}{\pgfqpoint{0.055556in}{0.000000in}}{%
\pgfpathmoveto{\pgfqpoint{0.000000in}{0.000000in}}%
\pgfpathlineto{\pgfqpoint{0.055556in}{0.000000in}}%
\pgfusepath{stroke,fill}%
}%
\begin{pgfscope}%
\pgfsys@transformshift{0.651083in}{1.752637in}%
\pgfsys@useobject{currentmarker}{}%
\end{pgfscope}%
\end{pgfscope}%
\begin{pgfscope}%
\pgfsetbuttcap%
\pgfsetroundjoin%
\definecolor{currentfill}{rgb}{0.000000,0.000000,0.000000}%
\pgfsetfillcolor{currentfill}%
\pgfsetlinewidth{0.501875pt}%
\definecolor{currentstroke}{rgb}{0.000000,0.000000,0.000000}%
\pgfsetstrokecolor{currentstroke}%
\pgfsetdash{}{0pt}%
\pgfsys@defobject{currentmarker}{\pgfqpoint{-0.055556in}{0.000000in}}{\pgfqpoint{0.000000in}{0.000000in}}{%
\pgfpathmoveto{\pgfqpoint{0.000000in}{0.000000in}}%
\pgfpathlineto{\pgfqpoint{-0.055556in}{0.000000in}}%
\pgfusepath{stroke,fill}%
}%
\begin{pgfscope}%
\pgfsys@transformshift{4.687795in}{1.752637in}%
\pgfsys@useobject{currentmarker}{}%
\end{pgfscope}%
\end{pgfscope}%
\begin{pgfscope}%
\pgftext[x=0.595527in,y=1.752637in,right,]{{\rmfamily\fontsize{8.000000}{9.600000}\selectfont \(\displaystyle 5\)}}%
\end{pgfscope}%
\begin{pgfscope}%
\pgfsetbuttcap%
\pgfsetroundjoin%
\definecolor{currentfill}{rgb}{0.000000,0.000000,0.000000}%
\pgfsetfillcolor{currentfill}%
\pgfsetlinewidth{0.501875pt}%
\definecolor{currentstroke}{rgb}{0.000000,0.000000,0.000000}%
\pgfsetstrokecolor{currentstroke}%
\pgfsetdash{}{0pt}%
\pgfsys@defobject{currentmarker}{\pgfqpoint{0.000000in}{0.000000in}}{\pgfqpoint{0.055556in}{0.000000in}}{%
\pgfpathmoveto{\pgfqpoint{0.000000in}{0.000000in}}%
\pgfpathlineto{\pgfqpoint{0.055556in}{0.000000in}}%
\pgfusepath{stroke,fill}%
}%
\begin{pgfscope}%
\pgfsys@transformshift{0.651083in}{2.038782in}%
\pgfsys@useobject{currentmarker}{}%
\end{pgfscope}%
\end{pgfscope}%
\begin{pgfscope}%
\pgfsetbuttcap%
\pgfsetroundjoin%
\definecolor{currentfill}{rgb}{0.000000,0.000000,0.000000}%
\pgfsetfillcolor{currentfill}%
\pgfsetlinewidth{0.501875pt}%
\definecolor{currentstroke}{rgb}{0.000000,0.000000,0.000000}%
\pgfsetstrokecolor{currentstroke}%
\pgfsetdash{}{0pt}%
\pgfsys@defobject{currentmarker}{\pgfqpoint{-0.055556in}{0.000000in}}{\pgfqpoint{0.000000in}{0.000000in}}{%
\pgfpathmoveto{\pgfqpoint{0.000000in}{0.000000in}}%
\pgfpathlineto{\pgfqpoint{-0.055556in}{0.000000in}}%
\pgfusepath{stroke,fill}%
}%
\begin{pgfscope}%
\pgfsys@transformshift{4.687795in}{2.038782in}%
\pgfsys@useobject{currentmarker}{}%
\end{pgfscope}%
\end{pgfscope}%
\begin{pgfscope}%
\pgftext[x=0.595527in,y=2.038782in,right,]{{\rmfamily\fontsize{8.000000}{9.600000}\selectfont \(\displaystyle 6\)}}%
\end{pgfscope}%
\begin{pgfscope}%
\pgfsetbuttcap%
\pgfsetroundjoin%
\definecolor{currentfill}{rgb}{0.000000,0.000000,0.000000}%
\pgfsetfillcolor{currentfill}%
\pgfsetlinewidth{0.501875pt}%
\definecolor{currentstroke}{rgb}{0.000000,0.000000,0.000000}%
\pgfsetstrokecolor{currentstroke}%
\pgfsetdash{}{0pt}%
\pgfsys@defobject{currentmarker}{\pgfqpoint{0.000000in}{0.000000in}}{\pgfqpoint{0.055556in}{0.000000in}}{%
\pgfpathmoveto{\pgfqpoint{0.000000in}{0.000000in}}%
\pgfpathlineto{\pgfqpoint{0.055556in}{0.000000in}}%
\pgfusepath{stroke,fill}%
}%
\begin{pgfscope}%
\pgfsys@transformshift{0.651083in}{2.324927in}%
\pgfsys@useobject{currentmarker}{}%
\end{pgfscope}%
\end{pgfscope}%
\begin{pgfscope}%
\pgfsetbuttcap%
\pgfsetroundjoin%
\definecolor{currentfill}{rgb}{0.000000,0.000000,0.000000}%
\pgfsetfillcolor{currentfill}%
\pgfsetlinewidth{0.501875pt}%
\definecolor{currentstroke}{rgb}{0.000000,0.000000,0.000000}%
\pgfsetstrokecolor{currentstroke}%
\pgfsetdash{}{0pt}%
\pgfsys@defobject{currentmarker}{\pgfqpoint{-0.055556in}{0.000000in}}{\pgfqpoint{0.000000in}{0.000000in}}{%
\pgfpathmoveto{\pgfqpoint{0.000000in}{0.000000in}}%
\pgfpathlineto{\pgfqpoint{-0.055556in}{0.000000in}}%
\pgfusepath{stroke,fill}%
}%
\begin{pgfscope}%
\pgfsys@transformshift{4.687795in}{2.324927in}%
\pgfsys@useobject{currentmarker}{}%
\end{pgfscope}%
\end{pgfscope}%
\begin{pgfscope}%
\pgftext[x=0.595527in,y=2.324927in,right,]{{\rmfamily\fontsize{8.000000}{9.600000}\selectfont \(\displaystyle 7\)}}%
\end{pgfscope}%
\begin{pgfscope}%
\pgfsetbuttcap%
\pgfsetroundjoin%
\definecolor{currentfill}{rgb}{0.000000,0.000000,0.000000}%
\pgfsetfillcolor{currentfill}%
\pgfsetlinewidth{0.501875pt}%
\definecolor{currentstroke}{rgb}{0.000000,0.000000,0.000000}%
\pgfsetstrokecolor{currentstroke}%
\pgfsetdash{}{0pt}%
\pgfsys@defobject{currentmarker}{\pgfqpoint{0.000000in}{0.000000in}}{\pgfqpoint{0.055556in}{0.000000in}}{%
\pgfpathmoveto{\pgfqpoint{0.000000in}{0.000000in}}%
\pgfpathlineto{\pgfqpoint{0.055556in}{0.000000in}}%
\pgfusepath{stroke,fill}%
}%
\begin{pgfscope}%
\pgfsys@transformshift{0.651083in}{2.611072in}%
\pgfsys@useobject{currentmarker}{}%
\end{pgfscope}%
\end{pgfscope}%
\begin{pgfscope}%
\pgfsetbuttcap%
\pgfsetroundjoin%
\definecolor{currentfill}{rgb}{0.000000,0.000000,0.000000}%
\pgfsetfillcolor{currentfill}%
\pgfsetlinewidth{0.501875pt}%
\definecolor{currentstroke}{rgb}{0.000000,0.000000,0.000000}%
\pgfsetstrokecolor{currentstroke}%
\pgfsetdash{}{0pt}%
\pgfsys@defobject{currentmarker}{\pgfqpoint{-0.055556in}{0.000000in}}{\pgfqpoint{0.000000in}{0.000000in}}{%
\pgfpathmoveto{\pgfqpoint{0.000000in}{0.000000in}}%
\pgfpathlineto{\pgfqpoint{-0.055556in}{0.000000in}}%
\pgfusepath{stroke,fill}%
}%
\begin{pgfscope}%
\pgfsys@transformshift{4.687795in}{2.611072in}%
\pgfsys@useobject{currentmarker}{}%
\end{pgfscope}%
\end{pgfscope}%
\begin{pgfscope}%
\pgftext[x=0.595527in,y=2.611072in,right,]{{\rmfamily\fontsize{8.000000}{9.600000}\selectfont \(\displaystyle 8\)}}%
\end{pgfscope}%
\begin{pgfscope}%
\pgfsetbuttcap%
\pgfsetroundjoin%
\definecolor{currentfill}{rgb}{0.000000,0.000000,0.000000}%
\pgfsetfillcolor{currentfill}%
\pgfsetlinewidth{0.501875pt}%
\definecolor{currentstroke}{rgb}{0.000000,0.000000,0.000000}%
\pgfsetstrokecolor{currentstroke}%
\pgfsetdash{}{0pt}%
\pgfsys@defobject{currentmarker}{\pgfqpoint{0.000000in}{0.000000in}}{\pgfqpoint{0.055556in}{0.000000in}}{%
\pgfpathmoveto{\pgfqpoint{0.000000in}{0.000000in}}%
\pgfpathlineto{\pgfqpoint{0.055556in}{0.000000in}}%
\pgfusepath{stroke,fill}%
}%
\begin{pgfscope}%
\pgfsys@transformshift{0.651083in}{2.897217in}%
\pgfsys@useobject{currentmarker}{}%
\end{pgfscope}%
\end{pgfscope}%
\begin{pgfscope}%
\pgfsetbuttcap%
\pgfsetroundjoin%
\definecolor{currentfill}{rgb}{0.000000,0.000000,0.000000}%
\pgfsetfillcolor{currentfill}%
\pgfsetlinewidth{0.501875pt}%
\definecolor{currentstroke}{rgb}{0.000000,0.000000,0.000000}%
\pgfsetstrokecolor{currentstroke}%
\pgfsetdash{}{0pt}%
\pgfsys@defobject{currentmarker}{\pgfqpoint{-0.055556in}{0.000000in}}{\pgfqpoint{0.000000in}{0.000000in}}{%
\pgfpathmoveto{\pgfqpoint{0.000000in}{0.000000in}}%
\pgfpathlineto{\pgfqpoint{-0.055556in}{0.000000in}}%
\pgfusepath{stroke,fill}%
}%
\begin{pgfscope}%
\pgfsys@transformshift{4.687795in}{2.897217in}%
\pgfsys@useobject{currentmarker}{}%
\end{pgfscope}%
\end{pgfscope}%
\begin{pgfscope}%
\pgftext[x=0.595527in,y=2.897217in,right,]{{\rmfamily\fontsize{8.000000}{9.600000}\selectfont \(\displaystyle 9\)}}%
\end{pgfscope}%
\begin{pgfscope}%
\pgftext[x=0.467054in,y=1.609565in,,bottom,rotate=90.000000]{{\rmfamily\fontsize{10.000000}{12.000000}\selectfont \(\displaystyle  h_j(\delta) = \sum_{r=0}^{n-1} | u_{jr}(\delta) | \)}}%
\end{pgfscope}%
\begin{pgfscope}%
\pgfsetbuttcap%
\pgfsetroundjoin%
\pgfsetlinewidth{1.003750pt}%
\definecolor{currentstroke}{rgb}{0.000000,0.000000,0.000000}%
\pgfsetstrokecolor{currentstroke}%
\pgfsetdash{}{0pt}%
\pgfpathmoveto{\pgfqpoint{0.651083in}{2.897217in}}%
\pgfpathlineto{\pgfqpoint{4.687795in}{2.897217in}}%
\pgfusepath{stroke}%
\end{pgfscope}%
\begin{pgfscope}%
\pgfsetbuttcap%
\pgfsetroundjoin%
\pgfsetlinewidth{1.003750pt}%
\definecolor{currentstroke}{rgb}{0.000000,0.000000,0.000000}%
\pgfsetstrokecolor{currentstroke}%
\pgfsetdash{}{0pt}%
\pgfpathmoveto{\pgfqpoint{4.687795in}{0.321913in}}%
\pgfpathlineto{\pgfqpoint{4.687795in}{2.897217in}}%
\pgfusepath{stroke}%
\end{pgfscope}%
\begin{pgfscope}%
\pgfsetbuttcap%
\pgfsetroundjoin%
\pgfsetlinewidth{1.003750pt}%
\definecolor{currentstroke}{rgb}{0.000000,0.000000,0.000000}%
\pgfsetstrokecolor{currentstroke}%
\pgfsetdash{}{0pt}%
\pgfpathmoveto{\pgfqpoint{0.651083in}{0.321913in}}%
\pgfpathlineto{\pgfqpoint{4.687795in}{0.321913in}}%
\pgfusepath{stroke}%
\end{pgfscope}%
\begin{pgfscope}%
\pgfsetbuttcap%
\pgfsetroundjoin%
\pgfsetlinewidth{1.003750pt}%
\definecolor{currentstroke}{rgb}{0.000000,0.000000,0.000000}%
\pgfsetstrokecolor{currentstroke}%
\pgfsetdash{}{0pt}%
\pgfpathmoveto{\pgfqpoint{0.651083in}{0.321913in}}%
\pgfpathlineto{\pgfqpoint{0.651083in}{2.897217in}}%
\pgfusepath{stroke}%
\end{pgfscope}%
\begin{pgfscope}%
\pgfsetbuttcap%
\pgfsetroundjoin%
\definecolor{currentfill}{rgb}{0.300000,0.300000,0.300000}%
\pgfsetfillcolor{currentfill}%
\pgfsetfillopacity{0.500000}%
\pgfsetlinewidth{1.003750pt}%
\definecolor{currentstroke}{rgb}{0.300000,0.300000,0.300000}%
\pgfsetstrokecolor{currentstroke}%
\pgfsetstrokeopacity{0.500000}%
\pgfsetdash{}{0pt}%
\pgfpathmoveto{\pgfqpoint{1.705792in}{2.341919in}}%
\pgfpathlineto{\pgfqpoint{3.688641in}{2.341919in}}%
\pgfpathquadraticcurveto{\pgfqpoint{3.710864in}{2.341919in}}{\pgfqpoint{3.710864in}{2.364141in}}%
\pgfpathlineto{\pgfqpoint{3.710864in}{2.662896in}}%
\pgfpathquadraticcurveto{\pgfqpoint{3.710864in}{2.685118in}}{\pgfqpoint{3.688641in}{2.685118in}}%
\pgfpathlineto{\pgfqpoint{1.705792in}{2.685118in}}%
\pgfpathquadraticcurveto{\pgfqpoint{1.683570in}{2.685118in}}{\pgfqpoint{1.683570in}{2.662896in}}%
\pgfpathlineto{\pgfqpoint{1.683570in}{2.364141in}}%
\pgfpathquadraticcurveto{\pgfqpoint{1.683570in}{2.341919in}}{\pgfqpoint{1.705792in}{2.341919in}}%
\pgfpathclose%
\pgfusepath{stroke,fill}%
\end{pgfscope}%
\begin{pgfscope}%
\pgfsetbuttcap%
\pgfsetroundjoin%
\definecolor{currentfill}{rgb}{1.000000,1.000000,1.000000}%
\pgfsetfillcolor{currentfill}%
\pgfsetlinewidth{1.003750pt}%
\definecolor{currentstroke}{rgb}{0.000000,0.000000,0.000000}%
\pgfsetstrokecolor{currentstroke}%
\pgfsetdash{}{0pt}%
\pgfpathmoveto{\pgfqpoint{1.678014in}{2.369697in}}%
\pgfpathlineto{\pgfqpoint{3.660864in}{2.369697in}}%
\pgfpathquadraticcurveto{\pgfqpoint{3.683086in}{2.369697in}}{\pgfqpoint{3.683086in}{2.391919in}}%
\pgfpathlineto{\pgfqpoint{3.683086in}{2.690674in}}%
\pgfpathquadraticcurveto{\pgfqpoint{3.683086in}{2.712896in}}{\pgfqpoint{3.660864in}{2.712896in}}%
\pgfpathlineto{\pgfqpoint{1.678014in}{2.712896in}}%
\pgfpathquadraticcurveto{\pgfqpoint{1.655792in}{2.712896in}}{\pgfqpoint{1.655792in}{2.690674in}}%
\pgfpathlineto{\pgfqpoint{1.655792in}{2.391919in}}%
\pgfpathquadraticcurveto{\pgfqpoint{1.655792in}{2.369697in}}{\pgfqpoint{1.678014in}{2.369697in}}%
\pgfpathclose%
\pgfusepath{stroke,fill}%
\end{pgfscope}%
\begin{pgfscope}%
\pgfsetrectcap%
\pgfsetroundjoin%
\pgfsetlinewidth{1.003750pt}%
\definecolor{currentstroke}{rgb}{0.000000,0.000000,0.000000}%
\pgfsetstrokecolor{currentstroke}%
\pgfsetdash{}{0pt}%
\pgfpathmoveto{\pgfqpoint{1.733570in}{2.629563in}}%
\pgfpathlineto{\pgfqpoint{1.889125in}{2.629563in}}%
\pgfusepath{stroke}%
\end{pgfscope}%
\begin{pgfscope}%
\pgftext[x=2.011348in,y=2.590674in,left,base]{{\rmfamily\fontsize{8.000000}{9.600000}\selectfont j=0}}%
\end{pgfscope}%
\begin{pgfscope}%
\pgfsetrectcap%
\pgfsetroundjoin%
\pgfsetlinewidth{1.003750pt}%
\definecolor{currentstroke}{rgb}{0.000000,0.000000,1.000000}%
\pgfsetstrokecolor{currentstroke}%
\pgfsetdash{}{0pt}%
\pgfpathmoveto{\pgfqpoint{1.733570in}{2.474630in}}%
\pgfpathlineto{\pgfqpoint{1.889125in}{2.474630in}}%
\pgfusepath{stroke}%
\end{pgfscope}%
\begin{pgfscope}%
\pgftext[x=2.011348in,y=2.435741in,left,base]{{\rmfamily\fontsize{8.000000}{9.600000}\selectfont j=1}}%
\end{pgfscope}%
\begin{pgfscope}%
\pgfsetrectcap%
\pgfsetroundjoin%
\pgfsetlinewidth{1.003750pt}%
\definecolor{currentstroke}{rgb}{0.000000,0.500000,0.000000}%
\pgfsetstrokecolor{currentstroke}%
\pgfsetdash{}{0pt}%
\pgfpathmoveto{\pgfqpoint{2.453779in}{2.629563in}}%
\pgfpathlineto{\pgfqpoint{2.609334in}{2.629563in}}%
\pgfusepath{stroke}%
\end{pgfscope}%
\begin{pgfscope}%
\pgftext[x=2.731557in,y=2.590674in,left,base]{{\rmfamily\fontsize{8.000000}{9.600000}\selectfont j=2}}%
\end{pgfscope}%
\begin{pgfscope}%
\pgfsetrectcap%
\pgfsetroundjoin%
\pgfsetlinewidth{1.003750pt}%
\definecolor{currentstroke}{rgb}{1.000000,0.000000,0.000000}%
\pgfsetstrokecolor{currentstroke}%
\pgfsetdash{}{0pt}%
\pgfpathmoveto{\pgfqpoint{2.453779in}{2.474630in}}%
\pgfpathlineto{\pgfqpoint{2.609334in}{2.474630in}}%
\pgfusepath{stroke}%
\end{pgfscope}%
\begin{pgfscope}%
\pgftext[x=2.731557in,y=2.435741in,left,base]{{\rmfamily\fontsize{8.000000}{9.600000}\selectfont j=3}}%
\end{pgfscope}%
\begin{pgfscope}%
\pgfsetrectcap%
\pgfsetroundjoin%
\pgfsetlinewidth{1.003750pt}%
\definecolor{currentstroke}{rgb}{0.000000,0.750000,0.750000}%
\pgfsetstrokecolor{currentstroke}%
\pgfsetdash{}{0pt}%
\pgfpathmoveto{\pgfqpoint{3.173988in}{2.629563in}}%
\pgfpathlineto{\pgfqpoint{3.329543in}{2.629563in}}%
\pgfusepath{stroke}%
\end{pgfscope}%
\begin{pgfscope}%
\pgftext[x=3.451766in,y=2.590674in,left,base]{{\rmfamily\fontsize{8.000000}{9.600000}\selectfont j=4}}%
\end{pgfscope}%
\end{pgfpicture}%
\makeatother%
\endgroup%

        %\caption{$ h_j(\delta) = \sum_{r=0}^{n-1} \abs{ u_{jr}(\delta) } $}
        \caption{Betragssumme der $j$-ten Zeile von $\Vand{\delta}^{-1}$
        in Abhängigkeit der Auslenkung $\delta \in (-0.9, 0.9)$
        für den Fall $n=5$.}
        \label{fig:row_j}
    \end{figure}

    \noindent Der Beweis kann also nur erbracht werden, indem das gesamte Produkt
    abgeschätzt wird:
    \[
        \Pi_j \sum_{r=0}^{n-1} \abs{\sigma_r^j} \leq \Pi_0 \sum_{r=0}^{n-1} \abs{\sigma_r^0} \text{ für } j=1, \dots, n-1.
    \]


    \noindent In Tabelle \ref{tab:relative_error_assumption} sind für ausgewählte
    $n \in \N$ die numerischen Fehler angegeben, welche durch Verwendung der
    Formel aus der Vermutung entstehen.
    Für $\delta \in \{ 0, 0.01, \dots, 0.99 \}$ wurde dabei jeweils die
    Kondition der Vandermonde-Matrix $\Vand{\delta}$ direkt berechnet und die
    Differenz mit der Formel aus Vermutung
    \ref{assumption:infty_outlier_condition} gebildet.
    Die Beträge dieser Differenzen wurden für alle $\delta$ aufaddiert und
    durch $n$ dividiert.
    Die resultierenden Werte sind in Tabelle
    \ref{tab:relative_error_assumption} eingetragen.
    Die entstandenen Fehler sind in der Größenordnung numerischer
    Rechenungeauigkeiten bei der Verwendung von Gleitkommazahlen einzuordnen
    und stützen damit Gleichung \eqref{eq:infty_condition_assumption}.
    \begin{table}[h]
        \centering
        \begin{tabular}[c]{l|l}
            $ n $ & Relativer Fehler \\
            \hline
            $ 2  $ & $ 3.499422973618493 \cdot 10^{-13} $ \\
            $ 5  $ & $ 7.613465413669474 \cdot 10^{-13} $ \\
            $ 10 $ & $ 1.033839680530946 \cdot 10^{-12} $ \\
            $ 20 $ & $ 2.271782761908981 \cdot 10^{-12} $ \\
            $ 30 $ & $ 2.729076224265251 \cdot 10^{-12} $ \\
            $ 40 $ & $ 3.564437633940542 \cdot 10^{-12} $ \\
            $ 50 $ & $ 3.976907692049281 \cdot 10^{-12} $
        \end{tabular}
        \caption{Relativer Fehler bei der Berechnung der Konditionszahlen
                $\cond[\infty]{\Vand{\delta}}$ mit Hilfe von Gleichung
                \eqref{eq:infty_condition_assumption} für verschiedene $n\in\N$.}
        \label{tab:relative_error_assumption}
    \end{table}

\end{remark}

\subsection{Kondition bezüglich der Frobeniusnorm}
% FIXME: Next paragraph is crap.
Wir wollen nun eine explizite Formel für die Kondition der Vandermonde-Matrix
$\Vand{\delta}$ bezüglich der Frobeniusnorm beweisen.
Dazu sei an die Definition der Frobeniusnorm (\ref{eq:frobenius_norm}) erinnert.
Bereits in Lemma \ref{lemma:frobenius_norm_vandermonde_unit_circle} wurde
gezeigt, dass die Frobeniusnorm einer Vandermonde-Matrix mit $n$ Knoten auf dem
Einheitskreis $n$ beträgt.
Schwierigkeiten kann also nur die Norm der inversen Vandermonde-Matrix
bereiten.  Wir betrachten im Folgenden die zwei Fälle $j=0$ und
$j \in \{1, \dots, n-1\}$ gesondert.

\begin{lemma}
    \label{lemma:inverse_outlier_zero_row_quadratic_sum}
    Bezeichnen wir mit $u_{jr} \in \C$ für $j, r = 0, \dots, n-1$ die Elemente
    der inversen Vandermonde-Matrix $V(\delta)^{-1}$, so gilt
    \[
        \sum_{r=0}^{n-1} \abs{u_{0r}}^2
        = n \cdot \prod_{k=1}^{n-1} \frac{1}{4 \cdot \sin^2 \left( \pi \frac{k-\delta}{n} \right)}.
    \]
\end{lemma}
\begin{proof}
    Mit Lemma \ref{lemma:outlier_sigma_zero} und Gleichung (\ref{eq:pi_0}) aus
    dem Beiweis von \lemmaref{inverse_outlier_vandermonde_first_row_abs_sum}
    erhalten wir
    \[
        \begin{split}
            \sum_{r=0}^{n-1} \abs{u_{0r}}^2
            &= \sum_{r=0}^{n-1} \abs{\Pi_0}^2 \abs{\sigma_{n-1-r}^{0}(z_1, \dots, z_{n-1})}^2\\
            &\overset{(\ref{eq:outlier_sigma_zero})}=
                \abs{\Pi_0}^2 \sum_{r=0}^{n-1} 1^2
            = n \cdot \abs{\Pi_0}^2\\
            &\overset{(\ref{eq:pi_0})}=
                n \cdot \prod_{k=1}^{n-1} \frac{1}{4 \cdot \sin^2 \left( \pi \frac{k-\delta}{n} \right)}.
        \end{split}
    \]
\end{proof}

\noindent Als Vorbereitung für den Fall $j \in \{1, \dots, n-1\}$ benötigen wir noch zwei
Hilfslemmata.
\begin{lemma}
    \label{lemma:uptau_j}
    Sei $\w_n \defeq e^{\frac{2 \pi i}{n}}$.
    Definiere $\uptau_j \in \C$ für $j=0, \dots, n-1$ durch
    \[
        \uptau_j \defeq \prod_{\substack{k=0\\k\neq j}}^{n-1} \left( \w_n^j - \w_n^k\right).
    \]
    Dann gilt $\abs{\uptau_j} = n$ für alle $j = 0, \dots, n-1$.
\end{lemma}
\begin{proof}
    Wir zeigen $\abs{\uptau_j} = \abs{\uptau_0}$ für alle $j = 1, \dots, n-1$ und
    $\uptau_0 = n$.
    Es gilt
    \[
        \begin{split}
            \abs{\uptau_j}
            &= \abs{\uptau_j \cdot \w_n^{-j}}
            = \prod_{\substack{k=0\\ k\neq j}}^{n-1} \abs{\w_n^0 - \w_n^{k-j} }\\
            &\overset{k' = k-j}{=}
            \prod_{\substack{k'=-j\\ k\neq 0}}^{n-1-j} \abs{1 - \wn^{k'}}
            = \prod_{\substack{k'=0\\ k\neq 0}}^{n-1} \abs{1 - \wn^{k'}}
            = \abs{\uptau_0},
        \end{split}
    \]
    wobei wir im letzten Schritt nur die Reihenfolge der Faktoren verändern und
    die $2\pi$-Periodizität von $e^{i \varphi}$ ausnutzen.

    \noindent Mit $p(z) \defeq \prod_{k=1}^{n-1} (z - \wn^k)$ können wir $\uptau_0 = p(1)$
    schreiben.

    Bereits im Beweis von \lemmaref{outlier_sigma_zero} haben wir gesehen, dass
    \[
        p(z) = \prod_{k=1}^{n-1} (z-\w_n^k) = \sum_{k=0}^{n-1} z^k
    \]
    gilt.
    Damit folgt sofort
    \[
        \uptau_0 = p(1) = \sum_{k=0}^{n-1} 1^k = n.
    \]
\end{proof}

\begin{lemma}
    \label{lemma:sum_shifted_unit_roots}
    Für $j \in \{1, \dots, n-1\}$ gilt
    \[
        \sum_{r=0}^{n-1} \cos \left( \frac{2 \pi j r}{n} - \varphi \right) = 0.
    \]
\end{lemma}
\begin{proof}
    Mit $p(z) \defeq \sum_{k=0}^{n-1} z^n = \prod_{j=1}^{n-1} (z - \wn^j)$ gilt
    \[
        \begin{split}
            \sum_{r=0}^{n-1} \cos \left( \frac{2 \pi j r}{n} - \varphi \right)
            &= \sum_{r=0}^{n-1} \Re\left( e^{\frac{2 \pi i j r}{n} - i \varphi} \right)
            = \Re \left( \sum_{r=0}^{n-1} e^{\frac{2 \pi i j r}{n} - i \varphi} \right)\\
            &= \Re \left( e^{-i \varphi}\sum_{r=0}^{n-1} \left( \wn^j \right)^r \right)
            = \Re \left( e^{-i \varphi} p\left( \wn^j \right) \right)
            = 0,
        \end{split}
    \]
    wobei $\Re(z)$ den Realteil von $z$ bezeichnet.
\end{proof}

\begin{lemma}
    \label{lemma:inverse_outlier_nonzero_row_quadratic_sum}
    Sei $z(\delta) = (z_0(\delta), z_1, \dots, z_{n-1}) \in \Cn$ wie zuvor.
    Erneut bezeichnen wir mit $u_{jr}$ die Elemente der inversen
    Vandermonde-Matrix $\Vand{\delta}^{-1}$.
    Dann gilt für ${j = 1, \dots, n-1}$
    \begin{equation}
        \sum_{r=0}^{n-1} \abs{u_{jr}}^2
        = \frac{\rho(\delta)^2 + 1}{n}
    \end{equation}
    mit
    \[
        \rho(\delta) \defeq \frac{\sin \left( \pi \frac{\delta}{n} \right)}{\sin \left( \pi \frac{j - \delta}{n} \right)}.
    \]
\end{lemma}
\begin{proof}
    In diesem Beweis werden wir der Übersicht wegen meist auf die explizite
    Erwähnung der Abhängigkeit von $\delta$ verzichten und nur kurz
    $z_0 = z_0(\delta)$ schreiben.

    \noindent Sei $j \in \{1, \dots, n-1\}$ fixiert.
    Wir erinnern uns zunächst daran, dass mit Lemma
    \ref{lemma:elementary_symmetric_polynomials_const_multiplication}
    für ${r = 0, \dots, n-1}$
    \[
        \abs{ \sigma_r^j (z)}
        = \abs{(-1)^r \sigma_r^j (-z) }
        = \abs{\sigma_r^j (-z) }
    \]
    gilt.
    Weiterhin ist nach Definition $\sigma_r^j (-z)$ der $n\!-\!j\!-\!1$-te
    Koeffizient des Polynoms
    \[
        \begin{split}
            p(z)
            &= \sum_{k=0}^{n-1} a_k z^k
            \defeq \prod_{\substack{k=0 \\ k\neq j}}^{n-1} (z-z_k)
            = \left( \prod_{k=1}^{n-1} (z-z_k) \right) \cdot \frac{z-z_0(\delta)}{z-z_j}.
        \end{split}
    \]
    Im Beweis von Lemma
    \ref{lemma:outlier_sigma_zero}
    haben wir bereits die Gleichheit
    \[
        \prod_{k=1}^{n-1} (z-z_k) = \sum_{k=0}^{n-1} z^k
    \]
    gezeigt, so dass wir damit
    \[
        p(z)
        = \sum_{k=0}^{n-1} a_k z^k
        = \sum_{k=0}^{n-1} z^k \cdot \frac{z-z_0(\delta)}{z-z_j}
    \]
    erhalten.
    Multiplikation der Gleichung mit $(z-z_j)$ liefert die äquivalente
    Darstellung
    \[
        (z-z_j) \cdot \sum_{k=0}^{n-1} a_k z^k
        = (z-z_0(\delta)) \cdot \sum_{k=0}^{n-1} z^k
    \]
    oder ausmultipliziert
    \[
        \sum_{k=0}^{n-1} \left( a_k z^{k+1} - a_k z_j z^k \right)
        = \sum_{k=0}^{n-1} \left( z^{k+1} - z_0 z^k \right).
    \]

    \noindent Durch Vergleich der Koeffizienten erhalten wir die $n+1$
    Gleichungen
    \begin{equation*}
        \begin{split}
            -a_0 z_j              &= -z_0\\
            a_0 - a_1 z_j         &= 1 - z_0\\
            a_1 - a_2 z_j         &= 1 - z_0\\
                                  &\;\; \vdots\\
            a_{n-2} - a_{n-1} z_j &= 1 - z_0\\
            a_{n-1}               &= 1
        \end{split}
    \end{equation*}
    und in etwas kompakterer Form
    \[
        \begin{split}
            -a_0 z_j            &= -z_0, \\
            a_{k-1} - a_{k} z_j &= 1 - z_0 \text{ für } k \in \{1, \dots, n-1\},\\
            a_{n-1}             &= 1.
        \end{split}
    \]
    Es zeigt sich nun, dass für $k=1, \dots, n$
    \begin{equation}
        \label{eq:elementy_symmetric_polynomials_nonzero_row}
        a_{n-k}
        = (1 - z_0) \left( \sum_{r=0}^{k-2} z_j^r \right) + z_j^{k-1}
    \end{equation}
    gilt.
    Für $k=1$ ist dies offensichtlich wahr.
    Ist die Gleichung für ein $k \in \{1, \dots, n\}$ bereits gezeigt, so folgt
    mit Hilfe der Gleichung
    $a_{n-(k+1)} - a_{n-k} z_j = 1 - z_0$, dass unsere Behauptung tatsächlich
    auch für $a_{n-(k+1)}$ gilt:
    \[
        \begin{split}
            a_{n-k-1}
            &= 1-z_0 + z_j a_{n-k}\\
            &= (1-z_0) + z_j \left( (1 - z_0) \left( \sum_{r=0}^{k-2} z_j^r \right) + z_j^{k-1} \right)\\
            &= (1-z_0) + (1 - z_0) \left( \sum_{r=0}^{k-2} z_j^{r+1} \right) + z_j^{k}\\
            &= (1 - z_0) \left( \sum_{r=0}^{k-1} z_j^r \right) + z_j^{k}.
        \end{split}
    \]
    Induktion nach $k = 1, \dots, n$ liefert also die Richtigkeit der Gleichung
    \eqref{eq:elementy_symmetric_polynomials_nonzero_row}.
    Wir identifizieren nun die elementarsymmetrischen Polynome gemäß ihrer
    Definition durch
    ${\sigma_r^j(-z) = a_{n-(r+1)}}$ für $r=0, \dots, n-1$ und
    formen unter Verwendung der geometrischen Reihe weiter um:
    \[
        \begin{split}
            \sigma_r^j(-z)
            &= (1 - z_0) \left( \sum_{l=0}^{r-1} z_j^l \right) + z_j^{r}\\
            &= (1 - z_0) \frac{1-z_j^r}{1-z_j} + z_j^{r}
            = \frac{(1-z_0)(1-z_j^r) + (1-z_j) z_j^r}{1-z_j}\\
            &= \frac{1 + z_0 z_j^r - z_0 - z_j^{r+1}}{1-z_j}
            = \frac{(1 - z_0) + (z_0 - z_j) z_j^r}{1-z_j}.
        \end{split}
    \]

    \noindent Damit gilt für alle $r=0, \dots, n-1$
    \[
        \begin{split}
            u_{j,n-1-r}
            &\overset{(\ref{eq:explicit_inverse_vandermonde})}=
                \Pi_j \cdot \sigma_{r}^j(-z) \\
            &= \frac{(1 - z_0) + (z_0 - z_j) z_j^r}{(1-z_j) \cdot \Pi_j}\\
            &= \frac{(1 - z_0) + (z_0 - z_j) z_j^r}{(z_j-z_0) \cdot \left((z_j - 1) (z_j - z_1) \dots (z_j - z_{n-1}) \right)}\\
            &\eqdef \frac{(1 - z_0) + (z_0 - z_j) z_j^r}{(z_j-z_0) \cdot \uptau_j }\\
            &= \frac{1}{\uptau_j} \left( \frac{1 - z_0}{z_j-z_0} - z_j^r \right),
        \end{split}
    \]
    mit
    \[
        \uptau_j \defeq \prod_{\substack{k=0\\k\neq j}}^{n-1} \left( \exp\left(\frac{2 \pi i j}{n}\right) - \exp\left(\frac{2 \pi i k}{n}\right) \right).
    \]

    \noindent Wir schreiben nun $ \rho e^{i\varphi} \defeq \frac{1-z_0}{z_j-z_0} $
    mit $\rho, \varphi \in \R$, $\rho \geq 0$ und erhalten zusammen mit $\abs{\uptau_j} = n$ aus \lemmaref{uptau_j}
    \[
        \begin{split}
            \abs{u_{j,n-1-r}}^2
            &= \frac{1}{n^2} \abs{ \rho e^{i\varphi} - z_j^r }^2
            = \frac{1}{n^2} \left( \rho^2 + 1 - \rho \left( e^{\varphi - \frac{2\pi j r}{n}} + e^{\frac{2 \pi j r } {n} - \varphi} \right) \right)\\
            &= \frac{1}{n^2} \left( \rho^2 + 1 - 2 \rho \cos \left( \frac{2 \pi j r}{n} - \varphi \right) \right).
        \end{split}
    \]

    \noindent Mit Hilfe von \lemmaref{sum_shifted_unit_roots} folgt insgesamt
    für die $j$-te Zeile
    \[
        \begin{split}
            \sum_{r=0}^{n-1} \abs{u_{j,n-1-r}}^2
            &= \sum_{r=0}^{n-1} \frac{1}{n^2} \left( \rho^2 + 1 - 2 \rho \cos \left( \frac{2 \pi j r}{n} - \varphi \right) \right)\\
            &= \frac{\rho^2 + 1}{n} - \frac{2 \rho}{n^2} \sum_{r=0}^{n-1} \cos \left( \frac{2 \pi j r}{n} - \varphi \right)
            = \frac{\rho^2 + 1}{n}.
        \end{split}
    \]

    \noindent Nach Definition ist $\rho = \frac{\abs{ 1 - z_0}}{\abs{z_j - z_0}}$.
    Mit
    $\abs{1-z_0} = 2 \cdot \sin \left( \pi \frac{\delta}{n} \right)$
    und
    $\abs{z_j-z_0} = 2 \cdot \sin \left( \pi \frac{j - \delta}{n} \right)$
    folgt
    \[
        \rho = \frac{\sin \left( \pi \frac{\delta}{n} \right)}{\sin \left( \pi \frac{j - \delta}{n} \right)},
    \]
    und damit die Behauptung.
\end{proof}

Schließlich können wir die Frobeniusnorm der inversen Vandermonde-Matrix
$\Vand{\delta}^{-1}$ explizit angeben und beweisen.
\begin{theorem}
    Es gilt
    \begin{equation}
        \norm{ \Vand{\delta}^{-1} }_F^2
        = n \cdot \prod_{k=1}^{n-1} \frac{1}{4 \cdot \sin^2 \left( \pi \frac{k-\delta}{n} \right)}
          + \frac{\sin^2 \left( \frac{\pi \delta}{n} \right)}{n} \cdot \sum_{k=1}^{n-1} \frac{1}{\sin^2 \left( \pi \frac{k-\delta}{n} \right)}
          + \frac{n-1}{n}.
    \end{equation}
\end{theorem}
\begin{proof}
    Die Behauptung folgt sofort mit Hilfe der Lemmata
    \ref{lemma:inverse_outlier_zero_row_quadratic_sum} und
    \ref{lemma:inverse_outlier_nonzero_row_quadratic_sum}:
    \[
        \begin{split}
            \norm{ \Vand{\delta}^{-1} }_F^2
            &= \sum_{j=0}^{n-1} \sum_{r=0}^{n-1} \abs{u_{jr}}^2
             = \sum_{r=0}^{n-1} \abs{u_{0r}}^2
             + \sum_{j=1}^{n-1} \sum_{r=0}^{n-1} \abs{u_{jr}}^2\\
            &= n \cdot \prod_{k=1}^{n-1} \frac{1}{4 \cdot \sin^2 \left( \pi \frac{k-\delta}{n} \right)}
             + \sum_{j=1}^{n-1} \left( \frac{1}{n} \cdot \frac{\sin^2 \left( \pi \frac{\delta}{n} \right)}{\sin^2 \left( \pi \frac{j - \delta}{n} \right)}
             + \frac{1}{n} \right)\\
            &= n \cdot \prod_{k=1}^{n-1} \frac{1}{4 \cdot \sin^2 \left( \pi \frac{k-\delta}{n} \right)}
             + \frac{\sin^2 \left( \pi \frac{\delta}{n} \right)}{n} \cdot \sum_{j=1}^{n-1} \frac{1}{\sin^2 \left( \pi \frac{j - \delta}{n} \right)}
             + \frac{n-1}{n}.
         \end{split}
     \]
\end{proof}
