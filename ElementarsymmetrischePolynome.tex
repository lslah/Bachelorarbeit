\chapter{Elementarsymmetrische Polynome}

In diesem Abschnitt führen wir die \emph{elementarsymmetrischen Polynome} ein.
Diese liefern uns später eine explizite Darstellung der inversen
Vandermonde-Matrix und sind Grundlage für den Beweis einer oberen Schranke von
$\norm{V^{-1}}_\infty$.

\section{Definition und Eigenschaften}

\begin{mydef}[Elementarsymmetrische Polynome]
    Für $x = (x_1, \dots, x_{n}) \in \Cn$ definieren wir die
    \emph{k-ten elementarsymmetrischen Polynome
    $\sigma_{k}(x) = \sigma_{k}(x_1, \dots, x_{n})$, $k = 0, \dots, n$
    in den $n$ Variablen $x_j$, $j = 1, \dots, n$} durch
    \[
        \prod_{k=1}^{n} (z + x_k)
        = \sum_{k=0}^{n} \sigma_{k}(x_1, \dots, x_{n}) z^{n-k}.
    \]
\end{mydef}

\begin{notation}
    Als abkürzende Schreibweise setzen wir für gegebenes
    $x = (x_1, \dots, x_{n}) \in \Cn$
    \[
        \sigma_{k}^{j}(x) \defeq \sigma_{k}(x_1, \dots, x_{j-1}, x_{j+1}, \dots, x_{n}).
    \]
    Damit ist $\sigma_{k}^{j}(x)$ das k-te elementarsymmetrische Polynom in den
    $n-1$ Variablen $x_r$, $r \in \{ 1, \dots, n \} \setminus \{ j \}$.
\end{notation}

\begin{remark}
    Ausmultiplizieren des Polynoms und Koeffizientenvergleich liefern:
    \begin{equation}
        \label{eq:explicit_elementary_symmetric_polynomials}
        \sigma_{k}(x_1, \dots, x_{n})
        = \sum_{1 \leq j_1 < \dots < j_k \leq n} x_{j_1} \cdots x_{j_k}
        = \sum_{1 \leq j_1 < \dots < j_k \leq n} \left( \prod_{r=1}^k x_{j_r} \right).
    \end{equation}
\end{remark}

\begin{lemma}
    \label{lemma:recursion_elementary_symmetric_polynomials}
    Die elementarsymmetrischen Polynome erfüllen die Rekursionsformel:
    \begin{equation}
        \label{eq:recursion_elementary_symmetric_polynomials}
        \sigma_k (x_1, \dots, x_{n}) = \sigma_k (x_1, \dots, x_{n-1}) + x_{n} \sigma_{k-1} (x_1, \dots, x_{n-1}).
    \end{equation}
\end{lemma}
\begin{proof}
    Wir verwenden Gleichung (\ref{eq:explicit_elementary_symmetric_polynomials})
    und teilen die rechte Seite in zwei Gruppen von Summanden, je nachdem, ob
    $x_n$ ein Faktor im Summand ist:
    \begin{equation*}
        \begin{split}
            \sigma_k (x_1, \dots, x_{n})
            &= \sum_{1 \leq j_1 < \dots < j_k \leq n} x_{j_1} \cdots x_{j_k}\\
            &= \sum_{1 \leq j_1 < \dots < j_k \leq n-1} x_{j_1} \cdots x_{j_k}
            + \sum_{1 \leq j_1 < \dots < j_{k-1} \leq n-1} x_{j_1} \cdots x_{j_k} \cdot x_{n}\\
            &= \sigma_k (x_1, \dots, x_{n-1}) + x_{n} \sigma_{k-1} (x_1, \dots, x_{n-1}).
        \end{split}
    \end{equation*}
\end{proof}

\noindent Wir beweisen noch zwei technische Lemmata, die uns später nützen.

\begin{lemma}
    \label{lemma:elementary_symmetric_polynomials_const_multiplication}
    Für $x = (x_1, \dots, x_n) \in \Cn$ und $\alpha \in \C$ gilt:
    \begin{equation}
        \label{eq:elementary_symmetric_polynomials_const_multiplication}
        \sigma_{k}(\alpha x)
        = \alpha^{k} \sigma_{k}(x)
    \end{equation}
\end{lemma}
\begin{proof}
    Unter Verwendung von Gleichung
    (\ref{eq:explicit_elementary_symmetric_polynomials})
    erhalten wir:
    \[
        \sigma_{k}(\alpha x)
        = \sum_{1 \leq j_1 < \dots < j_k \leq n} \alpha x_{j_1} \cdots \alpha x_{j_k}
        = \alpha^k \sum_{1 \leq j_1 < \dots < j_k \leq n} x_{j_1} \cdots x_{j_k}
        = \alpha^k \sigma_{k}(x).
    \]
\end{proof}

\begin{lemma}
    \label{lemma:outlier_sigma_zero}
    Bezeichnen wir für $n \in \N$ die erste $n$-te Einheitswurzel mit
    $\wn \defeq e^{\frac{2 \pi i}{n}}$, so gilt
    \[
        \abs{\sigma_r(\wn^1, \dots, \wn^{n-1})} = 1
    \]
    für alle $r = 0, \dots, n-1$.
\end{lemma}
\begin{proof}
    Nach Lemma \ref{lemma:elementary_symmetric_polynomials_const_multiplication}
    gilt
    \[
        \abs{\sigma_r(-\wn^1, \dots, -\wn^{n-1})}
        = \abs{(-1)^r \sigma_r(\wn^1, \dots, \wn^{n-1})}
        = \abs{\sigma_r(\wn^1, \dots, \wn^{n-1})}.
    \]
    Nach Definition der elementarsymmetrischen Polynome ist
    $\sigma_r({-\wn^1, \dots, -\wn^{n-1}})$
    der ${n\!-\!r\!-\!1}$-te Koeffizient des Polynoms vom
    Grad $n-1$, das eindeutig durch die $n-2$ Nullstellen
    $\wn^k$, $k=1, \dots, n-1$ bestimmt ist.
    Wir zeigen, dass dies das Polynom $p(z) = \sum_{r=0}^{n-1} z^r$ ist.

    \noindent Tatsächlich ist $\wn^k = e^{2 \pi i k / n}$ für $k = 1, \dots, n-1$
    eine Nullstelle von $p(z)$, denn wegen $\wn^k \neq 1$ folgt mit Hilfe der
    geometrischen Reihe
    \[
        \begin{split}
            p(\wn^k)
            &= \sum_{r=0}^{n-1} \left( \wn^k \right)^r
            = \frac{1 - \left( \wn^k \right)^n}{1 - \left( \wn^k \right)}
            = \frac{1 - \left( \wn^n \right)^k}{1 - \left( \wn^k \right)}
            = 0.
        \end{split}
    \]
    Damit ist gezeigt, dass für $r=0, \dots, n-1$
    \[
        \abs{\sigma_r(\wn^1, \dots, \wn^{n-1})}
        = \abs{\sigma_r(-\wn^1, \dots, -\wn^{n-1})}
        = 1
    \]
    gilt.
\end{proof}

\begin{corollary}
    \label{corollary:outlier_sigma_zero_row_sum}
    Es gilt
    \[
        \sum_{r=0}^{n-1} \abs{\sigma_r(\wn^1, \dots, \wn^{n-1})} = n.
    \]
\end{corollary}

\section[Abschätzung der Betragssumme der elementarsymmetrischen Polynome]{Eine Abschätzung von \boldmath $\sum_{k=0}^{n} \abs{\sigma_{k}}$}

Die folgende Abschätzung liefert uns im nächsten Abschnitt die obere Schranke
der Zeilensummennorm der inversen Vandermonde-Matrix.
\begin{lemma}[\cite{gautschi4}]
    \label{lemma:elementary_symmetric_polynomials_inequality}
    Sei $x = (x_1, \dots, x_n) \in \Cn$ wie zuvor.
    Dann gilt:
    \begin{equation}
        \label{eq:elementary_symmetric_polynomials_inequality}
        \sum_{k=0}^{n} \abs{\sigma_{k}(x)} \leq \prod_{k=1}^{n} \left( 1 + \abs{x_k} \right)
    \end{equation}
    Gleichheit gilt, wenn $x_k = r_k \cdot e^{i\varphi}$ für ein
    festes $\varphi \in \R$ und beliebige $r_k \in \R_+$,
    d.h. wenn alle $x_k$ auf einer Halbgeraden durch den Nullpunkt
    in der komplexen Ebene liegen.
\end{lemma}

\begin{remark}
    Für die letzte Aussage des Lemmas, lässt sich sogar Äquivalenz zeigen, d.h.
    Gleichheit gilt \emph{genau dann}, wenn $x_k = r_k \cdot e^{i\varphi}$ für
    ein festes $\varphi \in \R$ und beliebige $r_k \in \R_+$.
    Da diese Aussage jedoch für den Rest der Arbeit nicht von Bedeutung ist,
    beweisen wir nur die schwächere Form des Lemmas.
\end{remark}

\begin{proof}[Beweis von Lemma \ref{lemma:elementary_symmetric_polynomials_inequality}]
    Zum Beweis, wenden wir vollständige Induktion über $n \in \N$ und
    Lemma \ref{lemma:recursion_elementary_symmetric_polynomials} an.\\[0.5em]
    \textbf{Induktionsanfang ($n=1$):}
    Es ist $ (z+x_0) = \sigma_0(x_0) \cdot z^1 + \sigma_1(x_0) \cdot z^0$,
    also $\sigma_0(x_0) = 1$ und $\sigma_1(x_0) = x_0$.
    Damit ergibt sich: $\sum_{k=0}^{1} \abs{\sigma_k(x_0)} = \abs{1} + \abs{x_0} = 1 + \abs{x_0}$.\\[0.5em]
%
    \noindent \textbf{Induktionvoraussetzung:}
    Sei die Behauptung für $n-1 \in \N$ erfüllt.\\[0.5em]
%
    \noindent \textbf{Induktionsschritt ($n-1 \rightarrow n$):}
    Unter Verwendung von $\sigma_{0}(x_1, \dots, x_n) = 1$
    und $\sigma_{n}(x_1, \dots, x_n) = x_1 \cdots x_n$, folgt:
    \begin{equation*}
      \begin{split}
        \prod_{k=1}^{n} \left( 1 + \abs{x_k} \right)
        &= ( 1 + \abs{x_n} ) \cdot \prod_{k=1}^{n-1} \left( 1 + \abs{x_k} \right)\\
        &\overset{IV}{\geq}  ( 1 + \abs{x_n} ) \cdot \sum_{k=0}^{n-1} \abs{\sigma_{k}(x_1, \dots, x_{n-1})}\\
        &= \sum_{k=0}^{n-1} \abs{\sigma_{k}(x_1, \dots, x_{n-1})}
        + \sum_{k=1}^{n} \abs{x_n \cdot \sigma_{k-1}(x_1, \dots, x_{n-1})}\\
        &\geq \sum_{k=1}^{n-1} \abs{\sigma_{k}(x_1, \dots, x_{n-1}) + x_n \cdot \sigma_{k-1}(x_1, \dots, x_{n-1})} + \abs{\sigma_{0}(x_1, \dots, x_{n-1})}\\
        &+ \abs{x_n \cdot \sigma_{n-1}(x_1, \dots, x_{n-1})}\\
        &\overset{(\ref{eq:recursion_elementary_symmetric_polynomials})}{=}
        \sum_{k=1}^{n-1} \abs{\sigma_{k}(x_1, \dots, x_{n})} + \abs{\sigma_{0}(x_1, \dots, x_{n})} + \abs{\sigma_{n}(x_1, \dots, x_n)}\\
        &= \sum_{k=0}^{n} \abs{\sigma_{k}(x_1, \dots, x_{n-1})}.
      \end{split}
    \end{equation*}

    Um den Spezialfall $x_k = r_k e^{i \varphi}$ der Behauptung zu beweisen,
    führen wir die Induktion analog, mit der neuen Induktionsbehauptung, dass
    Gleichheit gelte.  Es bleibt nur zu zeigen, dass die zweite Abschätzung
    oben zur Gleichheit wird, d.h. dass
    $\sigma_{k}(x_1, \dots, x_{n-1})$ und $x_n \cdot \sigma_{k-1}(x_1, \dots, x_{n-1})$
    linear abhängig sind und die Dreiecksungleichung als Gleichung
    \[
        \abs{\sigma_{k}(x_1, \dots, x_{n-1})} + \abs{x_n \cdot \sigma_{k-1}(x_1, \dots, x_{n-1})}
        = \abs{\sigma_{k}(x_1, \dots, x_{n-1}) + x_n \cdot \sigma_{k-1}(x_1, \dots, x_{n-1})}
    \]
    gilt.
    Dazu verwenden wir Gleichung
    (\ref{eq:explicit_elementary_symmetric_polynomials}) und stellen fest, dass
    \begin{equation*}
        \begin{split}
            \sigma_{k}(x_1, \dots, x_{n-1})
            &= \sum_{1 \leq j_1 < \dots < j_k \leq n-1} x_{j_1} \cdots x_{j_k}\\
            &= \sum_{1 \leq j_1 < \dots < j_k \leq n-1} r_{j_1} e^{i\varphi} \cdots r_{j_k} e^{i\varphi}\\
            &= e^{k\cdot i\varphi} \cdot \left( \sum_{1 \leq j_1 < \dots < j_k \leq n-1} r_{j_1} \cdots r_{j_k} \right)
        \end{split}
    \end{equation*}
    und
    \begin{equation*}
        \begin{split}
            x_n \cdot \sigma_{k-1}(x_1, \dots, x_{n-1})
            &= \sum_{1 \leq j_1 < \dots < j_{k-1} \leq n-1} x_n \cdot x_{j_1} \cdots x_{j_{k-1}}\\
            &= \sum_{1 \leq j_1 < \dots < j_{k-1} \leq n-1} r_n e^{i\varphi} \cdot r_{j_1} e^{i\varphi} \cdots r_{j_{k-1}} e^{i\varphi}\\
            &= e^{k\cdot i\varphi} \cdot \left( \sum_{1 \leq j_1 < \dots < j_{k-1} \leq n-1} r_n \cdot r_{j_1} \cdots r_{j_{k-1}} \right)
        \end{split}
    \end{equation*}
    in dieselbe Richtung $e^{k\cdot i\varphi}$ zeigen, also linear abhängig sind, wie behauptet.
\end{proof}
