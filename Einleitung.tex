\chapter{Einleitung}

% Reeller Teil: Polynominterpolation -> Koeffizientenmatrix = Vandermonde-Matrix -> schlecht konditioniert
% Komplexer Teil: Im Gegensatz zum reellen Fall gut konditioniert. Reihe von Anwendungen
% Forschungsstand: Anlehnung an abstract von Gautschi (ist schon älter)
% Thema: eventuell explizite Fragestellung
% Aufbau: Grundlagen -> Ungleichung -> Reeller Fall -> Komplexer Fall -> Fazit

Schon früh im Studium der numerischen Mathematik trifft man auf das Problem
$n+1$ Punkte im $\R^2$ durch ein Polynom $n$-ten Grades zu interpolieren.  Bei
der Modellierung dieser Aufgabe mit Hilfe der Monom-Basis entsteht ein lineares
Gleichungssystem.  Die zugehörige Koeffizienten-Matrix ist nur von den
reellen Stützstellen abhängig und wird Vandermonde-Matrix genannt.  Es zeigt sich
schnell, dass die Vandermonde-Matrix in diesem Fall schlecht konditioniert ist
und damit bei der Lösung des Interpolationsproblems große numerische Fehler
hervorruft.

Anders verhält es sich, wenn man Vandermonde-Matrizen mit Stützstellen in der
komplexen Ebene betrachtet.  So ist die Vandermonde-Matrix zu den $n$-ten
Einheitswurzeln sogar perfekt konditioniert bezüglich der Spektralnorm.  In
einer Reihe von Anwendungen entstehen Vandermonde-Matrizen, die durch Knoten
auf dem Einheitskreis definiert sind.

Forschungsergebnisse zur Konditionszahl von Vandermonde-Matrizen mit reellen
und komplexen Stützstellen wurden 1990 in \cite{gautschi1} zusammengefasst.
Seither gab es kaum neue Erkenntnisse auf diesem Gebiet, obgleich
diesbezügliche Fragestellungen von hoher Aktualität zur Entwicklung
verbesserter numerischer Algorithmen für die Prony-Methode sind.
Wie etwa verändert sich die Kondition der Vandermonde-Matrix bei Verteilungen
die von den Einheitswurzeln abweichen?  Was passiert, wenn alle Stützstellen
bis auf einen äquidistant auf dem Einheitskreis liegen?
Diese Fragen sollen in der vorliegenden Arbeit untersucht werden.

Nachdem zunächst einige grundlegende Begriffe eingeführt werden, folgt eine
Ungleichung für die Kondition bzgl. der Zeilensummennorm von
Vandermonde-Matrizen mit beliebigen Knoten.  Diese wird anschließend zur
Berechnung der Kondition einiger spezieller Vandermonde-Matrizen mit reellen
Knoten verwendet.  Im letzten Abschnitt wird schließlich eine Konfiguration von
Stützstellen auf dem Einheitskreis untersucht, welche leicht vom äquidistanten
Fall der $n$-ten Einheitswurzeln abweicht. Die Ergebnisse werden schließlich im
Fazit zusammengefasst.
