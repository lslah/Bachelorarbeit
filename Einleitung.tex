\chapter{Einleitung}

Schon früh im Studium der numerischen Mathematik trifft man bei der reellen
Polynominterpolation auf Vandermonde-Matrizen.  Man stellt schnell fest, dass
diese dort nur für theoretische Betrachtungen sinnvoll sind.  Für die
praktische Berechnung der Koeffizienten des Interpolationspolynoms wird auf
andere Methoden als die Invertierung der Vandermonde-Matrix zurückgegriffen, da
sich diese Berechnung als schlecht konditioniert herausstellt und damit große
numerische Fehler hervorruft.

Anders verhält es sich wenn man Vandermonde-Matrizen mit Knoten in der
komplexen Ebene betrachtet.  So ist die Vandermonde-Matrix zu den $n$-ten
Einheitswurzeln sogar perfekt konditioniert bezüglich der Spektralnorm.
In einer Reihe von Anwendungen entstehen Vandermonde-Matrizen, die durch Knoten
auf dem Einheitskreis definiert sind, die teilweise nur leicht von der
äquidistanten Verteilung abweichen.

In dieser Arbeit werden Vandermonde-Matrizen sowohl mit reellen als auch
komplexen Knoten untersucht.
Nachdem zunächst einige grundlegende Begriffe wiederholt oder neu eingeführt
werden, folgt eine Ungleichung für die Kondition bzgl. der Zeilensummennorm von
Vandermonde-Matrizen mit belibiebigen Knoten.
Diese wird anschließend zur Berechnung der Kondition einiger spezieller
Vandermonde-Matrizen mit reellen Knoten verwendet.
Im letzten Abschnitt wird schließlich eine Konfiguration von Knoten auf dem
Einheitskreis untersucht, welche leicht vom äquidistanten Fall der $n$-ten
Einheitswurzeln abweicht.
Explizite Formeln für die Berechnung der Konditionszahlen bzgl. Zeilensummen-
und Frobeniusnorm der zugehörigen Vandermonde-Matrizen werden hergeleitet.
