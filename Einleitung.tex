\chapter{Einleitung}

Schon früh im Studium der numerischen Mathematik trifft man bei der
reellen Polynominterpolation auf Vandermonde-Matrizen.
Man stellt schnell fest, dass diese dort nur für theoretische Betrachtungen
sinnvoll sind.
Für die praktische Berechnung der Koeffizienten des Interpolationspolynoms
wird auf andere Methoden als die Invertierung der Vandermonde-Matrix
zurückgegriffen, da sich diese Berechnung als schlecht konditioniert herausstellt und damit große numerische Fehler hervorruft.

Anders verhält es sich wenn man Vandermonde-Matrizen mit Knoten in der
komplexen Ebene betrachtet.  So ist die Vandermonde-Matrix zu den $n$-ten
Einheitswurzeln sogar optimal konditioniert bezüglich der Spektralnorm.
In einer Reihe von Anwendungen entstehen Vandermonde-Matrizen, die durch Knoten
auf dem Einheitskreis definiert sind, die teilweise nur leicht von der
äquidistanten Verteilung abweichen.

In dieser Arbeit wird eine spezielle Art von Vandermonde-Matrizen untersucht,
bei der $n-1$ Knoten ihre optimale Position als $n$-te Einheitswurzel einnehmen
und der zur Einheitswurzel $1$ gehörige Knoten um einen Winkel $\delta \in
[\frac{2 \pi}{n})$ ausgelenkt wird.
Ziel ist es, explizite geschlossene Formeln für die Kondition dieser
Vandermonde-Matrizen bzgl. der Frobeniusnorm und der Zeilensummennorm zu
finden.
