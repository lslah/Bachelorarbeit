
\chapter{Vandermonde-Matrizen mit reellen Stützstellen}
Als Anwendung von Satz \ref{thm:inverse_vandermonde_inequality} werden wir nun
die Kondition einer Vandermonde-Matrix zu nicht-negativen harmonischen Stützstellen
berechnen.
Weitere Konditionen von Vandermonde-Matrizen mit reellen Stützstellen werden
angegeben.
Die Beispiele wurden dabei aus \cite[S. 197-199]{gautschi1} entnommen.

Zunächst beweisen wir ein allgemeineres Lemma über die Zeilensummennorm
der Vandermonde-Matrix mit Stützstellen innerhalb des komplexen
Einheitskreises.
Betrachtet man nämlich nur Stützstellen $z_j \in \C$ mit $\abs{z_j} \leq 1$,
so gilt $\norm{\Vand{z}}_{\infty} = n$.
Somit muss in diesem Fall zur Bestimmung von $\cond[\infty]{\Vand{z}}$ nur noch die
Norm der Inversen $\norm{\Vand{z}^{-1}}_{\infty}$ berechnet werden.
Insbesondere gilt dies für reelle Stützstellen
$x = (x_0, \dots, x_{n-1}) \in \Rn$ mit $\abs{x_j} \leq 1$.

\begin{lemma}
    \label{lemma:infty_norm_vandermonde_unit_roots}
    Seien Stützstellen $z = (z_0, \dots, z_{n-1}) \in \Cn$
    mit $\abs{z_j} \leq 1$ für alle\linebreak
    $j=0,\dots,n-1$ gegeben.
    Dann gilt
    \[
        \norm{V(z)}_{\infty} = n.
    \]
\end{lemma}

\begin{proof}
    Wegen $\abs{z_j^k} \leq \abs{z_j^r}$ für $k > r$ und alle $j = 0, \dots, n-1$, folgt
    \[
        \norm{V(z)}_{\infty}
        = \max_{k=0,\dots,n-1} \left( \sum_{j=0}^{n-1} \abs{z_j^k} \right)
        = \sum_{j=0}^{n-1} \abs{z_j^0}
        = n.
    \]
\end{proof}

\section{Nicht-negative Stützstellen}
Betrachte die Stützstellen $z_j = x_j \in \R_+$ für $j = 0, \dots, n-1$
mit $x_k \neq x_j$ für $k \neq j$.
Diese liegen auf einer Halbgeraden durch den Ursprung und erfüllen damit die
Zusatzbedingung von Satz \ref{thm:inverse_vandermonde_inequality},
so dass bei der oberen Schranke von (\ref{eq:inverse_vandermonde_inequality})
Gleichheit gilt.
Mit Hilfe dieses Satzes können wir also $\norm{V^{-1}}_\infty$ explizit berechnen.
\begin{lemma}
    \label{lemma:nonnegative_real_nodes}
    Ist $V$ die zu den Stützstellen $x_j \in \R$, $x_j \geq 0$ gehörige
    Vandermone-Matrix und definiert man
    \[
        p(z) \defeq \prod_{j=0}^{n-1} \left( z - x_j \right),
    \]
    so folgt
    \[
        \norm{V^{-1}}_{\infty} = \frac{ \abs{p(-1)}}{\min_{j=0, \dots, n-1} \{ (1+x_j) \abs{p^{\prime}(x_j)} \}}.
    \]
\end{lemma}

\begin{proof}
    Es gilt
    \[
        \abs{p(-1)} = \abs{\prod_{j=0}^{n-1} (-1 - x_j)} = \prod_{j=0}^{n-1} \left( 1 + x_j \right).
    \]
    Weiter ist nach der Produktregel
    \[
        p^{\prime}(z) = \sum_{k=0}^{n-1} \left( \prod_{\substack{j=0\\ j\neq k}}^{n-1} \left( z - x_j \right) \right),
    \]
    also
    \[
        \abs{p^{\prime}(x_k)} = \prod_{\substack{j=0\\ j\neq k}}^{n-1} \abs{ x_k - x_j }.
    \]
    Insgesamt folgt mit Satz \ref{thm:inverse_vandermonde_inequality}
    \[
        \begin{split}
            \frac{ \abs{p(-1)}}{\min_{j=0, \dots, n-1} \{ (1+x_j) \abs{p^{\prime}(x_j)} \}}
            = \frac{\prod_{k=0}^{n-1} \left( 1 + x_k \right)}{\min_{j=0, \dots, n-1} \{ (1+x_j) \prod_{\substack{k=0\\ k\neq j}}^{n-1} \abs{ x_j - x_k }\}}\\
            = \max_{j=0, \dots, n-1} \left( \prod_{\substack{k=0\\ k \neq j}}^{n-1} \frac{\left( 1 + x_k \right)}{\abs{ x_j - x_k }} \right)
            \overset{(\ref{eq:inverse_vandermonde_inequality})}{=} \norm{\Vand{x}^{-1}}_{\infty}.
        \end{split}
    \]
\end{proof}

\begin{example}[Harmonische Stützstellen]
    Seien die Stützstellen $x_k = \frac{1}{k}, \; k=1, \dots, n$ gegeben.
    Mit den Bezeichnungen wie in \lemmaref{nonnegative_real_nodes} gilt dann
    \[
        \abs{p(-1)} = \prod_{k=1}^{n} \left( 1 + \frac{1}{k} \right) = n+1,
    \]
    wie eine einfache Induktion zeigt.
    Für $\delta_k \defeq (1 + x_k) \abs{ p^{\prime}(x_k) }$ gilt
    \[
        \begin{split}
            \delta_k &= \left( 1 + \frac{1}{k} \right) \prod_{\substack{r = 1\\ r\neq k}}^n \abs{ \frac{1}{k} - \frac{1}{r} }
                     = \left( \frac{k+1}{k} \right) \prod_{\substack{r = 1\\ r\neq k}}^n \abs{ \frac{r-k}{rk}}\\
                     &= \left( \frac{k+1}{k^n} \right) \frac{k}{n!} \prod_{\substack{r = 1\\ r\neq k}}^n \abs{ r-k }
                     = \left( \frac{k+1}{k^n} \right) \frac{k}{n!} (n-k)! \, (k-1)!\\
                     &= \left( \frac{(k+1)! \, (n-k)!}{k^n n!} \right).
        \end{split}
    \]
    Weiter folgt dann
    \[
        \min_{k=1, \dots, n} \delta_k \leq \delta_n = \frac{n+1}{n^n}
    \]
    und schließlich mit Lemma \ref{lemma:nonnegative_real_nodes}:
    \[
        \cond[\infty]{V} = \norm{V}_\infty \cdot \norm{V^{-1}}_\infty \geq n \cdot (n+1) \frac{n^n}{n+1} = n^{n+1}.
    \]
\end{example}

\begin{example}[Äquidistante Stützstellen \cite{gautschi3}]
    Betrachte die Stützstellen $x_k = \frac{k-1}{n-1}, \; k=1, \dots, n$.
    Ähnliche Untersuchungen wie im vorherigen Beispiel liefern die
    asymptotische Formel
    \[
        \cond[\infty]{\Vand{x}} \sim \frac{\sqrt{2}}{4\pi} \cdot 8^n \text{ für } n \rightarrow \infty.
    \]
    Für eine ausführliche Herleitung sei auf \cite[S. 344f]{gautschi3}
    verwiesen.
\end{example}

\section{Symmetrische Stützstellen}
Es zeigt sich, dass die Kondition der Vandermonde-Matrix verbessert werden
kann, wenn man die Stützstellen symmetrisch um den Ursprung anordnet.

Analog zu \lemmaref{nonnegative_real_nodes} lässt sich für den Fall
symmetrischer reeller Stützstellen, die folgende Aussage zeigen.
Der Beweis kann in \cite[S. 341]{gautschi3} nachgelesen werden.
\begin{lemma}[\cite{gautschi3}, ohne Beweis]
    \label{lemma:symmetric_real_nodes}
    Sei $x = (x_1, \dots, x_n) \in \Rn$ mit $x_j + x_{n+1-j} = 0$ für $j = 1, \dots, n$.
    Dann gilt
    \[
        \norm{\Vand{x}^{-1}}_\infty = \frac{\abs{p(i)}}{\min_{x_k \geq 0} \left\{ \frac{1 + x_k^2}{1 + x_k} \abs{p^{\prime}(x_k)} \right\} },
    \]
    mit $p(x) \defeq \prod_{k=1}^{n} (x - x_k)$.
\end{lemma}

\begin{example}[Äquidistante, symmetrische Stützstellen \cite{gautschi1}]
    Seien $x_k = 1 - \frac{2(k-1)}{n-1}$ für ${k = 1, \dots, n}$.
    Dann ergibt sich unter Verwendung von \lemmaref{symmetric_real_nodes}
    \[
        \cond[\infty]{\Vand{x}} \sim \frac{1}{\pi} e^{-\frac{1}{4} \pi} e^{n\left( \frac{1}{4} \pi + \frac{1}{2} \ln 2 \right)} \text{ für } n \rightarrow \infty.
    \]
\end{example}

Anstelle der exponentiellen Wachstumsrate von $8$ im nicht-negativen Fall, wird mit
symmetrischen, äquidistanten Stützstellen eine Wachstumsrate von
\[
    \exp \left( \frac{1}{4} \pi + \frac{1}{2} \ln 2 \right) = 3.1017\dots
\]
erreicht.
Die Kondition kann noch weiter verbessert werden, wenn die sogenannten
\emph{Tschebycheff Knoten} verwendet werden.

\begin{example}[Tschebyscheff Stützstellen \cite{gautschi3}]
    Wir betrachten die Knoten $x_k = \cos \left( \frac{(2k-1) \pi}{2n} \right)$
    für $k=1, \dots, n$.
    Unter Anwendung von \lemmaref{symmetric_real_nodes} lässt sich die
    asymptotische Formel
    \[
        \cond[\infty]{\Vand{x}} \sim \frac{3^{\frac{3}{4}}}{4} \left( 1 + \sqrt{2} \right)^n \text{ für } n \rightarrow \infty
    \]
    beweisen.
    Die Wachstumsrate beträgt hier $1 + \sqrt{2} = 2.4142..$.
\end{example}


