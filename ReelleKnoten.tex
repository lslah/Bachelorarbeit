
\chapter{Vandermonde-Matrizen mit reellen Stützstellen}
Als Anwendung von Satz \ref{thm:inverse_vandermonde_inequality} werden wir nun
einige ausgewählte Beispiele für Vandermonde-Matrizen mit reellen Stützstellen
betrachten und deren Kondition berechnen.
Die Beispiele wurden dabei aus \cite[S. 197-199]{gautschi1} entnommen.

Zunächst beweisen wir noch ein allgemeineres Lemma über die Zeilensummennorm
der Vandermonde-Matrix mit Stützstellen innerhalb des komplexen
Einheitskreises.
Betrachtet man nämlich nur Stützstellen $z_j \in \C$ mit $\abs{z_j} \leq 1$,
so gilt $\norm{\Vand{z}}_{\infty} = n$.
Somit muss in diesem Fall zur Bestimmung von $\cond[\infty]{\Vand{z}}$ nur noch die
Norm der Inversen $\norm{\Vand{z}^{-1}}_{\infty}$ berechnet werden.
Insbesondere gilt dies für reelle Stützstellen
$x = (x_0, \dots, x_{n-1}) \in \Rn$ mit $\abs{x_j} \leq 1$.

\begin{lemma}
    \label{lemma:infty_norm_vandermonde_unit_roots}
    Seien Stützstellen $z = (z_0, \dots, z_{n-1}) \in \Cn$
    mit $\abs{z_j} \leq 1$ für alle\linebreak
    $j=0,\dots,n-1$ gegeben.
    Dann gilt
    \[
        \norm{V(z)}_{\infty} = n.
    \]
\end{lemma}

\begin{proof}
    Wegen $\abs{z_j^k} \leq \abs{z_j^r}$ für $k > r$ und alle $j = 0, \dots, n-1$, folgt
    \[
        \norm{V(z)}_{\infty}
        = \max_{k=0,\dots,n-1} \left( \sum_{j=0}^{n-1} \abs{z_j^k} \right)
        = \sum_{j=0}^{n-1} \abs{z_j^0}
        = n.
    \]
\end{proof}

\section{Nicht-negative Stützstellen}
Betrachte die Stützstellen $z_j = x_j \in \R_+$ für $j = 0, \dots, n-1$
und $x_k \neq x_j$ für $k \neq j$.
Diese erfüllen die Zusatzbedingung von Satz \ref{thm:inverse_vandermonde_inequality},
so dass bei der oberen Schranke von (\ref{eq:inverse_vandermonde_inequality})
Gleichheit gilt.
\begin{lemma}
    \label{lemma:nonnegative_real_nodes}
    Ist $V$ die zu den Stützstellen $x_j \in \R$, $x_j \geq 0$ gehörige
    Vandermone-Matrix und definiert man
    \[
        p(z) \defeq \prod_{j=0}^{n-1} \left( z - x_j \right),
    \]
    so folgt
    \[
        \norm{V^{-1}}_{\infty} = \frac{ \abs{p(-1)}}{\min_{j=0, \dots, n-1} \{ (1+x_j) \abs{p^{\prime}(x_j)} \}}.
    \]
\end{lemma}

\begin{proof}
    Es gilt
    \[
        \abs{p(-1)} = \abs{\prod_{j=0}^{n-1} (-1 - x_j)} = \prod_{j=0}^{n-1} \left( 1 + x_j \right).
    \]
    Weiter ist nach der Produktregel
    \[
        p^{\prime}(z) = \sum_{k=0}^{n-1} \left( \prod_{\substack{j=0\\ j\neq k}}^{n-1} \left( z - x_j \right) \right),
    \]
    also
    \[
        \abs{p^{\prime}(x_k)} = \prod_{\substack{j=0\\ j\neq k}}^{n-1} \abs{ x_k - x_j }.
    \]
    Insgesamt folgt mit Satz \ref{thm:inverse_vandermonde_inequality}
    \[
        \begin{split}
            \frac{ \abs{p(-1)}}{\min_{j=0, \dots, n-1} \{ (1+x_j) \abs{p^{\prime}(x_j)} \}}
            = \frac{\prod_{k=0}^{n-1} \left( 1 + x_k \right)}{\min_{j=0, \dots, n-1} \{ (1+x_j) \prod_{\substack{k=0\\ k\neq j}}^{n-1} \abs{ x_j - x_k }\}}\\
            = \max_{j=0, \dots, n-1} \left( \prod_{\substack{k=0\\ k \neq j}}^{n-1} \frac{\left( 1 + x_k \right)}{\abs{ x_j - x_k }} \right)
            \overset{(\ref{eq:inverse_vandermonde_inequality})}{=} \norm{\Vand{x}^{-1}}_{\infty}.
        \end{split}
    \]
\end{proof}

\begin{example}[Harmonische Stützstellen]
    Seien die Stützstellen $x_k = \frac{1}{k}, \; k=1, \dots, n$ gegeben.
    Mit den Bezeichnungen wie in Lemma \ref{lemma:nonnegative_real_nodes} gilt dann
    \[
        \abs{p(-1)} = \prod_{k=1}^{n} \left( 1 + \frac{1}{k} \right) = n+1,
    \]
    wie eine einfache Induktion zeigt.
    Für $\delta_k \defeq (1 + x_k) \abs{ p^{\prime}(x_k) }$ gilt
    \[
        \begin{split}
            \delta_k &= \left( 1 + \frac{1}{k} \right) \prod_{\substack{r = 1\\ r\neq k}}^n \abs{ \frac{1}{k} - \frac{1}{r} }
                     = \left( \frac{k+1}{k} \right) \prod_{\substack{r = 1\\ r\neq k}}^n \abs{ \frac{r-k}{rk}}\\
                     &= \left( \frac{k+1}{k^n} \right) \frac{k}{n!} \prod_{\substack{r = 1\\ r\neq k}}^n \abs{ r-k }
                     = \left( \frac{k+1}{k^n} \right) \frac{k}{n!} (n-k)! \, (k-1)!\\
                     &= \left( \frac{(k+1)! \, (n-k)!}{k^n n!} \right).
        \end{split}
    \]
    Weiter folgt dann
    \[
        \min_{k=1, \dots, n} \delta_k \leq \delta_n = \frac{n+1}{n^n}
    \]
    und schließlich mit Lemma \ref{lemma:nonnegative_real_nodes}:
    \[
        \cond{V} = \norm{V}_\infty \cdot \norm{V^{-1}}_\infty \geq n \cdot (n+1) \frac{n^n}{n+1} = n^{n+1}.
    \]
\end{example}

\begin{example}[Äquidistante Stützstellen]
    Betrachte die Stützstellen $x_k = \frac{k-1}{n-1}, \; k=1, \dots, n$.
    Dann ist
    \[
        \begin{split}
            p(-1)
            &= \prod_{k=1}^{n} \left( 1 + \frac{k-1}{n-1} \right)
            = \prod_{k=1}^{n} \left( \frac{n + k - 2}{n-1} \right)\\
            &= \frac{1}{(n-1)^n} \frac{(2n-2)!\,}{(n-2)!\,}
            = \frac{(2n-2)!\,}{(n-1)^{n-1}(n-1)!\,}.
        \end{split}
    \]
    Mit $\delta_k \defeq (1 + x_k) \abs{ p^{\prime}(x_k) }$ gilt
    \[
        \begin{split}
            \delta_k
            &= \left( 1 + \frac{k-1}{n-1} \right) \cdot \prod_{\substack{j=1\\ j\neq k}}^{n} \abs{ \frac{k-j}{n-1} }
            = \frac{n+k-2}{(n-1)^{n}} \cdot \prod_{\substack{j=1\\ j\neq k}}^{n} \abs{ k-j }\\
            &= \frac{(n+k-2)(k-1)!\, (n-k)!\,}{(n-1)^{n}} .
        \end{split}
    \]
    ...\\ %TODO
    Insgesamt folgt
    \[
        \cond{\Vand{x}} \sim \frac{\sqrt{2}}{4\pi} \cdot 8^n \text{ für } n \rightarrow \infty.
    \]
\end{example}

\section{Symmetrische Stützstellen}
In diesem Abschnitt seien nun reelle Stützstellen gegeben, die symmetrisch um
den Ursprung angeordnet sind.

\begin{lemma}[ohne Beweis]
    Sei $x = (x_1, \dots, x_n) \in \Rn$ mit $x_j + x_{n+1-j} = 0$ für $j = 1, \dots, n$.
    Dann gilt
    \[
        \norm{\Vand{x}^{-1}} = \frac{\abs{p(i)}}{\min_{x_k \geq 0} \left\{ \frac{1 + x_k^2}{1 + x_k} \abs{p^{\prime}(x_k)} \right\} }.
    \]
\end{lemma}
