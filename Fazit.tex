\chapter{Fazit}

In der vorliegenden Arbeit wurden Vandermonde-Matrizen mit reellen und
komplexen Knoten untersucht.
Die Betrachtung von Konfigurationen mit Knoten auf der reellen Achse
bestätigte den Ruf von Vandermonde-Matrizen, schlecht konditioniert zu sein.
So konnten nur Anordnungen von $n$ reellen Knoten gefunden werden, bei denen die Kondition
bezüglich der Zeilensummennorm mindestens exponentiell in $n$ wächst.

Im Gegensatz dazu konnte gezeigt werden, dass Vandermonde-Matrizen zur
äquidistanten Verteilung der Knoten auf dem komplexen Einheitskreis nahezu
perfekt bezüglich der Frobenius- und der Zeilensummennorm konditioniert sind.
Die Konditionszahl wächst in diesem Fall nur linear in $n$.

Abschließend wurden Verteilungen untersucht, bei denen ein Knoten von der
Position als $n$-te Einheitswurzel abweicht.  Für diese Konfiguration
lieferten theoretische Betrachtungen die Vermutung über eine explizite Formel
der Konditionszahl in Bezug auf die Zeilensummennorm.  Numerische
Untersuchungen konnten diese Vermutung stützen.  Schließlich wurde für die
Kondition bezüglich der Frobeniusnorm eine explizite Formel aufgestellt und
bewiesen werden.

Diese Ergebnisse können nun hilfreich für die Entwicklung effizienter
Algorithmen bei der Prony-Methode eingebracht werden.
